\chapter{Graphs}
\chaplabel{graphs}

In this chapter, we study two representations of graphs and basic
algorithms on these representations.  Mathematically, a \emph{(directed)
graph} is a pair $G=(V,E)$ where $V$ is a set of \emph{vertices}
and $E$ is a set of ordered pairs of vertices called \emph{edges}.

An edge #(i,j)# is \emph{directed} from #i# to #j#;  #i# is called
the \emph{source} of the edge and #j# is called the \emph{target}.
A \emph{path} in $G$ is a sequence of vertices $v_0,\ldots,v_k$ such that,
for every $i\in\{1,\ldots,k\}$, the edge $(v_{i-1},v_{i})$ is in $E$.
A path $v_0,\ldots,v_k$ is a cycle if, additionally, the edge $(v_k,v_0)$
is in $E$.  A path (or cycle) is \emph{simple} if all of its vertices
are unique.  If there is a path from some vertex $v_i$ to some vertex
$v_j$ then we say that $v_j$ is \emph{reachable} from $v_i$.

Graphs have an enormous number of applications, due to their ability
to model so many phenomenon. There are many obvious examples. Computer
networks can be modelled as graphs, with vertices corresponding to
computers and edges corresponding to (directed) communication links
between those computers.  Street networks can be modelled as graphs,
with vertices representing intersections and edges representing streets
joining consecutive intersections.

Less obvious examples occur as soon as we realize that graphs can model
any pairwise relationships within a set. For example, in a university
setting we might have a timetable conflict graph whose vertices represent
courses offered in the university and in which the edge #(i,j)# is present
if and only if there is at least one student that is taking both class
#i# and class #j#.  Thus, an edge indicates that the exam for class #i#
can not be scheduled at the same time as the exam for class #j#.

Throughout this section, we will use #n# to denote the number of vertices
of $G$ and #m# to denote the number of edges of $G$.  That is, $#n#=|V|$
and $#m#=|E|$. Furthermore, we will assume that $V=\{0,\ldots,#n#-1\}$.
Any other data that we would like to associate with the elements of $V$
can be stored in an array of length $#n#$.

The typical operations we would like to perform on graphs are:
\begin{itemize}
  \item #addEdge(i,j)#: Add the edge $(#i#,#j#)$ to $E$.
  \item #removeEdge(i,j)#: Remove the edge $(#i#,#j#)$ from $E$.
  \item #hasEdge(i,j)#: Check if the edge $(#i#,#j#)\in E$ 
  \item #outEdges(i)#: Return a #List# of all integers $#j#$ such that
  $(#i#,#j#)\in E$
  \item #inEdges(i)#: Return a #List# of all integers $#j#$ such that
  $(#j#,#i#)\in E$
\end{itemize}

Note that these operations are not terribly difficult to implement
efficiently. For example, the first three operations can be implemented
directly by using a #USet#, so they can be implemented in constant
expected time using the hash tables discussed in \chapref{hashing}.
The last two operations can be implemented in constant time by storing,
for each vertex, a list of its adjacent vertices.

However, different applications of graphs need different performance
requirements from these operations and, ideally, we can use the simplest
implementation that satisfies all the application's requirements.
For this reason, we discuss two broad categories of graph representations.

\section{AdjacencyMatrix: Representing a Graph by a Matrix}
\seclabel{adjacency-matrix}

An \emph{adjacency matrix} is a way of representing a graph $G=(V,E)$ by
an $#n#\times#n#$ matrix, #a#, whose entries are boolean values.
\codeimport{ods/AdjacencyMatrix.a.n.AdjacencyMatrix(n0)}

The matrix entry #a[i][j]# is defined as
\[  #a[i][j]#= 
    \begin{cases}
      #true# & \text{if $#(i,j)#\in E$} \\
      #false# & \text{otherwise}
    \end{cases}
\]
With this representation, the #addEdge(i,j)#, #removeEdge(i,j)#, and #hasEdge(i,j)# operations just involve setting or reading the matrix entry #a[i][j]#:
\codeimport{ods/AdjacencyMatrix.addEdge(i,j).removeEdge(i,j).hasEdge(i,j)}
These operations clearly take constant time per operation.

Where the adjacency matrix performs poorly is with the #outEdges(i)# and
#inEdges(i)# operations.  To implement these, we must scan all #n#
entries in the corresponding row or column of #a# and gather up all the
indices, #j#, where #a[i][j]#, respectively #a[j][i]#, is true and return this
list of indices.
\codeimport{ods/AdjacencyMatrix.outEdges(i).inEdges(i)}
These operations clearly take $O(#n#)$ time per operation.  

Another drawback of the adjacency matrix representation is that it
is big.  It stores an $#n#\times #n#$ boolean matrix, so it requires at
least $#n#^2$ bits of memory.  The implementation here uses a matrix
of \javaonly{boolean}\cpponly{bool} values so it actually uses on the
order of $#n#^2$ bytes of memory.  A more careful implementation, that
packs #w# boolean values into each word of memory could reduce this
space usage to $O(#n#^2/#w#)$ words of memory.

\begin{thm}
The #AdjacencyMatrix# data structure implements the #Graph# interface.
An #AdjacencyMatrix# supports the operations
\begin{itemize}
  \item #addEdge(i,j)#, #removeEdge(i,j)#, and #hasEdge(i,j)# in constant
  time per operation; and
  \item #inEdges(i)#, and #outEdges(i)# in $O(#n#)$ time per operation.
\end{itemize}
The space used by an #AdjacencyMatrix# is  $O(#n#^2)$.
\end{thm}

Despite the high memory usage and poor performance of the #inEdges(i)#
and #outEdges(i)# operations, an #AdjacencyMatrix# can still be useful for
some applications.  In particular, when the graph $G$ is \emph{dense},
i.e., it has close to $#n#^2$ edges, then a memory usage of $#n#^2$
may be acceptable.

The #AdjacencyMatrix# data structure is also commonly used because
algebraic operations the matrix #a# can be used to efficiently compute
properties of the graph $G$.  This is a topic for a course on algorithms,
but we point out one such property here:  If we treat the entries of #a#
as integers (1 for #true# and 0 for #false#) and multiply #a# by itself
using matrix multiplication then we get the matrix $#a#^2$.  Recall,
from the definition of matrix multiplication, that
\[
    #a^2[i][j]# = \sum_{k=0}^{#n#-1} #a[i][k]#\cdot #a[k][j]# \enspace .
\]
Interpreting this sum in terms of the graph $G$, this formula counts the
number of vertices, $#k#$, such that the $G$ contains both edges #(i,k)#
and #(k,j)#.  That is, it counts the number of paths from $#i#$ to $#j#$
(through intermediate vertices, $#k#$) that have length exactly 2.
This observation is the foundation of an algorithm that computes the
shortest paths between all pairs of vertices in $G$ using only $O(\log
#n#)$ matrix multiplications.

\section{#AdjacencyLists#: A Graph as a Collection of List}
\seclabel{adjacency-list}

An \emph{adjacency list} representation of a graph takes a more
vertex-centric approach.  There are many different possible
implementations of adjacency lists.  In this section, we present a simple
one.  At the end of the section, we discuss different possibilities.
In this section, the graph $G=(V,E)$ is represented as an array,
#adj#, of lists.  The list #adj[i]# contains a list of all the vertices
adjacent to vertex #i#.  That is, it contains every index #j# such that
$#(i,j)#\in E$.
\codeimport{ods/AdjacencyLists.adj.n.AdjacencyLists(n0)}
In this particular implementation, we represent each list in #adj#
as an #ArrayStack#, because we would like constant time access by
position. Other options are also possible.  Specifically, we could have
implemented #adj# as a #DLList#.

The #addEdge(i,j)# operation just appends the value #j# to the list #adj[i]#:
\codeimport{ods/AdjacencyLists.addEdge(i,j)}
This takes constant time.

The #removeEdge(i,j)# operation searches through the list #adj[i]#
until it finds #j# and then removes it:
\codeimport{ods/AdjacencyLists.removeEdge(i,j)}
This takes $O(\deg(#i#))$ time, where $\deg(#i#)$ (the \emph{degre} of
$#i#$) counts the number of edges in $E$ that have $#i#$ as their source.

The #hasEdge(i,j)# operation is similar;  it searches through the list
#adj[i]# until it finds #j# (and returns true), or reaches the end of
the list (and returns false):
\codeimport{ods/AdjacencyLists.hasEdge(i,j)}
This also takes $O(\deg(#i#))$ time.

The #outEdges(i)# operation is very simple.  It simply returns the
list #adj[i]#:
\codeimport{ods/AdjacencyLists.outEdges(i)}
This clearly takes constant time.

The #inEdges(i)# operation is much more work.  It scans over every
vertex $j$ checking if the edge #(i,j)# exists and, if so adding #j#
to the output list:
\codeimport{ods/AdjacencyLists.inEdges(i)}
This operation is very slow. It scans the adjacency list of every vertex,
so it takes $O(#n# + #m#)$ time.

The following theorem summarizes the performance of the above data structure:

\begin{thm}
The #AdjacencyLists# data structure implements the #Graph# interface.
A #AdjacencyLists# supports the operations
\begin{itemize}
  \item #addEdge(i,j)# in constant time per operation;
  \item #removeEdge(i,j)# and #hasEdge(i,j)# in $O(\deg(#i#))$ time
    per operation;
  \item #outEdges(i)# in constant time per operation; and
  \item #inEdges(i)# in $O(#n#+#m#)$ time per operation.
\end{itemize}
The space used by a #AdjacencyLists# is  $O(#n#+#m#)$.
\end{thm}

As alluded to earlier, there are many different choices to be made when
implementing a graph as an adjacency list.  Some questions that come
up include:
\begin{itemize}
  \item What type of collection should be used to store each element
  of #adj#?  One could use an array-based list,  a linked-list, or even
  a hashtable.
  \item Should there be a second adjacency list, #inadj#, that stores,
  for each #i#, this list of vertices, #j#, such that $#(j,i)#\in E$?
  This can greatly reduce the running-time of the #inEdges(i)#
  operation, but requires slightly more work in the #addEdge(i,j)#
  and #removeEdge(i,j)# operations.
  \item Should the entry for the edge #(i,j)# in #adj[i]# be linked by
  a reference to the corresponding entry in #inadj[j]#.
  \item Should edges be first-class objects with their own associated data?
\end{itemize}
Most of these questions come down to a tradeoff between complexity of
implementation and performance features of the implementation.

\section{Graph Traversal}

In this section we present two algorithms for exploring a graph, starting
at one of its vertices, #i#, and finding all vertices that are reachable
from #i#.  Both of these algorithms are best suited to graphs represented
using an adjacency list representation.  Therefore, when analyzing these
algorithms we will assume that the underlying representation is as an
#AdjacencyLists#.

\subsection{Breadth-First Search}

The \emph{bread-first-search} algorithm starts at a vertex #i# and visits,
first the neighbours of #i#, then the neighbours of the neighbours of #i#,
then the neighbours of the neighbours of the neighbours of #i#, and so on.

This algorithm is a generalization of the breadth-first-search algorithm
for binary trees (\secref{bintree:traversal}), and is very similar; it
uses a queue, #q#, that initially contains only #i#.  It then repeatedly
extracts an element from #q# and adds its neighbours to #q#, provided
that these neighbours have not been in #q#.  The only major difference
between the breadth-first-search algorithm for graphs and for trees
is that the algorithm for graphs has to ensure that it does not visit
add the same vertex to #q# more than once.  It does this by using an
auxiliary boolean array, #seen#, that keeps track of which vertices have
already been discovered.
\codeimport{ods/Algorithms.bfs(g,r)}

Analyzing the running-time of the #bfs(g,i)# routine is fairly
straightforward.  The use of the #seen# array ensures that no vertex is
added to #q# more than once.  Adding (and later removing) each vertex
from #q# takes constant time per vertex for a total of $O(#n#)$ time.
Since each vertex is processed at most once, each adjacency list is
processed at most once, so each edge of $G$ is processed at most once.
This processing, which is done in the inner loop takes constant time
per iteration, for a total of $O(#m#)$ time.  Therefore, the entire
algorithm runs in $O(#n#+#m#)$ time.

The following theorem summarizes the performance of the #bfs(g,r)# algorithm.
\begin{thm}\thmlabel{bfs-graph}
  When given as input a #Graph#, #g#, that is implemented using the
  #AdjacencyLists# data structure, the #bfs(g,r)# algorithm runs in $O(#n#+#m#)$
  time.
\end{thm}

A breadth-first traversal has some very special properties.  Calling
#bfs(g,r)# will eventually enqueue (and eventually dequeue) every vertex
#j# such that there is a directed path from #r# to #j#.  Moreover,
the vertices at distance 0 from #r# (#r# itself) will enter #q# before
the vertices at distance 1, which will enter #q# before the vertices at
distance 2, and so on.  Thus, the #bfs(g,r)# method outputs vertices
in increasing order of distance from #r# and vertices that can not be
reached from #r# are never output at all.

A particularly useful application of the breadth-first-search algorithm
is, therefore, in computing shortest paths.  To compute the shortest
path from #r# to every other vertex, we use a variant of #bfs(g,r)#
that uses an auxilliary array, #p#, of length #n#.  When a new vertex
#j# is added to #q#, we set #p[j]=i#.  In this way, #p[j]# becomes the
second last node on a shortest path from #r# to #j#.  Repeating this,
by taking #p[p[j]#, #p[p[p[j]]]#, and so on we can reconstruct the
(reversal of) a shortest path from #r# to #j#.



\subsection{Depth-First Search}

The \emph{depth-first-search} algorithm is similar to the standard
algorithm for traversing binary trees.  It first fully explores one
subtree before returning to the current node and then exploring the
other subtree.  Another way to think of depth-first-search is by saying
that it is similar to breadth-first search except that it uses a stack
instead of a queue.

During the execution of the depth-first-search algorithm, each vertex
is assigned a color: #white# if we have never seen the vertex before,
#grey# if we are currently visiting that vertex, and #black# if we are
done visiting that vertex.  The easiest way to think of depth-first-search
is as a recursive algorithm.  It starts by visiting #r#.  When visiting a
vertex #i#, we first mark #i# as grey.  Next, we scan #i#'s adjacency list
and recursively visit any white vertex we find in this list.  Finally,
we are done processing #i#, so we color #i# black and return.
\codeimport{ods/Algorithms.dfs(g,r).dfs(g,i,c)}

Although depth-first-search may best be thought of as a recursive
algorithm, recursion is not the best way to implement it. Indeed, the code
given above will fail for many large graphs by causing a stack overflow.
An alternative implementation is to replace the recursion stack with an
explicit stack.  The following implementation does just that:
\codeimport{ods/Algorithms.dfs2(g,r)}
In the above code, when next vertex #i# is processed, it is colored grey
and then replaced, on the stack, with its adjacent vertices.  During the
next round, one of these vertices will be visited.

Not surprisingly, the running time of #dfs(g,r)# and #dfs2(g,r)# is the
same as that of #bfs(g,r)#:
\begin{thm}\thmlabel{dfs-graph}
  When given as input a #Graph#, #g#, that is implemented using the
  #AdjacencyLists# data structure, the #dfs(g,r)# and #dfs2(g,r)# algorithms
  each run in $O(#n#+#m#)$ time.
\end{thm}

As with the breadth-first-search algorithm, there is an underlying
tree associated with each execution of depth-first-search.  When a node
$#i#\neq #r#$ goes from #white# to #grey#, this is because #dfs(g,i,c)#
was called recursively while processing some node #i'#.  (In the case
of #dfs2(g,r)# algorithm, #i# is one of the nodes that replaced #i'#
on the stack.)  If we think of #i'# as the parent of #i#, then we obtain
a tree rooted at #r#.

An important property of the depth-first-search algorithm is that, once
the algorithm starts colors the node #i# grey, #i# will not become black
until every node #j# for which there is a path from #i# to #j# that uses
only white nodes is also black.  One application of this property is the
detection of cycles.  Any time the #dfs(g,i,c)# algorithm chooses not
to recurse on a node #j# because #j# is grey, this implies that there
is a cycle in $G$ since there is a path from #j# to #i# on the recursive
stack and there is an edge from #i# to #j#.  On the other hand, if there
is a cycle $C$ in $G$, and there is a path from #r# to any vertex in $C$,
then $dfs(g,i,c)$ will eventually discover $C$ or some other cycle.

\section{Discussion and Exercises}

The running times of the depth-first-search and breadth-first-search
algorithms are somewhat overstated by the Theorems~\ref{thm:bfs-graph} and
\ref{thm:dfs-graph}.  Define $#n#_{#r#}$ as the number of vertices, #i#
of $G$, for which there exists a path from #r# to #i#.  Define $#m#_#r#$
as the number of edges that have these vertices as their sources.
Then the following theorem is a more precise statement of the running
times of the breadth-first-search and depth-first-search algorithms.
(This more refined statement of the running time is useful in some of
the applications of these algorithms outlined in the exercises.)
\begin{thm}\thmlabel{graph-traversal}
  When given as input a #Graph#, #g#, that is implemented using the
  #AdjacencyLists# data structure, the #bfs(g,r)#, #dfs(g,r)# and #dfs2(g,r)#
  algorithms each run in $O(#n#_{#r#}+#m#_{#r#})$ time.
\end{thm}

Breadth-first search seems to have been discovered independently by
Moore \cite{m59} and Lee \cite{l61} in the contexts of maze exploration
and circuit routing, respectively.

Adjacency-list representations of graphs were first popularized by
Hopcroft and Tarjan \cite{ht73} as an alternative to the (then more
common) adjacency-matrix representation.  This representation, and
depth-first-search played a major part in the celebrated Hopcroft-Tarjan
planarity testing algorithm that can determine, in $O(#n#)$ time, if
a graph can be drawn, in the plane, and in such a way that no pair of
edges cross each other \cite{ht74}.

In the following exercises, an undirected graph is one in which, for
every #i# and #j#, the edge $(#i#,#j#)$ is present if and only if the
edge $(#j#,#i#)$ is present.
\begin{exc}
  Let $G$ be an undirected graph.  We say $G$ is \emph{connected} if, for
  every pair of vertices #i# and #j# in $G$, there is a path from $#i#$
  to $#j#$. Show how to test if $G$ is connected in $O(#n#+#m#)$ time.
\end{exc}

\begin{exc}
  We say $G$ is \emph{connected} if, for
  every pair of vertices #i# and #j# in $G$, there is a path from $#i#$
  to $#j#$. Show how to test if $G$ is connected in $O(#n#+#m#)$ time.
\end{exc}



\begin{exc}
  Let $G$ be an undirected graph.  A \emph{connected-component labelling} of
  $G$ partitions to the vertices of $G$ into maximal sets, each of which
  forms a connected subgraph.  Show how to compute a connected component
  labelling of $G$ in $O(#n#+#m#)$ time.
\end{exc}

\begin{exc}
  Let $G$ be an undirected graph.  A \emph{spanning forest} of $G$ is a
  collection of trees, one per component, whose edges are edges of $G$
  and whose vertices contain all vertices of $G$.  Show how to compute
  a spanning forest of of $G$ in $O(#n#+#m#)$ time.
\end{exc}

\begin{exc}
  Given a graph $G=(V,E)$ and some special vertex $#r#\in V$, show how
  to compute the length of the shortest path from $#r#$ to #i# for every
  vertex $#i#\in V$.
\end{exc}



