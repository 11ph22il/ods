\documentclass{book}
%\usepackage{fullpage}
\usepackage{amsthm}
\usepackage{listings}
\usepackage[usenames,dvipsnames]{color}


\lstset{
  language=java,
  basicstyle=\small\ttfamily,
  keywordstyle=\color{RoyalPurple},
  identifierstyle=\color{Blue},
  stringstyle=\color{Blue},
  showstringspaces=false,
  commentstyle=\color{ForestGreen},
  frame=single
}

\newtheorem{thm}{Theorem}
\setlength{\parskip}{1ex}

\newlength{\pad}\setlength{\pad}{1pt}

\newcommand{\javaimport}[1]{\noindent\frame{\begin{minipage}{\textwidth}\vspace{\pad}\hspace{\pad}\texttt{#1}\vspace{\pad}\end{minipage}}\vspace{1ex}}
\newcommand{\javaimportwithclass}[1]{\javaimport{#1}}

\begin{document}
\chapter{Array-Based Lists and Queues}

In this chapter, we study implementations of the #List# and #Queue# interfaces
where the underlying data is stored in an array.

\section{ArrayStack}

An #ArrayStack# implements the list interface using an array #a#, called
the \emph{backing array}.  The list element with index #i# is stored
in #a[i]#.  At most times, #a# is larger than strictly necessary,
so an integer #n# is used to keep track of the number of elements
actually stored in #a#.  In this way, the list elements are stored in
#a[0]#,\ldots,#a[n-1]# and, at all times, $#a.length# > #n#$.

\subsection{The Basics}

\javaimportwithclass{ods/ArrayStack.a.n.ArrayStack().size()}

Accessing and modifying the elements of an #ArrayStack# using #get(i)#
and #set(i,x)# is trivial. After any doing any necessary bounds-checking
we simply return or set, respectively, #a[i]#.

\javaimport{ods/ArrayStack.get(i).set(i,x)}

To implement the #add(i,x)# operation, we first check if #a# is already
full.  If so, we call the method #resize()# to increase the size of #a#;
how #resize()# is implemented will be discussed later.  For now, it is
sufficient to know that, after a call to #resize()#, we can be sure that
$#a.length# > #n#$.  With this out of the way, we now shift the elements
$#a[i]#,\ldots,#a[n-1]#$ right by one position to make room for #x#,
we set #a[i]# equal to #x# and increment #n#.

\javaimport{ods/ArrayStack.add(i,x)}
% TODO: Add shifting figure
If we ignore the cost of the potential call to #resize()#, the cost of the
#add(i,x)# operation is proportional to the number of elements we have
to shift to make room for #x#.  Therefore the cost of this operation
(ignoring the cost of resizing #a#) is $O(#n#-#i#+1)$.

Implementing the #remove(i)# operation is similar.  We shift the elements
$#a[i+1]#,\ldots,#a[n-1]#$ left by one position (overwriting #a[i]#) and
decrease the value of #n#.  After doing this, we check if #n# is getting
much smaller than #a.length# by checking if $#a.length# > 3#n#$. If so,
we call #resize()# to reduce the size of #a#.

\javaimport{ods/ArrayStack.remove(i)}
% TODO: Add shifting figure
If we ignore the cost of the #shrink()# method, the cost of a #remove(i)#
operation is proportional to the number of the number elements we shift,
which is $O(#n#-#i#)$. 

\subsection{Growing and Shrinking}

The #resize()# method is fairly straightforward; it allocates a new
array #b# whose size is $2#n#$ and copies the #n# elements of #a# into
the first #n# positions in #b#, and then sets #a# to #b#. Thus, after a call to #resize()#, $#a.length# = 2#n#$.

\javaimport{ods/ArrayStack.resize()}

Analyzing the cost of #resize()# operation is easy. It allocates an
array #b# of size $2#n#$ and copies the #n# elements of #a# into #b#.
This takes $O(#n#)$ time.

The running time analysis from the previous section ignored the cost
calls to #resize()#.  In this section we analyze this cost using a
technique known as \emph{amortized analysis}.  This technique does not
try and determine the cost of resizing during each individual #add(i,x)#
and #remove# operation.  Instead, it considers the cost of all calls to
#resize()# during a sequence of $m$ calls to #add(i,x)# or #remove(x)#.
In particular, we will show:

\begin{thm}
  If an empty #ArrayList# is created and any sequence of $m\ge 1$ calls
  to #add(i,x)# and #remove(i)# are performed, then the total cost of all work
  done during all calls to #resize()# is $O(m)$.
\end{thm}

\begin{proof}
  We will show that anytime #resize()# is called, the number of calls
  to #add# or #remove# since the last call to #resize()# is at least
  $#n#/2$.  Therefore, if $n_i$ denotes the value of #n# during the $i$th call to #resize()#, then the total number of calls to #add# or #remove# is at least
  \[
     \sum_{i=1}^{r} n_i/2 \le m  \enspace .
  \]
  On the other hand, the total work done during all calls to #resize()# is 
  \[
     \sum_{i=1}^{r} O(n_i) \le O(m)  \enspace ,
  \]
  which will prove the theorem.  All that remains is to show that the
  number of calls to #add(i,x)# or #remove(i)# between the $(i-1)$th and
  the $i$th call to #resize()# is at least $n_i/2$.
  
  There are two cases to consider. In the first case, #resize()# is
  being called by #add(i,x)# because the backing array #a# is full, i.e.,
  $#a.length()# = #n#=n_i$.  Consider the previous call to #resize()#:
  After this previous call, the size of #a# was #a.length#, but the number
  of elements stored in #a# was $#a.length#/2=n_i/2$.  But now the number
  of elements stored in #a# is $n_i=#a.length#$, so there must have been at
  least $n_i/2$ calls to #add(i,x)# since the previous call to #resize()#.
  % TODO: Add figure
  
  The second case to consider is then #resize()# is being called by
  #remove(i)# because $#a.length# \ge 3#n#=3n_i$.  Again, after the
  previous call to #resize()# the number of elements stored in #a# was
  exactly #a.length/2#.  Now there are $n_i\le#a.length#/3$ elements
  stored in #a#.  Therefore, the number of #remove(i)# operations since
  the last call to #resize()# is at least
  \[
      #a.length#/2 - #a.length#/3 = #a.length#/6 
         = (#a.length#/3)/2 \ge n_i/2 \enspace .
  \]
  In either case, the number of calls to #add(i,x)# or #remove(i)# that
  occur between the $(i-1)$th call to #resize()# and the $i$th call to
  resize is at least $n_i/2$, as required to complete the proof.
\end{proof}

\subsection{Summary}

We have presented the #ArrayStack#, a data structure that implements the #List# interface.  If we ignore the cost of calls to the #resize()# method, an #ArrayStack# has the following performance bounds:

\begin{tabular}{cccc}
#get(i)# & #set(i,x)# & #add(i,x)# & #remove(i)# \\
$O(1)$ & $O(1)$ & $O(1+#n#-#i#)$ & $O(#n#-#i#)$
\end{tabular}

Furthermore, if we begin with an empty #ArrayStack# and perform any
sequence of $m$ #add(i,x)# and #remove(i)# operations, then the total
cost of all work done by #resize()# during this sequence of operations
is $O(m)$.  Thus, the amortized cost of resizing, when amortized over
all #add(i,x)# and #remove(i)# operations is $O(1)$ per operation.

The #ArrayStack# is efficient way to implement a #Stack#: a sequence
that allows us to insert (#push(x)#) or remove (#pop()#) elements at
one end. In particular, if we can implement #push(x)# as #add(n,x)#
and #pop()# as #remove(n-1)#, in which case these operations will run
in $O(1)$ amortized time.

\section{FastArrayStack}

Much of the work done by an #ArrayStack# involves shifting (by
#add(i,x)# and #remove(i)#) and copying (by #resize()#) of data.  In the
implementations shown above, this was done using #for# loops. It turns
out that many programming environments have specific functions that are
very efficient at copying and moving blocks of data.  In the C and C++
programming languages there is the #memcpy(d,s,n)# function.  In Java
there is the #System.arraycopy(s,i,d,j,n)# method.

\javaimport{ods/FastArrayStack.add(i,x).remove(i).resize()}

These functions are usually highly optimized and may even use special
machine instructions that can do this copying much faster than we
could do with #for# loop. In the Java implementations here, the use of
#System.arraycopy(s,i,d,j,n)# resulted in speedups of a factor of 2-3
depending on the types of operations performed.



\section{ArrayQueue}

In this section, we present the #ArrayQueue# data structure, which
implements a FIFO (first-in-first-out) queue; elements are removed (using
the #remove()# operation) from the queue in the same order they are added
(using the #add(x)# operation).

Notice that an #ArrayStack# is a poor choice for an implementation of a
FIFO queue.  The reason is that we must choose one end of the list to
add to and then remove from the other end.  One of the two operations
must work on the head of the list, which involves calling #add(i,x)#
or #remove(i)# with a value of $#i#=0$.  This gives a running time
of $\Theta(n)$.

To obtain an efficient array-based implementation of a queue, we
first notice that the problem would be easy if we had an infinite
array #a#.  We could maintain one index #j# that keeps track of the
next element to remove and an integer #n# that counts the number of
elements in the queue.  In this way, the queue elements are always stored in
\[ #a[j]#,#a[j+1]#,\ldots,#a[j+n-1]# \enspace . \]
  Initially, both #i# and #n# are
set to 0.  To add an element, we place it in #a[j+n]# and increment #n#.
To remove an element, we remove it from #a[j]#, increment #j#, and
decrement #n#.

Of course, the problem with this solution is that it requires an infinite
array #a#.  An #ArrayStack# simulates this by using a finite array #a#
and \emph{modular arithmetic}.  This is the kind of arithmetic used when
we are talking about time of day.  For example 10 o'clock plus 5
hours gives 3 o'clock.  Formally, we say that
\[
    10 + 5 = 15 \equiv 3 \pmod{12} \enspace .
\]
We read the latter part of this equation as ``15 is congruent to 3 modulo
12.'' We can also treat $\bmod$ as binary operator, so that
\[
   15 \bmod 12 = 3 \enspace .
\]

More generally, for an integer $a$ and positive integer $m$, $a \bmod m$
is the unique integer $r\in\{0,\ldots,m-1\}$ such that $a = r + km$ for
some integer $k$.  Less formally, the value $r$ is the remainder we get
when we divide $a$ by $m$.  In many programming languages, including Java,
the $\bmod$ operator is represented using the #%# symbol.\footnote{This
is sometimes referred to as the \emph{brain-dead} mod operator since
it does not correctly implement the mathematical mod operator when the
first argument is negative.}

Modular arithmetic is useful for simulating an infinite
array, since $#i#\bmod #a.length#$ always gives a value
in the range $0,\ldots,#a.length-1#$.  In this way,
we can store our queue elements at array locations \[
#a[j%a.length]#,#a[(j+1)%a.length]#,\ldots,#a[(j+n-1)%a.length]# \]
% TODO: figure

The only remaining thing to worry about is taking care that the number
of elements in the #ArrayQueue# does not exceed the size of #a#.

\javaimportwithclass{ods/ArrayQueue.a.j.n}

To implement #add(x)#, we first check if #a# is full and, if necessary,
call #resize()# to increase the size of #a#.  Next, we store #x# in
#a[(j+n)%a.length]# and increment #n#.

\javaimport{ods/ArrayQueue.offer(x)}

To implement #remove()# we first store #a[j]# so that we can return
it later.  Next, we decrement #n# and ``increment'' #j# by setting
$#j#=(#j#+1)\bmod #a.length#$ and return the stored value of #a[j]#. If
necessary, we may call #resize()# to decrease the size of #a#.

\javaimport{ods/ArrayQueue.poll()}

Finally, the #resize()# operation is very similar to the #resize()#
operation of #ArrayQueue#.  It allocates a new array #b# of size $2#n#$
and copies
\[
   #a[j]#,#a[(j+1)%a.length]#,\ldots,#a[(j+n-1)%a.length]#
\]
onto
\[
   #b[0]#,#b[1]#,\ldots,#b[n-1]#
\]
and sets $#j#=0$.

\end{document}

