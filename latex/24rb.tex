\chapter{2-4 Trees and Red-Black Trees}
\chaplabel{24redblack}
\section{2-4 Trees}
\section{Red-Black Trees}

Property 1: (black-height) The same number of black nodes on every root to leaf path.

Property 2: (max-degree) No two red nodes are adjacent

Property 3: (left-leaning) If #u.left# is black, then #u.right# is black.

\subsection{Insertion}

Do a normal BST insertion and make a new red leaf.  The only thing we
have to worry about is the max-degree property.  In the 2-4 view, represents
either a node of degree 5 or a node of degree 4.

For a node of degree 5, we apply the split transformation (color the
node red and color its two red children black) and proceed up the tree.

A node of degree 4 can be fixed with a single or double rotation depending
on whether its a zig-zig or a zig-zag.  This is where the cases can be
simplified if we use a left-leaning tree (since the zig-zag is converted
to a zig-zig by the left-leaning transformation).


\begin{figure}
  \begin{center}
    \includegraphics{figs/rb-addfix}
  \end{center}
  \caption{A single round in the process of fixing Property~2 after
  an insertion.}
  \figlabel{rb-addfix}
\end{figure}

\subsection{Deletion}

Do a standard deletion.  If the deleted node is red then we are done.

\begin{figure}
  \begin{center}
    \includegraphics{figs/rb-removefix}
  \end{center}
  \caption{A single round in the process of eliminating a double-black node
   after a removal.}
  \figlabel{rb-removefix}
\end{figure}


