\chapter{Sorting Algorithms}
\chaplabel{sorting}

\section{Comparison-Based Sorting}

In this section, we present three sorting algorithms: merge-sort,
Quicksort, and heap-sort.  All these algorithms take an input array #a#
and sort the elements of #a# into non-decreasing order in $O(#n#\log #n#)$
(expected) time.  These algorithms are all \emph{comparison-based}.
They don't care what type of data is being sorted, the only 

\subsection{Merge-Sort}

The #mergeSort(a)# algorithm is a classic example of recursive divide and
conquer: If the length of #a# is at most 1, then #a# is already
sorted, so we do nothing.  Otherwise, we split #a# into two halves
$#a0#=#a[0]#,\ldots,#a[n/2-1]#$ and $#a1#=#a[n/2]#,\ldots,#a[n-1]#$.
We recursively sort #a0# and #a1#, and then we merge (the now sorted)
#a0# and #a1# to get our fully sorted array #a#:
\javaimport{ods/Algorithms.mergeSort(a,c)}
Compared to sorting, merging the two sorted arrays #a0# and #a1# fairly
easy.  We add elements to #a# one at a time.  If #a0# or #a1# is empty
we add the next element from the other (non-empty) array. Otherwise,
we take the minimum of the next element in #a0# and the next element in
#a1# and add it to #a#:
\javaimport{ods/Algorithms.merge(a0,a1,a,c)}
Notice that the #merge(a0,a1,a,c)# algorithm performs at most $#n#-1$
comparisons before running out of elements in one of #a0# or #a1#.

The following theorem shows that #mergeSort(a,c)# is a very efficient algorithm:
\begin{thm}
  The #mergeSort(a,c)# algorithm runs in $O(#n#\log #n#)$ time and
  performs at most $#n#\log #n#$ comparisons.
\end{thm}

\begin{proof}
The proof is by induction on $#n#$.  The base case, in which $#n#=1$,
is trivial.

Merging two sorted lists of total length $#n#$ requires at most $#n#-1$
comparisons. Let $C(#n#)$ denote the maximum number of comparisons performed by
#mergeSort(a,c)# on an array #a# of length #n#.  If $#n#$ is even, then we apply the inductive hypothesis to
the two subproblems and obtain
\begin{align*}
  C(#n#) 
  &\le #n#-1 + 2C(#n#/2) \\
  &= #n#-1 + 2((#n#/2)\log(#n#/2) \\
  &= #n#-1 + #n#\log(#n#/2) \\
  &= #n#-1 + #n#\log #n#-#n# \\
  &< #n#\log #n# \enspace .
\end{align*}
The case where $#n#$ is odd is slightly more complicated.  For this case,
we use the inequality (justified below) that shows $\log (x+1/2) <
\log x + (\log e)/2x$.  So we have
\begin{align*}
  C(#n#) 
  &= #n#-1 + C(\lceil #n#/2 \rceil) + C(\lfloor #n#/2 \rfloor) \\
  &= #n#-1 + \lceil #n#/2 \rceil\log \lceil #n#/2 \rceil 
           + \lfloor #n#/2 \rfloor\log \lfloor #n#/2 \rfloor \\
  &= #n#-1 + (#n#/2+1/2)\log (#n#/2+1/2) 
           + \lfloor #n#/2 \rfloor\log \lfloor #n#/2 \rfloor \\
  &< #n#-1 + \lceil #n#/2 \rceil(\log #n#/2 + (\log e)/#n#)
           + \lfloor #n#/2 \rfloor\log (#n#/2) \\
  &= #n#\log #n# - 1 + (#n#/2+1/2)(\log e/#n#) \\
  &< #n#\log #n# 
\end{align*}
since $(#n#/2+1/2)(\log e/#n#) < 1$ for all $#n#\ge 2$.

All that remains is to justify the inequality
$\log (x+1/2) <
\log x + (\log e)/2x$.  This inequality comes from the identity $\ln x = \int_1^x (1/x),\mathrm{d}x$.  This identity yields:
\[
   \ln(x+1/2) = \int_1^{x+1/2} (1/x)\,\mathrm{d}x
    = \int_1^{x} (1/x)\,\mathrm{d}x + \int_{x}^{x+1/2} (1/x)\,\mathrm{d}x
    < \int_1^{x} (1/x)\,\mathrm{d}x + 1/2x \enspace .\footnote{The same derivation yields the famous inequality, seen in many calculus books, $1+x < e^x$.}
\]
Multiplying both sides of this inequality by $\log e$ yields
$\log (x+1/2) < \log x + (\log e)/2x$.
\end{proof}

\subsection{Quicksort}

The \emph{quicksort} algorithm is another classic divide and conquer
algorithm.  Unlike mergesort, which does merging after solving the two
subproblems, quicksort does all its work upfront.

The algorithm is simple to describe:  Pick a random element #x# from #a#;
partition #a# into the set of elements less than #x#, the set of
elements equal to #x#, and the set of elements greater than #x#; and,
finally, recursively sort the first and third sets in this partition.
\javaimport{ods/Algorithms.quickSort(a,c).quickSort(a,i,n,c)}
All of this is done in-place, so that instead of making copies of
subarrays being sorted, the #quickSort(a,i,n,c)# method only sorts the
subarray $#a[i]#,\ldots,#a[i+n-1]#$.  Initially, this method is called
as #quickSort(a,0,a.length,c)#.

At the heart of the quicksort algorithm is the in-place partitioning done that, without any extra space swaps elements in #a# and computes indices #p# and #q# so that
\[
   #a[i]# \begin{cases} 
         {}< #x# & \text{if $0\le #i#\le #p#$} \\
         {}= #x# & \text{if $#p#< #i# < #q#$} \\
         {}> #x# & \text{if $#q#\le #i# \le #n#-1$}
     \end{cases}
\]
This partitioning, which is done by the #while# loop in the code, works
by iteratively increasing #p# and decreasing #q# while maintaining the
first and last of these conditions.  At each step, element at position
#j# is either moved to the front, left where it is, added to the back.
In the first two cases, #j# is incremented, while in the last case, #j#
is not incremented since the new element at position #j# has not been
processed yet.

The quicksort algorithms is very closely related to the random binary
search trees from \secref{rbst}.  In fact, if the input to quicksort
consists of #n# distinct elements, then the quicksort recursion tree is
a random binary search tree.

To see this, recall that when constructing a random binary search
tree the first thing we do is pick a random element #x# and make it
the root of the tree.  After this, every element will eventually be
compared to #x#, with smaller elements going into the left subtree and
larger elements going into the right subtree.  In quicksort, we select
a random element #x# and immediately compare everything to #x#, putting
the smaller elements at the beginning of the array and larger elements
at the end of the array.  Quicksort then recursively sorts the beginning
of the array and the end of the array, while the random binary search
tree recursively inserts smaller elements in the left subtree of the
root and larger elements in the right subtree of the root.

The above correspondence between random binary search trees and quicksort
means that we can translate \lemref{rbs} to a statement about quicksort:

\begin{lem}\lemlabel{quicksort}
  When quicksort is called to sort an array containing the integers
  $0,\ldots,#n#-1$, the expected number of times element #i# is compared to
  a pivot element is at most $H_{#i#+1} + H_{#n#-#i#}$.
\end{lem}

A little summing of harmonic numbers gives us the following theorem
about the running-time of quicksort:

\begin{thm}\thmlabel{quicksort-i}
  When quicksort is called to sort an array containing #n# distinct elements,
  the expected number of comparisons performed is $2#n#\ln #n# + O(#n#)$.
\end{thm}

\begin{proof}
Let #T# be the number of comparisons performed by quicksort when sorting #n# distsinct elements.  Using \lemref{quicksort}, we have:
\begin{align*}
  \E[T] &= \sum_{i=0}^{#n#-1}(H_{#i#+1}+H_{#n#-#i#}) \\ 
        &= 2\sum_{i=1}^{#n#}H_i \\ 
        &\le 2\sum_{i=1}^{#n#}H_{#n#} \\ 
        &\le 2#n#\ln#n# + 2#n# = 2#n#\ln #n# + O(#n#) \qedhere
\end{align*}
\end{proof}

\thmref{quicksort} describes the case where the elements being sorted are
all distinct.  When the input array, #a#, contains duplicate elements,
the expected running time of quicksort is no worse, and can be even
better; any time a duplicate element #x# is chosen as a pivot, all
occurrences of #x# get grouped together and don't take part in either
of the two subproblems.

\begin{thm}\thmlabel{quicksort}
  The #quickSort(a,c)# algorithm runs in $O(#n#\log #n#)$ expected time
  and does the expected number of comparisons it performs is at most
  $2#n#\ln #n# +O(#n#)$ comparisons.
\end{thm}

\subsection{Heap-sort}

The heap-sort algorithm is another in-place sorting algorithm.
Heap-sort uses the binary heaps discussed in \secref{binaryheap}.
Recall that the #BinaryHeap# data structure represents a heap using
a single array.  The heap-sort algorithm converts the input array #a#
into a heap and then repeatedly extracts the minimum value.

More specifically, a heap stores #n# elements at array locations
$#a[0]#,\ldots,#a[n-1]#$ with the smallest value stored at the root,
#a[0]#.  After transforming #a# into a #BinaryHeap#, the heap-sort
algorithm repeatedly swaps #a[0]# and #a[n-1]#, decrements #n#, and
calls #trickleDown(0)# so that $#a[0]#,\ldots,#a[n-2]#$ once again are
a valid heap representation. When this process ends (because $#n#=0$)
the elements of #a# are stored in decreasing order, so #a# is reversed
to obtain the final sorted order.\footnote{The algorithm
could alternatively redefine the #compare(x,y)# function so that the
heap sort algorithm stores the elements directly in ascending order.}

\javaimport{ods/BinaryHeap.sort(a,c)}

A key subroutine in heap sort is the constructor for turning an unsorted
array #a# into a heap.  It would be easy to do this in $O(#n#\log#n#)$
time by repeatedly calling the #BinaryHeap# #add(x)# method, but
we can do better by using a bottom-up algorithm.  Recall that,
in a binary heap, the children of #a[i]# are stored at positions
#a[2i+1]# and #a[2i+2]#.  This implies that the elements 
$#a#[\lfloor #n#/2\rfloor],\ldots,#a[n-1]#$ have no children. In other
words, each of
$#a#[\lfloor #n#/2\rfloor],\ldots,#a[n-1]#$ is a sub-heap of size 1.  Now,
working backwards, we can call #trickleDown(i)# for each 
$#i#\in\{\lfloor #n#/2\rfloor,\ldots,0\}$. This works, because by the time
we call
#trickleDown(i)#, each of the two children of #a[i]# are the root of a
sub-heap so calling #trickleDown(i)# makes #a[i]# into the root of its
own subheap.
\javaimport{ods/BinaryHeap.BinaryHeap(a,c)}

The interesting thing about this bottom-up strategy is that it is more
efficient than calling #add(x)# #n# times.  To see this, notice that,
for $#n#/2$ elements, we do no work at all, for $#n#/4$ elements, we call
#trickleDown(i)# on a subheap rooted at #a[i]# and whose height is 1, for
$#n#/8$ elements, we call #trickleDown(i)# on a subheap whose height is 2,
and so on.  Since the work done by #trickleDown(i)# is proportional to
the height of the sub-heap rooted at #a[i]#, this means that the total
work done is at most
\[
    \sum_{i=1}^{\log#n#} (i-1)#n#/2^{i}
    \le \sum_{i=1}^{\infty} i#n#/2^{i}
    = #n#\sum_{i=1}^{\infty} i/2^{i}
    =  2#n# \enspace .
\]
This last equality follows by recognizing that the sum
$\sum_{i=1}^{\infty} i/2^{i}$ is equal, by definition, to the expected
number times we toss a coin up to and including the first time the coin
comes up as heads and applying \lemref{coin-tosses}.

The following theorem describes the performance of #heapSort(a,c)#.
\begin{thm}
  The #heapSort(a,c)# algorithm runs in $O(#n#\log #n#)$ time and performs at
  most $2#n#\log #n# + O(#n#)$ comparisons.
\end{thm}

\begin{proof}
The algorithm runs in 3 steps:  (1)~Transforming #a# into a heap,
(2)~repeatedly extracting the minimum element from #a#, and (3)~reversing
the elements in #a#.  We have just argued that step~1 takes $O(#n#)$
time and performs $O(#n#)$ comparisons.  Step~3 takes $O(#n#)$ time and
performs no comparisons.  Step~2 performs #n# calls to #trickDown(0)#.
The $i$th such call operates on a heap of size $#n#-i$ and performs
at most $2\log(#n#-i)$ comparisons.  Summing this over $i$ gives
\[
   \sum_{i=0}^{#n#-i} 2\log(#n#-i) 
   \le \sum_{i=0}^{#n#-i} 2\log #n#
   =  2#n#\log #n#
\]
Adding the number of comparisons performed in each of the three steps
completes the proof.
\end{proof}

\subsection{A Lower-Bound for Comparison-Based Sorting}

We have now seen three comparison-based sorting algorithms that run
in $O(#n#\log #n#)$ time.  By now, we should be wondering if faster
algorithms can exist.  The short answer to this question is no.  If the
only operations allowed on the elements of #a# are comparisons then no
algorithm can avoid doing roughly $#n#\log #n#$ comparisons.  This is
not difficult to prove, but requires a little imagination.  Ultimately,
it follows from the fact that
\begin{equation}
   \log #n#! = \log #n# + \log (#n#-1) + \dots + \log(1) = n\log n - O(n)
    \eqlabel{log-factorial}
\end{equation}
(Proving \eqref{log-factorial} is left as \excref{log-factorial}.)

\subsection{Summary}

This section discusses 3 comparison-based sorting algorithms that each
run in $O(#n#\log#n#)$ time.  All these algorithms are asymptotically
optimal since any comparison-based sorting algorithm must perform at
least $#n#\log #n#-O(#n#)$ comparisons in the worst-case.  The following
table summarizes the performance of these algorithms:

\begin{tabular}{|l|r@{ }l|l|} \hline
  & \multicolumn{2}{c}{comparisons} & in-place  \\ \hline
Merge-sort & $#n#\log #n#$ &  worst-case & No  \\
Quicksort & $1.38#n#\log #n#$ & expected & Yes \\
Heap-sort & $2#n#\log #n#$ & worst-case & Yes \\ \hline
\end{tabular}

Each algorithm has advantages and disadvantages.  Merge-sort does the
fewest comparisons and does not rely on randomization.  Unfortunately,
it uses an auxilliary array during its merge phase.  Allocating this
array can be expensive and is a potential point of failure if memory
is limited.  Quicksort is an in-place algorithm and is a close second in
terms of the number of comparisons, but is randomized so this running
time is not always guaranteed.  Heap-sort does the most comparisons,
but it is in-place and deterministic.

There is one setting in which merge-sort is a clear-winner;  this occurs
when sorting a linked-list.  In this case, the auxiliary array is not
needed;  two sorted linked lists are very easily merged into a single
sorted linked-list by pointer manipulations.

\begin{exc}
  Implement a version of the merge-sort algorithm that sorts a #DLList#
  without using an auxiliary array. 
\end{exc}

\begin{exc}\exclabel{log-factorial}
  Prove that $\log #n#! = #n#\log #n#-O(#n#)$.
\end{exc}
