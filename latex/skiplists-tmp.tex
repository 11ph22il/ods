\chapter{Skiplists}
\chaplabel{skiplists}

In this chapter we discuss a beautiful data structure, the skiplist,
that has a variety of applications.  Using a skiplist we can implement
a \mbox{\texttt{List}} that is fast for all the operations \mbox{\texttt{get({\color{var}i})}}, \mbox{\texttt{set({\color{var}i},{\color{var}x})}},
\mbox{\texttt{add({\color{var}i},{\color{var}x})}}, and \mbox{\texttt{remove({\color{var}i})}}. We can also implement an \mbox{\texttt{SSet}} in which
all operations run in $O(\log n)$ time.  Finally, a skiplist can be used
to implement a \mbox{\texttt{Rope}} in which all operations run in $O(\log n)$ time.

Skiplists rely on \emph{randomization} for their efficiency.  In
particular, a skiplist uses coin tosses when an element is inserted\ldots

\section{The Basic Structure}

Conceptually, a skiplist is a sequence of singly-linked lists
$L_0,\ldots,L_h$, where each $L_r$ contains a subset of the items
in $L_{r-1}$.  We start with the input list $L_0$ that contains \mbox{\texttt{{\color{var}n}}}
items and construct $L_1$ from $L_0$, $L_2$ from $L_1$, and so on.
The items in $L_r$ are obtained by tossing a coin for each element \mbox{\texttt{{\color{var}x}}}
in $L_{r-1}$ and including \mbox{\texttt{{\color{var}x}}} in $L_r$ if the coin comes up heads.
This process ends when we create a list $L_r$ that is empty.  An example
of a skiplist is shown in \figref{skiplist}.

\begin{figure}
  \begin{center}
    \includegraphics{figs/skiplist}
  \end{center}
  \caption{A skiplist containing 6 elements.}
  \figlabel{skiplist}
\end{figure}

For an element \mbox{\texttt{{\color{var}x}}} in a skiplist, we call the \emph{height} of \mbox{\texttt{{\color{var}x}}} the
largest value $r$ such that \mbox{\texttt{{\color{var}x}}} appears in $L_r$.  Thus, for example,
elements that only appear in $L_0$ have height $0$.  Notice that the
height of \mbox{\texttt{{\color{var}x}}} corresponds to the following experiment:  Toss a coin
repeatedly until the first time it comes up tails.  How many times did
it come up heads?  The answer, not surprisingly, is that the expected
height of a node is 1.  The \emph{height} of a skiplist is the maximum
height of any of its nodes.

At the head of every list is a special node, called the \emph{sentinel}
that acts as dummy node for each list. The key property of skiplists
is that there is a short path, called the \emph{search path}, from the
sentinel in $L_h$ to every node in $L_0$.  Remembering how to construct
a search path for a node \mbox{\texttt{{\color{var}u}}} is easy (see \figref{skiplist-searchpath})
:  Start at the top left corner of your skiplist (the sentinel in $L_h$)
and always go right unless that would overshoot \mbox{\texttt{{\color{var}u}}}, in which case you
should take a step down into the list below.

More precisely, to construct the search path for the node \mbox{\texttt{{\color{var}u}}} in $L_0$
we start at the sentinel \mbox{\texttt{{\color{var}w}}} in $L_h$.  Next we examine \mbox{\texttt{{\color{var}w}.{\color{var}next}}}.
If \mbox{\texttt{{\color{var}w}.{\color{var}next}}} contains an item that appears before \mbox{\texttt{{\color{var}u}}} in $L_0$, then
we set $\mbox{\texttt{{\color{var}w}}}=\mbox{\texttt{{\color{var}w}.{\color{var}next}}}$.  Otherwise, we move down and continue the search
at the occurence of \mbox{\texttt{{\color{var}w}}} in the list $L_{h-1}$.  We continue this way
until we reach the predecessor of \mbox{\texttt{{\color{var}u}}} in $L_0$. 
\begin{figure}
  \begin{center}
    \includegraphics{figs/skiplist-searchpath}
  \end{center}
  \caption{The search path for node $4$ in a skiplist.}
  \figlabel{skiplist-searchpath}
\end{figure}

The following result, that we will prove in \secref{skiplist-analysis},  shows that the search path is quite short:

\begin{lem}\lemlabel{skiplist-searchpath}
The expected length of the search path for any node \mbox{\texttt{{\color{var}u}}} in $L_0$ is at
most $2\log_2 \mbox{\texttt{{\color{var}n}}} + O(1) = O(\log \mbox{\texttt{{\color{var}n}}})$.
\end{lem}

A space-efficient way to implement a \mbox{\texttt{Skiplist}} is to define a \mbox{\texttt{Node}}
\mbox{\texttt{{\color{var}u}}} as consisting of a data value \mbox{\texttt{{\color{var}x}}} and an array \mbox{\texttt{{\color{var}next}}} of pointers,
where \mbox{\texttt{{\color{var}u}.{\color{var}next}[{\color{var}i}]}} points to \mbox{\texttt{{\color{var}u}}}'s successor in the list $L_{\mbox{\texttt{{\color{var}i}}}}$.
In this way, the data, \mbox{\texttt{{\color{var}x}}}, in a node is stored only once even though \mbox{\texttt{{\color{var}x}}}
many appear in several lists.

\begin{Verbatim}[tabsize=2,frame=single,commandchars=\\@\$,label=\texttt{SkiplistSet},labelposition=topline]
	@\color@keyword$class$ Node {
		@\color@keyword$T$ @\color@var$x$;
		Node[] @\color@var$next$;
		Node(@\color@keyword$T$ @\color@var$ix$, @\color@keyword$int$ @\color@var$h$) {
			@\color@var$x$ = @\color@var$ix$;
			@\color@var$next$ = (Node[])Array.newInstance(Node.@\color@keyword$class$, @\color@var$h$+1);
		}
	}
\end{Verbatim}

The next three sections of this chapter discuss three different
applications of skiplists.  In each of these applications, $L_0$ stores the
main structure (a list of elements, a sorted set of elements, or a sequence
of strings).  The primary difference between these three structures is how
the search path is navigated; In particular the three structures differ in
how they decide if the search path should go down in $L_{r-1}$ or go right
in $L_r$.

\section{SkiplistSets as SortedSets}

A \mbox{\texttt{SkiplistSet}} uses a skiplist structure to implement the \mbox{\texttt{SortedSet}}
interface.   When used this way, the list $L_0$ stores the elements of
the \mbox{\texttt{SortedSet}} in sorted order.  The \mbox{\texttt{find({\color{var}x})}} method works by following
the search path for the smallest value \mbox{\texttt{{\color{var}y}}} such that $\mbox{\texttt{{\color{var}y}}}\ge\mbox{\texttt{{\color{var}x}}}$:

\begin{Verbatim}[tabsize=2,frame=single,commandchars=\\@\$,label=\texttt{SkiplistSet},labelposition=topline]
	Node findPredNode(@\color@keyword$T$ @\color@var$x$) {
		Node @\color@var$u$ = @\color@var$sentinel$;
		@\color@keyword$int$ @\color@var$r$ = @\color@var$height$ - 1;
		@\color@keyword$while$ (@\color@var$r$ >= 0) {
			@\color@keyword$while$ (@\color@var$u$.@\color@var$next$[@\color@var$r$] != @\color@var$null$ && compare(@\color@var$u$.@\color@var$next$[@\color@var$r$].@\color@var$x$,@\color@var$x$) < 0)
				@\color@var$u$ = @\color@var$u$.@\color@var$next$[@\color@var$r$];   @\color@comment$// go right in list r$
			@\color@var$r$--;                 @\color@comment$// go down into list r-1$
		}
		@\color@var$return$ @\color@var$u$;
	}
	@\color@keyword$T$ find(@\color@keyword$T$ @\color@var$x$) {
		Node @\color@var$u$ = findPredNode(@\color@var$x$);
		@\color@var$return$ @\color@var$u$.@\color@var$next$[0] == @\color@var$null$ ? @\color@var$null$ : @\color@var$u$.@\color@var$next$[0].@\color@var$x$;
	}
\end{Verbatim}

Following the search path for \mbox{\texttt{{\color{var}y}}} is easy:  When situated at some node \mbox{\texttt{{\color{var}u}}}
in  $L_{\mbox{\texttt{{\color{var}r}}}}$, we look right to \mbox{\texttt{{\color{var}u}.{\color{var}next}[{\color{var}r}].{\color{var}x}}}.  If \mbox{\texttt{{\color{var}x}}} is smaller, we take
a step to the right in $L_{\mbox{\texttt{{\color{var}r}}}}$, otherwise we move down into $L_{\mbox{\texttt{{\color{var}r}}}-1}$.
Each step (right or down) in this search takes only constant time so,
by \lemref{skiplist-searchpath}, the expected running time of \mbox{\texttt{find({\color{var}x})}}
is $O(\mbox{\texttt{{\color{var}n}}})$.

Adding an element \mbox{\texttt{{\color{var}x}}} to a skiplist using the \mbox{\texttt{add({\color{var}x})}} operation is
just a slight variation on searching.  Before we begin, we toss coins
to determine the height, \mbox{\texttt{{\color{var}k}}}, of \mbox{\texttt{{\color{var}x}}}, so that \mbox{\texttt{{\color{var}x}}} will appear in all
lists $L_0,\ldots,L_{\mbox{\texttt{{\color{var}k}}}}$.  If necessary, we increase the height of
skiplist, and then follow the search path for \mbox{\texttt{{\color{var}x}}}.  Any time the search
path goes down from a list $L_{\mbox{\texttt{{\color{var}r}}}}$ into $L_{\mbox{\texttt{{\color{var}r}}}-1}$, with $\mbox{\texttt{{\color{var}r}}}\le \mbox{\texttt{{\color{var}k}}}$,
we splice \mbox{\texttt{{\color{var}x}}} into the list $L_{\mbox{\texttt{{\color{var}r}}}}$ the same way we splice a node into
an \mbox{\texttt{SLList}}:

\begin{Verbatim}[tabsize=2,frame=single,commandchars=\\@\$,label=\texttt{SkiplistSet},labelposition=topline]
	@\color@var$boolean$ add(Node @\color@var$w$) {
		@\color@keyword$int$ @\color@var$k$ = @\color@var$w$.@\color@var$next$.@\color@var$length$ - 1;
		Node @\color@var$u$ = @\color@var$sentinel$;
		@\color@keyword$int$ @\color@var$r$ = @\color@var$height$ - 1;
		@\color@keyword$int$ @\color@var$comp$ = 0;
		@\color@keyword$while$ (@\color@var$r$ >= 0) {
			@\color@keyword$while$ (@\color@var$u$.@\color@var$next$[@\color@var$r$] != @\color@var$null$ && (@\color@var$comp$ = compare(@\color@var$u$.@\color@var$next$[@\color@var$r$].@\color@var$x$,@\color@var$w$.@\color@var$x$)) < 0)
				@\color@var$u$ = @\color@var$u$.@\color@var$next$[@\color@var$r$];
			@\color@keyword$if$ (@\color@var$u$.@\color@var$next$[@\color@var$r$] != @\color@var$null$ && @\color@var$comp$ == 0) @\color@var$return$ @\color@var$false$; @\color@comment$// already present$
			@\color@keyword$if$ (@\color@var$r$ <= @\color@var$k$) {
				@\color@var$w$.@\color@var$next$[@\color@var$r$] = @\color@var$u$.@\color@var$next$[@\color@var$r$];
				@\color@var$u$.@\color@var$next$[@\color@var$r$] = @\color@var$w$;
			}
			@\color@var$r$--;
		}
		@\color@var$n$++;
		@\color@var$return$ @\color@var$true$;
	}
\end{Verbatim}
\begin{Verbatim}[tabsize=2,frame=single,commandchars=\\@\$,label=\texttt{SkiplistSet},labelposition=topline]
	@\color@var$boolean$ add(@\color@keyword$T$ @\color@var$x$) {
		Node @\color@var$w$ = @\color@keyword$new$ Node(@\color@var$x$, pickHeight());
		@\color@keyword$if$ (@\color@var$w$.@\color@var$next$.@\color@var$length$ > @\color@var$height$)
			@\color@var$height$ = @\color@var$w$.@\color@var$next$.@\color@var$length$;
		@\color@var$return$ add(@\color@var$w$);
	}
\end{Verbatim}

\begin{figure}
  \begin{center}
    \includegraphics{figs/skiplist-add}
  \end{center}
  \caption{Adding the node 3.5 to a skiplist.}
  \figlabel{skiplist-add}
\end{figure}

Removing an element \mbox{\texttt{{\color{var}x}}} is done in a similar way.  We search for \mbox{\texttt{{\color{var}x}}} and each time the search moves downward from a node \mbox{\texttt{{\color{var}u}}}, we check if $\mbox{\texttt{{\color{var}u}.{\color{var}next}.{\color{var}x}}}=\mbox{\texttt{{\color{var}x}}}$ and if so, we splice \mbox{\texttt{{\color{var}u}}} out of the list:

\begin{Verbatim}[tabsize=2,frame=single,commandchars=\\@\$,label=\texttt{SkiplistSet},labelposition=topline]
	@\color@var$boolean$ remove(@\color@keyword$T$ @\color@var$x$) {
		@\color@var$boolean$ @\color@var$removed$ = @\color@var$false$;
		Node @\color@var$u$ = @\color@var$sentinel$;
		@\color@keyword$int$ @\color@var$r$ = @\color@var$height$ - 1;
		@\color@keyword$int$ @\color@var$comp$ = 0;
		@\color@keyword$while$ (@\color@var$r$ >= 0) {
			@\color@keyword$while$ (@\color@var$u$.@\color@var$next$[@\color@var$r$] != @\color@var$null$ && (@\color@var$comp$ = compare(@\color@var$u$.@\color@var$next$[@\color@var$r$].@\color@var$x$, @\color@var$x$)) < 0) {
				@\color@var$u$ = @\color@var$u$.@\color@var$next$[@\color@var$r$];
			}
			@\color@keyword$if$ (@\color@var$u$.@\color@var$next$[@\color@var$r$] != @\color@var$null$ && @\color@var$comp$ == 0) {
				@\color@var$removed$ = @\color@var$true$;
				@\color@var$u$.@\color@var$next$[@\color@var$r$] = @\color@var$u$.@\color@var$next$[@\color@var$r$].@\color@var$next$[@\color@var$r$];
			}
			@\color@var$r$--;
		}
		@\color@keyword$if$ (@\color@var$removed$) @\color@var$n$--;
		@\color@var$return$ @\color@var$removed$;
	}
\end{Verbatim}

\begin{figure}
  \begin{center}
    \includegraphics{figs/skiplist-remove}
  \end{center}
  \caption{Removing the node 3 from a skiplist.}
  \figlabel{skiplist-remove}
\end{figure}

\subsection{Summary}

The following theorem summarizes the performance of skiplists when used to
implement sorted sets:

\begin{thm}\thmlabel{skiplist}
A \mbox{\texttt{SkiplistSet}} implements the \mbox{\texttt{SortedSet}} interface. A \mbox{\texttt{SkiplistSet}} supports
the operations \mbox{\texttt{add({\color{var}x})}}, \mbox{\texttt{remove({\color{var}x})}}, and \mbox{\texttt{find({\color{var}x})}} in $O(\log n)$
expected time per operation.
\end{thm}

\section{Skiplists as Lists}

A \mbox{\texttt{SkiplistList}} is a realization of the skiplist idea that implements
the \mbox{\texttt{List}} interface.  In a \mbox{\texttt{SkiplistList}}, $L_0$ contains the elements of the
list in the order they appear in the list.   Just like a \mbox{\texttt{SkiplistSet}}
elements can be added, removed, and accessed in $O(\log \mbox{\texttt{{\color{var}n}}})$ time.

For this idea to work, we need a way to follow the search path for
the \mbox{\texttt{{\color{var}i}}}th element in $L_0$.  The easiest way to do this is to define
the notion of the \emph{length} of an edge in some list $L_{\mbox{\texttt{{\color{var}r}}}}$.
The length of every edge in $L_{0}$ is 1.  The length of an edge \mbox{\texttt{{\color{var}e}}}
in $L_{\mbox{\texttt{{\color{var}r}}}}$, $\mbox{\texttt{{\color{var}r}}}>0$ is the sum of the lengths of the edges below \mbox{\texttt{{\color{var}e}}}
in $L_{\mbox{\texttt{{\color{var}r}}}-1}$.  Equivalently, the length of an edge \mbox{\texttt{{\color{var}e}}} in $L_{\mbox{\texttt{{\color{var}r}}}}$ is
the number of edges in $L_0$ below \mbox{\texttt{{\color{var}e}}}.  See \figref{skiplist-lengths} for
an example of a skiplist with the lengths of its edges shown.  Since the
edges of skiplists are stored in arrays, the lengths can be stored the same
way:

\begin{figure}
  \begin{center}
    \includegraphics{figs/skiplist-lengths}
  \end{center}
  \caption{The lengths of the edges in a skiplist.}
  \figlabel{skiplist-lengths}
\end{figure}

\begin{Verbatim}[tabsize=2,frame=single,commandchars=\\@\$,label=\texttt{SkiplistList},labelposition=topline]
	@\color@keyword$class$ Node {
		@\color@keyword$T$ @\color@var$x$;
		Node[] @\color@var$next$;
		@\color@keyword$int$[] @\color@var$length$;
		Node(@\color@keyword$T$ @\color@var$ix$, @\color@keyword$int$ @\color@var$h$) {
			@\color@var$x$ = @\color@var$ix$;
			@\color@var$next$ = (Node[])Array.newInstance(Node.@\color@keyword$class$, @\color@var$h$+1);
			@\color@var$length$ = @\color@keyword$new$ @\color@keyword$int$[@\color@var$h$+1];
		}
	}
\end{Verbatim}

The useful property of this definition of length is that, if we are
currently at a node that is at position \mbox{\texttt{{\color{var}j}}} in $L_0$ and we follow an
edge of length $\ell$, then we move to a node whose position, in $L_0$,
is $\mbox{\texttt{{\color{var}j}}}+\ell$.  In this way, while following a search path, we can keep
track of the position, \mbox{\texttt{{\color{var}j}}}, of the current node in $L_0$.  When at a
node \mbox{\texttt{{\color{var}u}}} in $L_{\mbox{\texttt{{\color{var}r}}}}$ we go right if \mbox{\texttt{{\color{var}j}}}
plus the length of the edge \mbox{\texttt{{\color{var}u}.{\color{var}next}}} is less than \mbox{\texttt{{\color{var}i}}}, otherwise we go
down into $L_{\mbox{\texttt{{\color{var}r}}}-1}$.

\begin{Verbatim}[tabsize=2,frame=single,commandchars=\\@\$,label=\texttt{SkiplistList},labelposition=topline]
	Node findPred(@\color@keyword$int$ @\color@var$i$) {
		Node @\color@var$u$ = @\color@var$sentinel$;
		@\color@keyword$int$ @\color@var$r$ = @\color@var$height$ - 1;
		@\color@keyword$int$ @\color@var$j$ = -1;   @\color@comment$// the index of the current node in list 0$
		@\color@keyword$while$ (@\color@var$r$ >= 0) {
			@\color@keyword$while$ (@\color@var$u$.@\color@var$next$[@\color@var$r$] != @\color@var$null$ && @\color@var$j$ + @\color@var$u$.@\color@var$length$[@\color@var$r$] < @\color@var$i$) {
				@\color@var$j$ += @\color@var$u$.@\color@var$length$[@\color@var$r$];
				@\color@var$u$ = @\color@var$u$.@\color@var$next$[@\color@var$r$];
			}
			@\color@var$r$--;
		}
		@\color@var$return$ @\color@var$u$;
	}
\end{Verbatim}
\begin{Verbatim}[tabsize=2,frame=single,commandchars=\\@\$,label=\texttt{SkiplistList},labelposition=topline]
	@\color@keyword$T$ get(@\color@keyword$int$ @\color@var$i$) {
		@\color@var$return$ findPred(@\color@var$i$).@\color@var$next$[0].@\color@var$x$;
	}
\end{Verbatim}
\begin{Verbatim}[tabsize=2,frame=single,commandchars=\\@\$,label=\texttt{SkiplistList},labelposition=topline]
	@\color@keyword$T$ set(@\color@keyword$int$ @\color@var$i$, @\color@keyword$T$ @\color@var$x$) {
		Node @\color@var$u$ = findPred(@\color@var$i$).@\color@var$next$[0];
		@\color@keyword$T$ @\color@var$y$ = @\color@var$u$.@\color@var$x$;
		@\color@var$u$.@\color@var$x$ = @\color@var$x$;
		@\color@var$return$ @\color@var$y$;
	}
\end{Verbatim}

Since the hardest part of the operations \mbox{\texttt{get({\color{var}i})}} and \mbox{\texttt{set({\color{var}i},{\color{var}x})}} is
finding the \mbox{\texttt{{\color{var}i}}}th node in $L_0$, these operations therefore run in
$O(\log \mbox{\texttt{{\color{var}n}}})$ time.

Adding an element to a \mbox{\texttt{SkiplistList}} is fairly straightforward.  We use
the same method as before. We first pick the height \mbox{\texttt{{\color{var}k}}} of the newly
inserted node and the follow the search path for \mbox{\texttt{{\color{var}i}}}.  Anytime the
search path moves down from $L_{\mbox{\texttt{{\color{var}r}}}}$ with $\mbox{\texttt{{\color{var}r}}}\le \mbox{\texttt{{\color{var}k}}}$, we splice the new
node into $L_{\mbox{\texttt{{\color{var}r}}}}$.  The only extra care needed is to ensure that the
lengths of edges are updated properly. 
See \figref{skiplist-addix}. 

\begin{figure}
  \begin{center}
    \includegraphics{figs/skiplist-addix}
  \end{center}
  \caption{Adding an element to a \mbox{\texttt{SkiplistList}}.}
  \figlabel{skiplist-addix}
\end{figure}

Note that, each time the search path goes down at a node \mbox{\texttt{{\color{var}u}}} in $L_{\mbox{\texttt{{\color{var}r}}}}$,
the length of the edge \mbox{\texttt{{\color{var}u}.{\color{var}next}[{\color{var}r}]}} increases by one, since we are adding
an element at position \mbox{\texttt{{\color{var}i}}}.  Splicing  the node \mbox{\texttt{{\color{var}w}}} between two nodes
\mbox{\texttt{{\color{var}u}}} and \mbox{\texttt{{\color{var}z}}} works as shown in \figref{skiplist-lengths-splice}.  While
following the search path we are already keeping track of the position,
\mbox{\texttt{{\color{var}j}}}, of \mbox{\texttt{{\color{var}w}}} in $L_0$.  Therefore, we know that the length of the edge from
\mbox{\texttt{{\color{var}u}}} to \mbox{\texttt{{\color{var}w}}} is $\mbox{\texttt{{\color{var}i}}}-\mbox{\texttt{{\color{var}j}}}$.  We can also deduce that the length of the edge
from \mbox{\texttt{{\color{var}w}}}  to \mbox{\texttt{{\color{var}z}}} from the length, $\ell$, of the edge from \mbox{\texttt{{\color{var}u}}} to \mbox{\texttt{{\color{var}z}}}.
Therefore, we can splice in \mbox{\texttt{{\color{var}w}}} and update the lengths of the edges in
constant time.

\begin{figure}
  \begin{center}
    \includegraphics{figs/skiplist-lengths-splice}
  \end{center}
  \caption{Updating the lengths of edges while splicing a node 
   \mbox{\texttt{{\color{var}w}}} into a skiplist.}
  \figlabel{skiplist-lengths-splice}
\end{figure}

This sounds more complicated than it actually is, and the code is actually
quite simple:

\begin{Verbatim}[tabsize=2,frame=single,commandchars=\\@\$,label=\texttt{SkiplistList},labelposition=topline]
	@\color@var$void$ add(@\color@keyword$int$ @\color@var$i$, @\color@keyword$T$ @\color@var$x$) {
		Node @\color@var$w$ = @\color@keyword$new$ Node(@\color@var$x$, pickHeight());
		@\color@keyword$if$ (@\color@var$w$.@\color@var$next$.@\color@var$length$ > @\color@var$height$) 
			@\color@var$height$ = @\color@var$w$.@\color@var$next$.@\color@var$length$;
		add(@\color@var$i$, @\color@var$w$);
	}
\end{Verbatim}
\begin{Verbatim}[tabsize=2,frame=single,commandchars=\\@\$,label=\texttt{SkiplistList},labelposition=topline]
	Node add(@\color@keyword$int$ @\color@var$i$, Node @\color@var$w$) {
		@\color@keyword$if$ (@\color@var$i$ < 0 || @\color@var$i$ > @\color@var$n$) @\color@keyword$throw$ @\color@keyword$new$ IndexOutOfBoundsException();
		Node @\color@var$u$ = @\color@var$sentinel$;
		@\color@keyword$int$ @\color@var$k$ = @\color@var$w$.@\color@var$next$.@\color@var$length$ - 1;
		@\color@keyword$int$ @\color@var$r$ = @\color@var$height$ - 1;
		@\color@keyword$int$ @\color@var$j$ = -1; @\color@comment$// index of u$
		@\color@keyword$while$ (@\color@var$r$ >= 0) {
			@\color@keyword$while$ (@\color@var$u$.@\color@var$next$[@\color@var$r$] != @\color@var$null$ && @\color@var$j$+@\color@var$u$.@\color@var$length$[@\color@var$r$] < @\color@var$i$) {
				@\color@var$j$ += @\color@var$u$.@\color@var$length$[@\color@var$r$];
				@\color@var$u$ = @\color@var$u$.@\color@var$next$[@\color@var$r$];
			}
			@\color@var$u$.@\color@var$length$[@\color@var$r$]++;    @\color@comment$// to account for new node in list 0$
			@\color@keyword$if$ (@\color@var$r$ <= @\color@var$k$) {
				@\color@var$w$.@\color@var$next$[@\color@var$r$] = @\color@var$u$.@\color@var$next$[@\color@var$r$];
				@\color@var$u$.@\color@var$next$[@\color@var$r$] = @\color@var$w$;
				@\color@var$w$.@\color@var$length$[@\color@var$r$] = @\color@var$u$.@\color@var$length$[@\color@var$r$] - (@\color@var$i$ - @\color@var$j$);
				@\color@var$u$.@\color@var$length$[@\color@var$r$] = @\color@var$i$ - @\color@var$j$;
			}
			@\color@var$r$--;
		}
		@\color@var$n$++;
		@\color@var$return$ @\color@var$u$;
	}
\end{Verbatim}


By now, the implementation of 
the \mbox{\texttt{remove({\color{var}i})}} operation in a \mbox{\texttt{SkiplistList}} should be obvious.  We follow the search path for the node at position \mbox{\texttt{{\color{var}i}}}.  Each time the search path takes a step down from a node \mbox{\texttt{{\color{var}u}}} at level \mbox{\texttt{{\color{var}r}}} we decrement the length of the edge leaving \mbox{\texttt{{\color{var}u}}} at level \mbox{\texttt{{\color{var}r}}}.  We also check if \mbox{\texttt{{\color{var}u}.{\color{var}next}}} is the element of rank \mbox{\texttt{{\color{var}i}}} and, if so, splice it out the list at level \mbox{\texttt{{\color{var}i}}}.   An example is shown in \figref{skiplist-removei}.
\begin{figure}
  \begin{center}
    \includegraphics{figs/skiplist-removei}
  \end{center}
  \caption{Removing an element from a \mbox{\texttt{SkiplistList}}.}
  \figlabel{skiplist-removei}
\end{figure}
\begin{Verbatim}[tabsize=2,frame=single,commandchars=\\@\$,label=\texttt{SkiplistList},labelposition=topline]
	@\color@keyword$T$ remove(@\color@keyword$int$ @\color@var$i$) {
		@\color@keyword$if$ (@\color@var$i$ < 0 || @\color@var$i$ > @\color@var$n$-1) @\color@keyword$throw$ @\color@keyword$new$ IndexOutOfBoundsException();
		@\color@keyword$T$ @\color@var$x$ = @\color@var$null$;
		Node @\color@var$u$ = @\color@var$sentinel$;
		@\color@keyword$int$ @\color@var$r$ = @\color@var$height$ - 1;
		@\color@keyword$int$ @\color@var$j$ = -1; @\color@comment$// index of node u$
		@\color@keyword$while$ (@\color@var$r$ >= 0) {
			@\color@keyword$while$ (@\color@var$u$.@\color@var$next$[@\color@var$r$] != @\color@var$null$ && @\color@var$j$+@\color@var$u$.@\color@var$length$[@\color@var$r$] < @\color@var$i$) {
				@\color@var$j$ += @\color@var$u$.@\color@var$length$[@\color@var$r$];
				@\color@var$u$ = @\color@var$u$.@\color@var$next$[@\color@var$r$];
			}
			@\color@var$u$.@\color@var$length$[@\color@var$r$]--;  @\color@comment$// for the node we are removing$
			@\color@keyword$if$ (@\color@var$j$ + @\color@var$u$.@\color@var$length$[@\color@var$r$] == @\color@var$i$ && @\color@var$u$.@\color@var$next$[@\color@var$r$] != @\color@var$null$) {
				@\color@var$x$ = @\color@var$u$.@\color@var$next$[@\color@var$r$].@\color@var$x$;
				@\color@var$u$.@\color@var$length$[@\color@var$r$] += @\color@var$u$.@\color@var$next$[@\color@var$r$].@\color@var$length$[@\color@var$r$];
				@\color@var$u$.@\color@var$next$[@\color@var$r$] = @\color@var$u$.@\color@var$next$[@\color@var$r$].@\color@var$next$[@\color@var$r$];
			}
			@\color@var$r$--;
		}
		@\color@var$n$--;
		@\color@var$return$ @\color@var$x$;
	}
\end{Verbatim}

\subsection{Summary}


\section{Skiplists as Ropes}

\section{Analysis of Skiplists}
\seclabel{skiplist-analysis}

In this section, we analyze the expected height, size, and length of
the search path in a skiplist.  This section requires a background in
basic probability.

\begin{lem}
  The expected height of a skiplist containing $n$ elements is at most
  $\log_2 n + 3$.
\end{lem}

\begin{proof}
  Define the indicator random variable
  \[ I_r = \left\{\begin{array}{ll}
     0 & \mbox{if $L_r$ is empty} \\
     1 & \mbox{if $L_r$ is non-empty}
     \end{array}\right.
  \]
  The height $h$ of the skiplist is then given by
  \[
       h = \sum_{i=0}^\infty I_r \enspace .
  \]
  Note that $I_r \le |L_r|$, so 
  \[
     \E[I_r] \le \E[|L_r|] = n/2^r \enspace .
  \]
  Therefore, we have
  \begin{eqnarray*}
       \E[h] &=& \E\left[\sum_{r=0}^\infty I_r\right] \\
        &=& \sum_{r=0}^{\infty} E[I_r] \\
        &=& \sum_{r=0}^{\lfloor\log n\rfloor} E[I_r]
                 + \sum_{r=\lfloor\log n\rfloor+1}^{\infty} E[I_r]  \\
        &=& \sum_{r=0}^{\lfloor\log n\rfloor} 1
                 + \sum_{r=\lfloor\log n\rfloor+1}^{\infty} n/2^r \\
        &\le& \log n + 1
                 + \sum_{r=0}^\infty 1/2^r \\
        &=& \log n + 3
  \end{eqnarray*}
\end{proof}

\begin{lem}
  The expected number of nodes in a skiplist containing $n$ elements,
  including all occurrences of the sentinel is $2n+O(\log n)$.
\end{lem}

\begin{proof}
The expected number of nodes, not including the sentinel is 
\[
  \E\left[\sum_{i=0}^{\infty} |L_i|\right] = 
  \sum_{i=0}^{\infty} n/2^i = 2n \enspace .
\]
The number of occurrences of the sentinel is equal to the height, $h$,
of the skiplist, so the expected number of occurrences of the sentinel
is at most $\log n+3 = O(\log n)$.
\end{proof}

\begin{lem}
The expected length of a search path in a skiplist is $2\log n + O(1)$.
\end{lem}

\begin{proof}
  The easiest way to see this is to consider the \emph{reverse search
  path} for a node \mbox{\texttt{{\color{var}x}}}.  This path starts at the predecessor of \mbox{\texttt{{\color{var}x}}}
  in $L_0$.  At any point in time, if the path can go up a level, then
  it does.  If it can not go up a level then it goes left.  Observe that
  the reverse search path for \mbox{\texttt{{\color{var}x}}} is identical to the search path for \mbox{\texttt{{\color{var}x}}},
  except that it is reversed.
  
  The number of nodes that the reverse search path visits at a particular
  level $r$ is related to the following experiment:  Toss a coin.
  If the coin comes up heads then go up and stop, otherwise go left and
  repeat the experiment.  The number of coin tosses before the heads then
  represents the number of steps to the left that a reverse search path
  takes at a particular level.  (Note that this overcounts the number of
  steps to the left, since this experiment ends either at the first head
  or when the search path reaches the sentinel, whichever comes first.)
  
  Let $R_r$ denote the number of steps the search path takes at level
  $r$ that go to the right.   We have just argued that $\E[R_r]\le 1$.
  Furthermore, $R_r\le |L_r|$, so
  \[
    \E[R_r] \le E[|L_r|] = n/2^r \enspace .
  \]
  We can now finish as in the proof of \lemref{skiplist-height}.
  Let $\ell$ be  the length of the search path for some node \mbox{\texttt{{\color{var}u}}} in a
  skiplist, and let $h$ be the height of the skiplist.  Then
  \begin{eqnarray*}
      \E[\ell] 
         &=& \E\left[ h + \sum_{r=0}^\infty R_r \right] \\
         &=& \E[h] + \sum_{r=0}^\infty \E[R_r]  \\
         &=& \E[h] + \sum_{r=0}^{\lfloor\log n\rfloor} \E[R_r] 
              + \sum_{r=\lfloor\log n\rfloor+1}^\infty \E[R_r] \\
         &\le& \E[h] + \sum_{r=0}^{\lfloor\log n\rfloor} 1
              + \sum_{r=\lfloor\log n\rfloor+1}^\infty n/2^r \\
         &\le& \E[h] + \sum_{r=0}^{\lfloor\log n\rfloor} 1
              + \sum_{r=0}^{\infty} 1/2^r \\
         &\le& \E[h] + \sum_{r=0}^{\lfloor\log n\rfloor} 1
              + \sum_{r=0}^{\infty} 1/2^r \\
         &\le& \E[h] + \log n + 3 \\
         &\le& 2\log n + 6
  \end{eqnarray*}
  as required.
\end{proof}


\section{Iteration and Finger Search}

%TODO: Write this section

\section{Summary and Exercises}

\begin{enumerate}
\item Suppose that, instead of promoting an element from $L_{i-1}$ into
$L_i$, we promote it with some probability $p$, $0 < p < 1$.  Show that
the expected length of the search path in this case is $(1/p)\log_{1/p}
n + O(1)$.  What is the value of $p$ that minimizes this expression? What
is the expected height of the skiplist? What is the expected number of
nodes in the skiplist?

\item Design and implement a version of a skiplist that implements the
\mbox{\texttt{SortedSet}} interface, but also allows access to elements by rank.
That is, it also supports the function \mbox{\texttt{get({\color{var}i})}}, which returns the element
whose rank is \mbox{\texttt{{\color{var}i}}}. (The rank of an element \mbox{\texttt{{\color{var}x}}} in a \mbox{\texttt{SortedSet}} is the number of elements in the \mbox{\texttt{SortedSet}} that are less than \mbox{\texttt{{\color{var}x}}}.)
\end{enumerate}

