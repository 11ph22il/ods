\chapter{Introduction}
\pagenumbering{arabic}

This chapter briefly reviews some of the main concepts used throughout
the rest of the book.   \secref{interfaces} describes the interfaces
implemented by all the data structures described in this book.
It should be considered required reading.  The remaining sections discuss
asymptotic (big-Oh) notation, probability and randomization, the model of
computation, and the sample code and typesetting conventions.  A reader
with or without a background in these areas can easily skip them now and
come back to them later if necessary.

\section{Interfaces}
\seclabel{interfaces}

In discussing data structures, it is important to understand the
difference between a data structure's interface and its implementation.
An interface describes what a data structure does, while an implementation
describes how the data structure does it.

An \emph{interface}, sometimes also called an \emph{abstract data type},
defines the set of operations supported by a data structure and
the semantics, or meaning, of those operations.  An interface tells us
nothing about how the data structure implements these operations, it only
provides the list of supported operations along with specifications
about what types of arguments each operation accepts and the value
returned by each operation.

A data structure \emph{implementation} on the other hand, includes the
internal representation of the data structure as well as the definitions
of the algorithms that implement the operations supported by the data
structure.  Thus, there can be many implementations of a single interface.
For example, in \chapref{arrays}, we will see implementations of the
#List# interface using arrays and in \chapref{linkedlists} we will
see implementations of the #List# interface using pointer-based data
structures.  Each implements the same interface, #List#,
but in different ways.

\subsection{The #Queue#, #Stack#, and #Deque# Interfaces}

The #Queue# interface represents a collection of elements to which we
can add elements and remove the next element.  More precisely, the operations
supported by the #Queue# interface are
\begin{itemize}
  \item #add(x)#: add the value #x# to the #Queue#
  \item #remove()#: remove the next (previously added) value, #y#, from the #Queue# and return #y#
\end{itemize}
Notice that the #remove()# operation takes no argument.  The #Queue#'s
\emph{queueing discipline} decides which element should be removed.
There are many possible queueing disciplines, the most common of which
include FIFO, priority, and LIFO.

A \emph{FIFO (first-in-first-out) #Queue#} removes items in the same
order they were added, much in the same way a queue (or line-up) works
when checking out at a cash register in a grocery store.

A \emph{priority #Queue#} always removes the smallest element from
the #Queue#, breaking ties arbitrarily.  This is similar to the way
many airlines manage upgrades to the business class on their flights.
When a business-class seat becomes available it is given to the most
important customer waiting on an upgrade.

A very common queueing discipline is the LIFO (last-in-first-out)
discipline.  In a \emph{LIFO Queue}, the most recently added element is
the next one removed.  This is best visualized in terms of a stack of
plates; plates are placed on the top of the stack and also removed from
the top of the stack. This structure is so common that it gets its own
name: #Stack#.  Often, when discussing a #Stack#, the names of #add(x)#
and #remove()# are changed to #push(x)# and #pop()#; this is to avoid
confusing the LIFO and FIFO queueing disciplines.

A #Deque# is a generalization of both the FIFO #Queue# and LIFO #Queue#
(#Stack#).   A #Deque# represents a sequence of elements, with a front
and a back.  Elements can be added at the front of the sequence or
the back of the sequence.  The names of the operations on a #Deque#
are self-explanatory: #addFirst(x)#, #removeFirst()#, #addLast(x)#,
and #removeLast()#.  Notice that a #Stack# can be implemented using only
#addFirst(x)# and #removeFirst()# while a FIFO #Queue# can be implemented
using only #addLast(x)# and #removeFirst()#.


\subsection{The #List# Interface: Linear Sequences}

This book will talk very little about the FIFO #Queue#, #Stack#, or
#Deque# interfaces.  This is because these interfaces are subsumed by
the #List# interface.  A #List# represents a sequence,
$#x#_0,\ldots,#x#_{#n#-1}$, of values.  The #List# interface includes
the following operations:

\begin{enumerate}
  \item #size()#: return #n#, the length of the list
  \item #get(i)#: return the value $#x#_{#i#}$
  \item #set(i,x)#: set the value of $#x#_{#i#}$ equal to #x#
  \item #add(i,x)#: add #x# at position #i#, displacing
    $#x#_{#i#},\ldots,#x#_{#n#-1}$; \\ 
    Set $#x#_{j+1}=#x#_j$, for all
    $j\in\{#n#-1,\ldots,#i#\}$, increment #n#, and set $#x#_i=#x#$
  \item #remove(i)# remove the value $#x#_{#i#}$, displacing
    $#x#_{#i+1#},\ldots,#x#_{#n#-1}$; \\ 
    Set $#x#_{j}=#x#_{j+1}$, for all
    $j\in\{#i#,\ldots,#n#-2\}$ and decrement #n#
\end{enumerate}
Notice that these operations are easily sufficient to implement the
#Deque# interface:
\begin{eqnarray*}
  #addFirst(x)# &\Rightarrow& #add(0,x)# \\
  #removeFirst()# &\Rightarrow& #remove(0)#  \\
  #addLast(x)# &\Rightarrow& #add(size(),x)# \\
  #removeLast()# &\Rightarrow& #remove(size()-1)#
\end{eqnarray*}

Although we will normally not discuss the #Stack#, #Deque# and FIFO
#Queue# interfaces in subsequent chapters, the terms #Stack# and #Deque#
are sometimes used in the names of data structures that implement the
#List# interface.  When this happens, it is to highlight the fact that
these data structures can be used to implement the #Stack# or #Deque#
interface very efficiently.  For example, the #ArrayDeque# class is an
implementation of the #List# interface that can implement the #Deque#
operations in constant (amortized) time per operation.


\subsection{The #USet# Interface: Unordered Sets}

The #USet# interface represents an unordered set of unique elements, mimicking a
mathematical \emp{set}. A #USet# contains #n#
\emph{distinct} elements; no element appears more than once; the elements
are in no specific order.  A #USet# supports the following operations:

\begin{enumerate}
  \item #size()#: return the number, #n#, of elements in the set
  \item #add(x)#: add the element #x# to the set if not already present; \\
    Add #x# to the set provided that there
    is no element #y# in the set such that #x# equals #y#.  Return #true#
    if #x# was added to the set and #false# otherwise.
  \item #remove(x)#: remove #x# from the set; \\
    Find an element #y# in the set such that #x# equals
    #y# and remove #y#.  Return #y#, or #null# if no such element exists.
  \item #find(x)#: find #x# in the set if it exists; \\
    Find an element #y# in the set such that #y# equals
    #x#.  Return #y#, or #null# if no such element exists.
\end{enumerate}

These definitions are a bit fussy about distinguishing #x#, the element
we are removing or finding, from #y#, the element we remove or find.
This is because #x# and #y# might actually be distinct objects that
are nevertheless treated as equal.\javaonly{\footnote{In Java, this is
done by overriding the class' #equals(y)# and #hashCode()# methods.}}
This is a very useful distinction since it allows for the creation of
\emph{dictionaries} or \emph{maps} that map keys onto values.  This is
done by creating a compound object called a #Pair# that contains a
\emph{key} and a \emph{value}. Two #Pair#s are treated as equal if their
keys are equal.  By storing #Pair#s in a #USet#, we can find the value
associated with any key #k# by creating a #Pair#, #x#, with key #k#
and using the #find(x)# method.


\subsection{The #SSet# Interface: Sorted Sets}
\seclabel{sset}

The #SSet# interface represents a sorted set of elements.  An #SSet#
stores elements from some total order, so that any two elements #x#
and #y# can be compared.  In code examples, this will be done with a
method called #compare(x,y)# in which
\[
    #compare(x,y)# 
      \begin{cases}
        {}<0 & \text{if $#x#<#y#$} \\
        {}>0 & \text{if $#x#>#y#$} \\
        {}=0 & \text{if $#x#=#y#$}
      \end{cases}
\]
An #SSet# supports the #size()#, #add(x)#, and #remove(x)# methods with
exactly the same semantics as in the #USet# interface.  The difference
between a #USet# and an #SSet# is in the #find(x)# method:
\begin{enumerate}
\setcounter{enumi}{3}
\item #find(x)#: locate #x# in the sorted set; \\
   Find the smallest element #y# in the set such that $#y# \ge #x#$.
   Return #y# or #null# if no such element exists.
\end{enumerate}

This version of the #find(x)# operation is sometimes referred to
as a \emph{successor search}.  It differs in a fundamental way from
#USet.find(x)# since it returns a meaningful result even when there is
no element in the set that is equal to #x#.

The distinction between the #USet# and #SSet# #find(x)# operations is very
important and is very often missed.  The extra functionality provided
by an #SSet# usually comes with a price that includes both a larger
running time and a higher implementation complexity.  For example, most
of the #SSet# implementations discussed in this book all have #find(x)#
operations with running times that are logarithmic in the size of the set.
On the other hand, the implementation of a #USet# as a #ChainedHashTable#
in \chapref{hashing} has a #find(x)# operation that runs in constant
expected time.  When choosing which of these structures to use, one should
always use a #USet# unless the extra functionality offered by an #SSet#
is really needed.



% \subsection{The DynamicString Interface}

\section{Mathematical Background}

In this section, we review some mathematical notations and tools
used throughout this book, including logarithms, big-Oh notation, and
probability theory.

\subsection{Logarithms}

In this book, the expression $\log_b k$ denotes the \emph{base-$b$ logarithm}
of $k$.  That is, the unique value $x$ that satisfies
\[
    b^{x} = k  \enspace .
\]
Most of the logarithms in this book are base 2 (\emph{binary logarithms}),
in which case we drop the base, so that $\log k$ is shorthand for
$\log_2 k$.

Another logarithm that comes up several times in this book is the
\emph{natural logarithm}.  Here we use the notation $\ln k$ to denote
$\log_e k$, where $e$ --- \emph{Euler's constant} --- is given by
\[
   e = \lim_{n\rightarrow\infty} \left(1+\frac{1}{n}\right)^n
   \approx  2.71828 \enspace .
\]
The natural logarithm comes up frequently because it is the value
of a particularly common integral:
\[
    \int_{1}^{k} 1/x\,\mathrm{d}x  = \ln k \enspace .
\]
Two of the most common manipulations we do with logarithms are removing
them from an exponent:
\[
    b^{\log_b k} = k
\]
and changing the base of a logarithm:
\[
    \log_b k = \frac{\log_a k}{\log_a b} \enspace .
\]
For example, we can use these two manipulations to compare the natural and binary logarithms
\[
   \ln k = \frac{\log k}{\log e} = \frac{\log k}{(\ln e)/(\ln 2)} = 
    (\ln 2)(\log k) \approx 0.693147\log k \enspace .
\]

\subsection{Factorials}
\seclabel{factorials}

In one or two places in this book, the \emph{factorial} function is used.
For a non-negative integer $n$, the notation $n!$ (pronounced ``$n$ factorial'') denotes
\[
   n! = 1\cdot2\cdot3\cdot\cdots\cdot n \enspace .
\]
Factorials appear because $n!$ counts the number of distinct
permutations, i.e., orderings, of n distinct elements.  For the special case $n=0$, $0!$ is defined as 1. 

The quantity $n!$ can be approximated using \emph{Stirling's Approximation}:
\[
	n! 
   = \sqrt{2\pi n}\left(\frac{n}{e}\right)^{n}e^{\alpha(n)} \enspace ,
\]
where
\[  
   \frac{1}{12n+1} <  \alpha(n) < \frac{1}{12n}  \enspace .
\]
Stirling's Approximation also approximates $\ln(n!)$:
\[
   \ln(n!) = n\ln n - n + \frac{1}{2}\ln(2\pi n) + \alpha(n)
\]
(In fact, Stirling's Approximation is most easily proven by approximating
$\ln(n!)=\ln 1 + \ln 2  + \cdots + \ln n$ by the integral
$\int_1^n \ln n\,\mathrm{d}n = n\ln n - n +1$.)

Related to the factorial function are the \emph{binomial coefficients}.
For a non-negative integer $n$ and an integer $k\in\{0,\ldots,n\}$,
the notation $\binom{n}{k}$ denotes:
\[
   \binom{n}{k} = \frac{n!}{k!(n-k)!} \enspace .
\]
The binomial coefficient $\binom{n}{k}$ (pronounced ``$n$ choose $k$'')
counts the number of subsets of an $n$ element set that have size $k$,
i.e., the number of ways of choosing $k$ distinct integers from the
set $\{1,\ldots,n\}$.

\subsection{Asymptotic Notation}

When analyzing data structures in this book, we will want to talk about
the running times of various operations.  The exact running times will,
of course, vary from computer to computer and even from run to run on
an individual computer.  Therefore, instead of analyzing running times
exactly, we will use the so-called \emph{big-Oh notation}: For a 
function $f(n)$, $O(f(n))$ denotes a set of functions,
\[
   O(f(n)) = \{g(n):\mbox{there exists $c>0$, and $n_0$ such that
             $g(n) \le c\cdot f(n)$ for all $n\ge n_0$}   \} \enspace .
\]
Thinking graphically, this set consists of the functions $g(n)$ where
$c\cdot f(n)$ starts to dominate $g(n)$ when $n$ is sufficiently large.

We generally use asymptotic notation to simplify functions.  For example,
in place of $5n\log n + 8n - 200$ we can write, simply,  $O(n\log n)$.
This is proven as follows:
\begin{align*} 
       5n\log n + 8n - 200
        & \le 5n\log n + 8n \\
        & \le 5n\log n + 8n\log n & \mbox{ for $n\ge 2$ (so that $\log n \ge 1$)}
            \\
        & \le 13n\log n 
\end{align*}
which demonstrates that the function $f(n)=5n\log n + 8n - 200$ is in the set $O(\log n)$ using the constants $c=13$ and $n_0 = 2$.

There are a number of useful shortcuts when using asymptotic notation.  First:
\[ O(n^{c_1}) \subset O(n^{c_2}) \enspace ,\]
 for any $c_1 < c_2$.  Second:
For any constants $a,b,c > 0$,
\[ O(a) \subset O(\log n) \subset O(n^{b}) \subset O({c}^n) \enspace . \]
These inclusion relations can be multiplied by any positive value, and they still hold. For example, multiplying by $n$ yields:
\[ O(n) \subset O(n\log n) \subset O(n^{1+b}) \subset O(n{c}^n) \enspace . \]

Continuing in a long and distinguished tradition,
we will abuse this notation by writing things
like $f_1(n) = O(f(n))$ when what we really mean is $f_1(n) \in O(f(n))$.
We will also make statements like ``the running time of this operation
is $O(f(n))$'' when this statement should be ``the running time of
this operation is a member of $O(f(n))$.''  These shortcuts are mainly
to avoid awkward language and to make it easier to use asymptotic
notation within strings of equations.

A particularly strange example of this comes when we write statements like
\[
     T(n) = 2\log n + O(1)  \enspace .
\]
Again, this would be more correctly written as
\[
     T(n) \le 2\log n + [\mbox{some member of $O(1)$]}  \enspace .
\]

The expression $O(1)$ also brings up another issue. Since there is
no variable in this expression, it may not be clear what variable is
getting arbitrarily large.  Without context, there is no way to tell.
In the example above, since the only variable in the rest of the equation
is $n$, we can assume that this should be read as $T(n) = 2\log n + O(f(n))$,
where $f(n) = 1$.

In a few cases, we will use asymptotic notation on functions with more than one variable. There seems to be no standard for this, but for our purposes, the following definition is sufficient:
\[
   O(f(n_1,\ldots,n_k)) = 
   \left\{\begin{array}{lll}
             g(n_1,\ldots,n_k):\mbox{there exists $c>0$, and $z$ such that} \\
             \quad \mbox{$g(n_1,\ldots,n_k) \le c\cdot f(n_1,\ldots,n_k)$} \\
             \qquad \mbox{for all $n_1,\ldots,n_k$ such that $g(n_1,\ldots,n_k)\ge z$}   
   \end{array}\right\} \enspace .
\]
This definition captures the situation we really care about:  when the
arguments $n_1,\ldots,n_k$ make $g$ take on large values.  This agrees
with the univariate definition of $O(f(n))$ when $f(n)$ is an increasing
function of $n$.  The reader should be warned that, although this works
for our purposes, other texts may treat multivariate functions and
asymptotic notation differently.


\subsection{Randomization and Probability}
\seclabel{randomization}

Some of the data structures presented in this book are \emph{randomized};
they make random choices that are independent of the data being stored
in them or the operations being performed on them.  For this reason,
performing the same set of operations more than once using these
structures could result in different running times.  When analyzing these
data structures we are interested in their average or \emph{expected}
running times.

Formally, the running time of an operation on a randomized data structure
is a random variable and we want to study its \emph{expected value}. For
a discrete random variable $X$ taking on values in some countable universe
$U$, the expected value of $X$, denoted by $\E[X]$ is given the by the formula
\[
    \E[X] = \sum_{x\in U} x\cdot\Pr\{X=x\} \enspace .
\]
Here $\Pr\{\mathcal{E}\}$ denotes the probability that the event
$\mathcal{E}$ occurs.  In all the examples in this book, these
probabilities are only with respect to whatever random choices are made
by the randomized data structure;  there is no assumption that the data
stored in the structure is random or that the sequence of operations
performed on the data structure is random.

One of the most important properties of expected values is \emph{linearity
of expectation}:  For any two random variables $X$ and $Y$,
\[
   \E[X+Y] = \E[X] + \E[Y] \enspace .
\]
More generally, for any random variables $X_1,\ldots,X_k$,
\[
   \E\left[\sum_{i=1}^k X_k\right] = \sum_{i=1}^k \E[X_i] \enspace .
\]
Linearity of expectation allows us to break down complicated random variables (like the left hand sides of the above equations) into sums of simpler random variables (the right hand sides).

A useful trick, that we will use repeatedly, is that of defining
\emph{indicator random variables}.  These binary variables are useful
when we want to count something and are best illustrated by an example.
Suppose we toss a fair coin $k$ times and we want to know the expected
number of times the coin comes up heads.  Intuitively, we know the answer
is $k/2$, but if we try to prove it using the definition of expected
value, we get
\begin{align*}
   \E[X] & = \sum_{i=0}^k i\cdot\Pr\{X=i\} \\
         & = \sum_{i=0}^k i\cdot\binom{k}{i}/2^k \\
         & = k\cdot \sum_{i=0}^{k-1}\binom{k-1}{i}/2^k \\
         & = k/2 \enspace .
\end{align*}
This requires that we know enough to calculate that $\Pr\{X=i\}
= \binom{k}{i}/2^k$, that we know the binomial identity
$i\binom{k}{i}=k\binom{k-1}{i}$, and that we know the binomial identity
$\sum_{i=0}^{k} \binom{k}{i} = 2^{k}$.

Using indicator variables and linearity of expectation makes things much easier:  For each $i\in\{1,\ldots,k\}$, define the indicator random variable
\[
    I_i = \begin{cases}
           1 & \text{if the $i$th coin toss is heads} \\
           0 & \text{otherwise.}
          \end{cases}
\]
Then 
\[ \E[I_i] = (1/2)1 + (1/2)0 = 1/2 \enspace . \]
Now, $X=\sum_{i=1}^k I_i$, so
\begin{align*}
   \E[X] & = \E\left[\sum_{i=1}^k I_i\right] \\
         & = \sum_{i=1}^k \E[I_i] \\
         & = \sum_{i=1}^k 1/2 \\
         & = k/2 \enspace .
\end{align*}
This is a bit more long-winded, but doesn't require that we know any
magical identities or compute any non-trivial probabilities. Even better:
It agrees with the intuition that we expect half the coins to come
up heads precisely because each individual coin has probability $1/2$ of coming
up heads.

\section{The Model of Computation}
\seclabel{model}

In this book, we will analyze the theoretical running times of operations
on the data structures we study.  To do this precisely, we need a
mathematical model of computation.  For this, we use the \emph{#w#-bit
word-RAM} model.  In this model, we have access to a random access memory
consisting of \emph{cells}, each of which stores a #w#-bit \emph{word}.
This implies a memory cell can represent, for example, any integer in
the set $\{0,\ldots,2^{#w#}-1\}$.

In the word-RAM model, basic operations on words take constant time.
This includes arithmetic operations (#+#, #-#, #*#, #/#, #%#), comparisons
($<$, $>$, $=$, $\le$, $\ge$), and bitwise boolean operations (bitwise-AND,
OR, and exclusive-OR).

Any cell can be read or written in constant time.  Our computer's memory
is managed by a memory management system from which we can allocate or
deallocate a block of memory of any size we like. Allocating a block
of memory of size $k$ takes $O(k)$ time and returns a reference to the
newly-allocated memory block.  This reference is small enough to be
represented by a single word.

The word-size, #w#, is a very important parameter of this model.  The only
assumption we will make on #w# is that it is at least $#w# \ge \log #n#$,
where #n# is the number of elements stored in any of our data structures.
This is a fairly modest assumption, since otherwise a word is not even
big enough to count the number of elements stored in the data structure.

Space is measured in words so that, when we talk about the amount of
space used by a data structure, we are referring to the number of words
of memory used by the structure.  All our data structures store values
of a generic type #T# and we assume an element of type #T# occupies one
word of memory.  (In reality, we are storing references to objects of
type #T#, and these references occupy only one word of memory.)

\javaonly{The #w#-bit word-RAM model is a fairly close match for the
(32-bit) Java Virtual Machine (JVM) when $#w#=32$.}\cpponly{The #w#-bit
word-RAM model is a fairly close match for modern desktop computers when
$#w#=32$ or $#w#=64$.} The data structures presented in this book don't
use any special tricks that are not implementable on the JVM and most
other architectures.

\section{Code Samples}

The code samples in this book are written in the \lang\ programming
language.  However to make the book accessible even to readers not
familiar with all of \lang's constructs and keywords, the code samples
have been simplified.  For example, a reader won't find any of the
keywords #public#, #protected#, #private#, or #static#.  A reader also
won't find much discussion about class hierarchies.  Which interfaces
a particular class implements or which class it extends, if relevant to
the discussion, will be clear from the accompanying text.

These conventions should make most of the code samples understandable by
anyone with a background in any of the languages from the ALGOL tradition,
including B, C, C++, C\#, D, Java, JavaScript, and so on.  Readers who want
the full details of all implementations are encouraged to look at the
\lang\ source code that accompanies this book.

This book mixes mathematical analysis of running times with \lang\
source code for the algorithms being analyzed.  This means that
some equations contain variables also found in the source code.
These variables are typeset consistently, both within the source code
and within equations.  The most common such variable is the variable #n#
that, without exception, always refers to the number of items currently
stored in the data structure.

\section{List of Data Structures}

The following table summarize the performance of data structures
described in this book that implement each of the interfaces, #List#,
#USet#, and #SSet#, described in \secref{interfaces}.

\begin{center}
\renewcommand{\thefootnote}{\Alph{footnote}}
\begin{tabular}{|l|l|l|l|} \hline
\multicolumn{4}{|c|}{#List# implementations} \\ \hline
 & #get(i)#/#set(i,x)# & #add(i,x)#/#remove(i)# & \\ \hline
#ArrayStack# & $O(1)$ & $O(1+#n#-#i#)$\footnotemark[1] & \sref{arraystack} \\
#ArrayDeque# & $O(1)$ & $O(1+\min\{#i#,#n#-#i#\})$\footnotemark[1] & \sref{arraydeque} \\
#DualArrayDeque# & $O(1)$ & $O(1+\min\{#i#,#n#-#i#\})$\footnotemark[1] & \sref{dualarraydeque}\\
#RootishArrayStack# & $O(1)$ & $O(1+#n#-#i#)$\footnotemark[1]  & \sref{rootisharraystack} \\
#DLList# & $O(1+\min\{#i#,#n#-#i#\})$ & $O(1+\min\{#i#,#n#-#i#\})$  & \sref{dllist} \\
#SEList# & $O(1+\min\{#i#,#n#-#i#\}/#b#)$ & $O(#b#+\min\{#i#,#n#-#i#\}/#b#)$\footnotemark[1]  & \sref{selist} \\
#SkiplistList# & $O(\log #n#)$\footnotemark[5] & $O(\log #n#)$\footnotemark[5]  & \sref{skiplistlist} \\ \hline
\multicolumn{4}{c}{} \\[2ex] \hline
\multicolumn{4}{|c|}{#USet# implementations} \\ \hline
 & #find(x)# & #add(x)#/#remove(x)# & \\ \hline
#ChainedHashTable# & $O(1)$\footnotemark[5] & $O(1)$\footnotemark[1]\footnotemark[5] & \sref{hashtable} \\ 
#LinearHashTable# & $O(1)$\footnotemark[5] & $O(1)$\footnotemark[1]\footnotemark[5] & \sref{linearhashtable} \\ \hline
\multicolumn{4}{c}{} \\[2ex] \hline
\multicolumn{4}{|c|}{#SSet# implementations} \\ \hline
 & #find(x)# & #add(x)#/#remove(x)# & \\ \hline
#SkiplistSSet# & $O(\log #n#)$\footnotemark[5] & $O(\log #n#)$\footnotemark[5] & \sref{skiplistset} \\ 
#Treap# & $O(\log #n#)$\footnotemark[5] & $O(\log #n#)$\footnotemark[5] & \sref{treap} \\ 
#ScapegoatTree# & $O(\log #n#)$ & $O(\log #n#)$\footnotemark[1] & \sref{scapegoattree} \\
#RedBlackTree# & $O(\log #n#)$ & $O(\log #n#)$ & \sref{redblacktree} \\ 
#BinaryTrie#\footnotemark[9] & $O(#w#)$ & $O(#w#)$ & \sref{binarytrie} \\ 
#XFastTrie#\footnotemark[9] & $O(\log #w#)$\footnotemark[1]\footnotemark[5] & $O(#w#)$\footnotemark[1]\footnotemark[5] & \sref{xfast} \\ 
#YFastTrie#\footnotemark[9] & $O(\log #w#)$\footnotemark[1]\footnotemark[5] & $O(\log #w#)$\footnotemark[1]\footnotemark[5] & \sref{yfast} \\ \hline
\multicolumn{4}{c}{} \\[2ex] \hline
\multicolumn{4}{|c|}{(Priority) #Queue# implementations} \\ \hline
 & #findMin()# & #add(x)#/#remove()# & \\ \hline
#BinaryHeap# & $O(1)$ & $O(\log #n#)$\footnotemark[1] & \sref{binaryheap} \\ \hline
#MeldableHeap# & $O(1)$ & $O(\log #n#)$\footnotemark[5] & \sref{meldableheap} \\ \hline
\end{tabular}
\footnotetext[1]{Denotes an \emph{amortized} running time. See \chapref{arrays}.}
\footnotetext[5]{Denotes an \emph{expected} running time.  See \secref{randomization}.}
\footnotetext[6]{This structure can only store #w#-bit integer data.}
\renewcommand{\thefootnote}{\arabic{footnote}}
\end{center}

\section{References}

The #List#, #USet#, and #SSet# interfaces described in
\secref{interfaces} are influenced by the Java Collections Framework
\cite{oracle_collections}.  These are essentially simplified versions of
the #List#, #Set#/#Map#, and #SortedSet#/#SortedMap# interfaces found in
the Java Collections Framework.  \javaonly{Indeed, the accompanying source
code includes wrapper classes for making #USet# and #SSet# implementations
into #Set#, #Map#, #SortedSet#, and #SortedMap# implementations.}

For a superb (and free) treatment of the mathematics discussed in this
chapter, including asymptotic notation, logarithms, factorials, Stirling's
approximation, basic probability, and lots more, see the textbook
by Leyman, Leighton, and Meyer \cite{llm11}.  For more information on
basic probability, especially as it relates to computer science, see the
textbook by Ross \cite{r01}.  Another good reference, that covers both
asymptotic notation and probability, is the textbook by Graham, Knuth,
and Patashnik \cite{gkp94}.

\javaonly{Readers wanting to brush up on their Java programming can find
many Java tutorials online \cite{oracle_tutorials}.}




