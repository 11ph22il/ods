\chapter{Binary Trees}
\chaplabel{binarytrees}

This chapter introduces one of the most fundmental structures in computer
science: binary trees.  There are lots of ways of defining binary trees.
Mathematically, a binary tree is a connected undirected graph with #n#
vertices, $#n#-1$ edges and no vertex of degree greater than 3.

For most computer science applications, binary trees are \emph{rooted}:
A special node, #root#, is called the \emph{root} of the tree.
For every node $#u#\neq #r#$, the second node on the path from #u#
to #r# is called the \emph{parent} of #u#.  Each of the (at most 2)
other nodes adjacent to #u# is called a \emph{child} of #u#. Most of
binary trees we are interested in are \emph{ordered}, so we distinguish
between the \emph{left child} and \emph{right child} of #u#.

In illustrations, binary trees are usually drawn from the root downward,
with the root at the top of the drawing and the left and right given by
left and right in the drawing.  A binary tree with X nodes is drawn this
way in \figref{binary-tree}.a.

\begin{figure}
  \begin{center}
    \begin{tabular}{cc}
      \includegraphics{figs/binary-tree-1} &
      \includegraphics{figs/binary-tree-2} \\
      (a) & (b) 
    \end{tabular}
  \end{center}
  \caption{A binary tree with (a)~13 real nodes and (b)~14 external nodes.}
  \figlabel{binary-tree}
\end{figure}

The \emph{depth} of a node #u# in a binary tree is the length of the
path from #u# to the root of the tree.   If a node #w# is on the path
from #u# to #root# then #w# is called an \emph{ancestor} of #u# and #u#
is a \emph{descendant} of #w#.  The \emph{subtree} of a node #u# is the
binary tree that is rooted at #u# and contains all of #u#'s descendants.
The \emph{height} of a node #u# is the length of the longest path from
#u# to one of its descendants.  The height of tree is height of its root.
A node #u# is a \emph{leaf} if it has no children.

We sometimes think of the tree as being augmented with \emph{external
nodes}. Any node that does not have a left child has an external node
as its left child and any node that does not have a right child has
an external node as its right child (see \figref{binary-tree}.b).
It is easy to verify, by induction, that a binary tree has $#n#\ge 1$
real nodes then it has $#n#+1$ external nodes.


\section{BinaryTree: A Basic Binary Tree}

The simplest way to represent a node in a binary tree is
to store the (at most 3) neighbours of the node explicitly:
\javaimport{ods/BinaryTreeNode.left.right.parent}
When one of these three neighbours is not present, we set it to #null#.
In this way, external nodes in the tree as well as the parent of the
root correspond to #null# values.

The binary tree itself can then be represented by a pointer to its root node:
\javaimport{ods/BinaryTree.root}

We can compute the depth of a node #u# by counting the number of steps on the path from #u# to the root:
\javaimport{ods/BinaryTree.depth(u)}


\subsection{Recursive Algorithms}

It is very easy to implement many operations using recursive algorithms on
binary trees.  For example, to compute the size of (number of nodes in)
a binary tree rooted at node #u#, we recursively compute the sizes of
the two subtrees rooted at the children of #u#, sum these sizes, and add 1:

\javaimport{ods/BinaryTree.size().size(u)}

To compute the height of a node in a #BinaryTree# we can compute the
height of its two subtrees, take the maximum, and add 1:

\javaimport{ods/BinaryTree.height(u)}

\subsection{Traversing Binary Trees}

The two algorithms from the previous section use recursion to visit all
the nodes in a binary tree.  Each of them visits the nodes of the binary tree in the same order as the following code:
\javaimport{ods/BinaryTree.traverse(u)}

Using recursion this way produces very short simple code, but can
be problematic.  The maximum depth of the recursion is given by the
maximum depth of a node in the binary tree (i.e., the tree's height).
If the height of the tree is very large, then this could very-well use
more stack space than is available, causing a crash.  

Luckily, traversing a binary tree can be done without recursion. This is
done using an algorithm that keeps track of where it came from to decide
where it will go next.  If we are at an external node, then we return
to the previous node.  If we arrive at a node #u# from #u.right# then
we are done visiting #u#'s subtree, so we return to #u#.parent.  If we
arrive at #u# from #u#.left then we visit #u#.right next.  Otherwise we
arrived at #u# from #u.parent# so this is our first visit to #u# and we
next visit #u.left#:
\javaimport{ods/BinaryTree.traverse2()}

In some implementations of binary trees, the #parent# field is not present
in a node.  When this is the case, a non-recursive implementation is
still possible, but the implementation has to use a #List# (or #Stack#)
to keep track of the path from the current node to the root.





\section{BinarySearchTree: An Unbalanced Binary Search Tree}


