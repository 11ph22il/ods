\chapter{Red-Black Trees}
\chaplabel{redblack}

In this chapter, we present red-black trees, a version of binary search
trees that have logarithmic depth.  Red-black trees are one of the most
widely-used data structures in practice.  They appear as the primary
search structure in many library implementations, including the Java
Collections Framework and several implementations of the C++ Standard
Template Library. They are also used within the Linux operating system
kernel.  There are several reasons for the popularity of red-black trees:
\begin{enumerate}
\item A red-black tree storing #n# values has height at most $2\log #n#$.
\item The #add(x)# and #remove(x)# operations on a red-black tree run
   in $O(\log #n#)$ \emph{worst-case} time.
\item The amortized number of rotations done during an #add(x)#
   or #remove(x)# operation is constant.
\end{enumerate}
The first two of these properties already put red-black trees 
ahead of skiplists, treaps, and scapegoat trees.
Skiplists and treaps rely on randomization and their $O(\log #n#)$
running times are only expected. Scapegoat trees have a guaranteed
bound on their height, but #add(x)# and #remove(x)# only run in $O(\log
#n#)$ amortized time.  The third property is just icing on the cake. It
tells us that  that the time needed to add or remove an element #x# is
dwarfed by the time it takes to find #x#.\footnote{Note that skiplists and
treaps also have this property in the expected sense. See
\excref{skiplist-changes} and \excref{treap-rotates}.}

However, the nice properties of red-black trees come with a price:
implementation complexity. Maintaining a bound of $2\log #n#$ on the
height is not easy. It requires a careful analysis of a number of cases
and it requires that the implementation does exactly the right thing
in each case.  One misplaced rotation or change of color produces a bug
that can be very difficult to understand and track down.

Rather than jumping directly into the implementation of red-black trees, we
will first provide some background on a related data structure: 2-4 trees.
This will give some insight into how red-black trees were discovered
and why efficiently maintaining red-black trees is even possible.

\section{2-4 Trees}

A 2-4 tree is a rooted tree with the following properties:
\begin{prp}[height]
  All leaves have the same depth.
\end{prp}
\begin{prp}[degree]
  Every internal node has 2, 3, or 4 children.
\end{prp}
An example of a 2-4 tree is shown in \figref{twofour-example}.
\begin{figure}
  \begin{center}
    \includegraphics{figs/24rb-2}
  \end{center}
  \caption{A 2-4 tree of height 3.}
  \figlabel{twofour-example}
\end{figure}
The properties of 2-4 trees imply that their height is logarithmic in
the number of leaves:
\begin{lem}\lemlabel{twofour-height}
  A 2-4 tree with #n# leaves has height at most $\log #n#$.
\end{lem}

\begin{proof}
The lower-bound of 2 on the degree of an internal node implies that, if the height of a 2-4 tree is $h$, then it has at least $2^h$ leaves.  
In other words,
\[
   #n# \ge 2^h \enspace .
\]
Taking logarithms on both sides of this equation gives $h \le \log #n#$.
\end{proof}

\subsection{Adding a Leaf}

Adding a leaf to a 2-4 tree is easy (see \figref{twofour-add}).  If we
want to add a leaf #u# as the child of some node #w# on the second-last
level, we simply make #u# a child of #w#.  This certainly maintains
the height property, but could violate the degree property;  if #w# had
4 children prior to adding #u#, then #w# now has 5 children.  In this
case, we split #w# into two nodes, #w# and #w#', having 2 and 3 children,
respectively.  But now #w#' has no parent, so we recursively make #w#'
a child of #w#'s parent.  Again, this may cause #w#'s parent to have too
many children in which case we split it.  This process goes on until we
reach a node that has fewer than 4 children, or until we split the root,
#r#, into two nodes #r# and #r'#.  In the latter case, we make a new
root that has #r# and #r'# as children.  This simultaneously increases
the depth of all leaves and so maintains the height property.

\begin{figure}
  \begin{center}
   \begin{tabular}{c}
     \includegraphics{figs/24tree-add-1} \\
     \includegraphics{figs/24tree-add-2} \\
     \includegraphics{figs/24tree-add-3}
   \end{tabular}
  \end{center}
  \caption{Adding a leaf to a 2-4 Tree.  This process stops after one split
  because #w.parent# has degree less than 4 before the addition.}
  \figlabel{twofour-add}
\end{figure}

Since the height of the 2-4 tree is never more than $\log #n#$, the
process of adding a leaf finishes after at most $\log #n#$ steps.

\subsection{Removing a Leaf}

Removing a leaf from a 2-4 tree is a little more tricky (see
\figref{twofour-remove}).  To remove a leaf #u# from its parent #w#, we
just remove it.  If #w# had only two children prior to the removal of #u#,
then #w# is left with only one child and violates the degree property.
\begin{figure}
  \begin{center}
   \begin{tabular}{c}
     \includegraphics{figs/24tree-remove-1} \\
     \includegraphics{figs/24tree-remove-2} \\
     \includegraphics{figs/24tree-remove-3} \\
     \includegraphics{figs/24tree-remove-4} \\
     \includegraphics{figs/24tree-remove-5} \\
   \end{tabular}
  \end{center}
  \caption{Removing a leaf from a 2-4 Tree.  This process goes all the way to the root because all of #u#'s ancestors and their siblings have degree 2.}
  \figlabel{twofour-remove}
\end{figure}

To correct this, we look at #w#'s sibling, #w'#.  The node #w'# is sure
to exist since #w#'s parent has at least 2 children.  If #w'# has 3 or
4 children, then we take one of these children from #w'# and give it to
#w#. Now #w# has 2 children and #w'# has 2 or 3 children and we are done.

On the other hand, if #w'# has only two children, then we merge #w# and
#w'# into a single node, #w#, that has 3 children.  Next we recursively
remove #w'# from the parent of #w'#.  This process ends when we reach
a node, #u#, where #u# or its sibling has more than 2 children; or we
reach the root.  In the latter case, if the root is left with only 1
child, then we delete the root and make its child the new root.  Again,
this simultaneously decreases the height of every leaf and therefore
maintains the height property.

Again, since the height of the tree is never more than $\log #n#$,
the process of removing a leaf finishes after at most $\log #n#$ steps.

\section{#RedBlackTree#: A Simulated 2-4 Tree}
\seclabel{redblacktree}

A red-black tree is a binary search tree in which each node, #u#,
has a \emph{color} which is either \emph{red} or \emph{black}.  Red is
represented by the value $0$ and black by the value $1$.
\javaimport{ods/RedBlackTree.red.black.Node<T>}
\cppimport{ods/RedBlackTree.RedBlackNode.red.black}

Before and after any operation on a red-black tree, the following two
properties are satisfied. Each property is defined both in terms of the
colors red and black, and in terms of the numeric values 0 and 1.
\begin{prp}[black-height]
  There are the same number of black nodes on every root to leaf
  path. (The sum of the colors on any root to leaf path is the same.)
\end{prp}

\begin{prp}[no-red-edge]
  No two red nodes are adjacent.  (For any node #u#, except the root,
  $#u.color# + #u.parent.color# \ge 1$.)
\end{prp}
Notice that we can always color the root, #r#, of a red-black tree black
without violating either of these two properties, so we will assume that
the root is black, and the algorithms for updating a red-black tree will
maintain this.  Another trick that simplifies red-black trees is to treat
the external nodes (represented by #nil#) as black nodes.  This way,
every real node, #u#, of a red-black tree has exactly two children,
each with a well-defined color.
An example of a red-black tree is shown in \figref{redblack-example}.

\begin{figure}
  \begin{center}
    \includegraphics{figs/24rb-1}
  \end{center}
  \caption{An example of a red-black tree with black-height 3.  External (#nil#) nodes are drawn as squares.} 
  \figlabel{redblack-example}
\end{figure}


\subsection{Red-Black Trees and 2-4 Trees}

At first it might seem surprising that a red-black tree can be efficiently
updated to maintain the black-height and no-red-edge properties, and
it seems unusual to even consider these as useful properties.  However,
red-black trees were designed to be an efficient simulation of 2-4 trees
as binary trees.

Refer to \figref{twofour-redblack}.
Consider any red-black tree, $T$, having #n# nodes and perform the
following transformation: Remove each red node #u# and connect #u#'s two
children directly to the (black) parent of #u#.  After this transformation
we are left with a tree $T'$ having only black nodes.
\begin{figure}
  \begin{center}
    \begin{tabular}{cc}
      \includegraphics{figs/24rb-3} \\
      \includegraphics{figs/24rb-2}
    \end{tabular}
  \end{center}
  \caption{Every red-black tree has a corresponding 2-4 tree.}
  \figlabel{twofour-redblack}
\end{figure}

Every internal node in $T'$ has 2, 3, or 4 children: A black node that
started out with two black children will still have two black children
after this transformation.  A black node that started out with one red
and one black child will have three children after this transformation.
A black node that started out with two red children will have 4 children
after this transformation.  Furthermore, the black-height property now
guarantees that every root-to-leaf path in $T'$ has the same length.
In other words, $T'$ is a 2-4 tree!

The 2-4 tree $T'$ has $#n#+1$ leaves that correspond to the $#n#+1$
external nodes of the red-black tree.  Therefore, this tree has height
$\log (#n#+1)$. Now, every root to leaf path in the 2-4 tree corresponds
to a path from the root of the red-black tree $T$ to an external node.
The first and last node in this path are black and at most one out of
every two internal nodes is red, so this path has at most $\log(#n#+1)$
black nodes and at most $\log(#n#+1)-1$ red nodes.  Therefore, the longest path from the root to any \emph{internal} node in $T$ is at most
\[
   2\log(#n#+1) -2 \le 2\log #n# \enspace ,
\]
for any $#n#\ge 1$.  This proves the most important property of
red-black trees:
\begin{lem}
The height of red-black tree with #n# nodes is at most $2\log #n#$.
\end{lem}

Now that we have seen the relationship between 2-4 trees and
red-black trees, it is not hard to believe that we can efficiently
maintain a red-black tree while adding and removing elements.  

We have already seen that adding an element in a #BinarySearchTree#
can be done by adding a new leaf.  Therefore, to implement #add(x)# in a
red-black tree we need a method of simulating splitting a degree 5 node
in a 2-4 tree.  A degree 5 node is represented by a black node that has
two red children one of which also has a red child. We can ``split''
this node by coloring it red and coloring its two children black.
An example of this is shown in \figref{rb-split}.

\begin{figure}
  \begin{center}
   \begin{tabular}{c}
     \includegraphics{figs/rb-split-1} \\
     \includegraphics{figs/rb-split-2} \\
     \includegraphics{figs/rb-split-3} \\
   \end{tabular}
  \end{center}
  \caption{Simulating a 2-4 tree split operation during an addition in a
    red-black tree.  (This simulates the 2-4 tree addition shown in \figref{twofour-add}.)}
  \figlabel{rb-split}
\end{figure}

Similarly, implementing #remove(x)# requires a method of merging two
nodes and borrowing a child from a sibling.  Merging two nodes is the
inverse of a split (shown in \figref{rb-split}), and involves coloring
two (black) siblings red and coloring their (red) parent black.
Borrowing from a sibling is the most complicated of the procedures and involves both rotations and recoloring of nodes.

Of course, during all of this we must still maintain the no-red-edge
property and the black-height property.  While it is no longer surprising
that this can be done, there are a large number of cases that have to
be considered if we try to do a direct simulation of a 2-4 tree by
a red-black tree.  At some point, it just becomes simpler to forget
about the underlying 2-4 tree and work directly towards maintaining the
red-black tree properties.

\subsection{Left-Leaning Red-Black Trees}

There is no single definition of a red-black tree.  Rather,
there are a family of structures that manage to maintain the
black-height and no-red-edge properties during #add(x)# and #remove(x)#
operations. Different structures go about it in different ways.  Here, we
implement a data structure that we call a #RedBlackTree#.  This structure
implements a particular variant of red-black trees that satisfies an
additional property:
\begin{prp}[left-leaning]\prplabel{left-leaning}\prplabel{redblack-last}
  At any node #u#, if #u.left# is black, then #u.right# is black.
\end{prp}
Note that the red-black tree shown in \figref{redblack-example}  does
not satisfy the left-leaning property.  It is violated by the parent of
red node in the rightmost path.

The reason for maintaining the left-leaning property is that it reduces
the number of cases encountered when updating the tree during #add(x)#
and #remove(x)# operations.  In terms of 2-4 trees, it implies that
every 2-4 tree has a unique representation:  A node of degree 2 becomes
a black node with 2 black children.  A node of degree 3 becomes a black
node whose left child is red and whose right child is black.  A node of
degree 4 becomes a black node with two red children.

Before we describe the implementation of #add(x)# and #remove(x)# in
detail, we first present some simple subroutines used by these methods
that are illustrated in \figref{redblack-flippullpush}.  The first two
subroutines are for manipulating colors while preserving the black-height
property. The #pushBlack(u)# method takes as input a black node #u#
that has two red children and colors #u# red and its two children black.
The #pullBlack(u)# method reverse this operation:
\codeimport{ods/RedBlackTree.pushBlack(u).pullBlack(u)}

\begin{figure}
  \begin{center}
    \includegraphics{figs/flippullpush}
  \end{center}
  \caption{Flips, pulls and pushes}
  \figlabel{redblack-flippullpush}
\end{figure}

The #flipLeft(u)# method swaps the colors of #u# and #u.right# and
then performs a left rotation at #u#.  This reverses the colors
of these two nodes as well as their parent-child relationship:
\codeimport{ods/RedBlackTree.flipLeft(u)} 
The #flipLeft(u)# operation is especially useful in restoring the
left-leaning property at a node $u$ that violates it (because #u.left#
is black and #u.right# is red).  In this special case, we can be assured
this operation preserves both the black-height and no-red-edge properties.
The #flipRight(u)# operation is symmetric to #flipLeft(u)# with the
roles of left and right reversed.
\codeimport{ods/RedBlackTree.flipRight(u)}

\subsection{Addition}

To implement #add(x)# in a #RedBlackTree#, we perform a standard
#BinarySearchTree# insertion, which adds a new leaf, #u#, with $#u.x#=#x#$
and set $#u.color#=#red#$.  Note that this does not change the black
height of any node, so it does not violate the black-height property.
It may, however violate the left-leaning property (if #u# is the
right child of its parent) and it may violate the no-red-edge property
(if #u#'s parent is #red#).  To restore these properties, we call the
method #addFixup(u)#.
\codeimport{ods/RedBlackTree.add(x)}

The #addFixup(u)# method, illustrated in \figref{rb-addfix}, takes
as input a node #u# whose color is red and which may be violating the
no-red-edge property and/or the left-leaning property.  The following
discussion is probably impossible to follow without referring to
\figref{rb-addfix} or recreating it on a piece of paper while reading
this discussion.  Indeed, the reader may wish to study this figure
before continuing.

\begin{figure}
  \begin{center}
    \includegraphics[scale=0.8]{figs/rb-addfix}
  \end{center}
  \caption{A single round in the process of fixing Property~2 after
  an insertion.}
  \figlabel{rb-addfix}
\end{figure}

If #u# is the root of the tree, then we can color #u# black and this
restores both properties.  If #u#'s sibling is also red, then #u#'s
parent must be black, so both the left-leaning and no-red-edge properties
already hold.

Otherwise, we first determine if #u#'s parent, #w# violates the
left-leaning property and, if so, perform a #flipLeft(w)# operation and
set $#u#=#w#$.  This leaves us in a well-defined state:  #u# is the left
child of its parent, #w#, so #w# now satisfies the left-leaning property.
All that remains is to ensure the no-red-edge property at #u#.  
We only have to worry about the case where #w# is red, since otherwise
#u# already satisfies the no-red-edge property.

Since we are not done yet, #u# is red and #w# is red.  The no-red-edge
property (which is only violated by #u# and not by #w#) implies that
#u#'s grandparent #g# exists and is black.  If #g#'s right child is red,
then the left-leaning property ensures that both #g#'s children are red,
and a call to #pushBlack(g)# makes #g# red and #w# black.  This restores
the no-red-edge property at #u#, but may cause it to be violated at #g#,
so the whole process starts over with $#u#=#g#$.

If #g#'s right child is black, then a call to #flipRight(g)# makes
#w# the (black) parent of #g# and gives #w# two red children #u# and
#g#. This ensures that #u# satisfies the no-red-edge property and #g#
satisfies the left-leaning property.  In this case we can stop.
\codeimport{ods/RedBlackTree.addFixup(u)}

The #insertFixup(u)# method takes constant time per iteration and each
iteration either finishes or moves #u# closer to the root.  This implies
that the #insertFixup(u)# method finishes after $O(\log #n#)$ iterations
in $O(\log #n#)$ time.

\subsection{Removal}

The #remove(x)# operation in a #RedBlackTree# tree is the most complicated
operation to implement, and this is true of all known implementations.
Like #remove(x)# in a \texttt{BinarySearchTree} the operation boils down to
finding a node #w# with only one child, #u#, and splicing #w# out of
the tree by having #w.parent# adopt #u#.

The problem with this is that, if #w# is black, then the black-height
property will now be violated at #w.parent#.  We get around this
problem, temporarily, by adding #w.color# to #u.color#.  Of course, this
introduces two other problems:  (1)~#u# and #w# both started out black,
then $#u.color#+#w.color#=2$ (double black), which is in invalid color.
If #w# was red, then it is replaced by a black node #u#, which may
violate the left-leaning property at $#u.parent#$.  Both of these problems
are resolved with a call to the #removeFixup(u)# method.
\codeimport{ods/RedBlackTree.remove(x)}

The #removeFixup(u)# method takes as input a node #u# whose color is black
(1) or double-black (2).  If #u# is double-black, then #removeFixup(u)#
performs a series of rotations and recoloring operations that move the
double-black node up the tree until it can be gotten rid of.  During this
process, the node #u# changes until, at the end of this process, #u#
refers to the root of the subtree that has been changed.  The root of
this subtree may have changed color.  In particular, it may have gone
from red to black, so the #removeFixup(u)# method finishes by checking
if #u#'s parent violates the left-leaning property and, if so, fixes it.
\codeimport{ods/RedBlackTree.removeFixup(u)}

The #removeFixup(u)# method is illustrated in \figref{rb-removefix}.
Again, the following text will be very difficult, if not impossible,
to follow without referring constantly to \figref{rb-removefix}.
Each iteration of the loop in #removeFixup(u)# processes the double-black node #u# based on one of four cases.

\begin{figure}
  \begin{center}
    \includegraphics[scale=0.8]{figs/rb-removefix}
  \end{center}
  \caption{A single round in the process of eliminating a double-black node
   after a removal.}
  \figlabel{rb-removefix}
\end{figure}

\noindent
Case 0: #u# is the root.  This is the easiest case to treat.  We recolor #u# to be black and this does not violate any of the red-black tree properties.

\noindent 
Case 1: #u#'s sibling, #v#, is red.  In this case, #u#'s sibling is the
left child of its parent, #w# (by the left-leaning property).  We perform
a right-flip at #w# and then proceed to the next iteration.  Note that
this causes #w#'s parent to violate the left-leaning property and it
causes the depth of #u# to increase.  However, it also implies that the
next iteration will be in Case~3 with #w# colored red.  When examining
Case~3, below, we will see that this means the process will stop during
the next iteration.
\codeimport{ods/RedBlackTree.removeFixupCase1(u)}

\noindent
Case 2: #u#'s sibling, #v#, is black and #u# is the left child of its
parent, #w#.  In this case, we call #pullBlack(w)#, making #u# black,
#v# red, and darkening the color of #w# to black or double-black.
At this point, #w# does not satisfy the left-leaning property, so we
call #flipLeft(w)# to fix this.

At this point, #w# is red and #v# is the root of the subtree we started
with.  We need to check if #w# causes no-red-edge property to be violated.
We do this by inspecting #w#'s right child, #q#.  If #q# is black,
then #w# satisfies the no-red-edge property and we can continue to the
next iteration with #u#=#v#.

Otherwise (#q# is red), both the no-red-edge property and the left-leaning
property are violated at #q# and #w#, respectively.  A call to
#rotateLeft(w)# restores the left-leaning property, but the no-red-edge
property is still violated.  At this point, #q# is the left child of
#v# and #w# is the left child of #q#, #q# and #w# are both red and #v#
is black or double-black.  A #flipRight(v)#  makes #q# the parent of
both #v# and #w#.  Following this up by a #pushBlack(q)# makes both #v#
and #w# black and sets the color of #q# back to the original color of #w#.

At this point, there is no more double-black node and the no-red-edge and
black-height properties are reestablished.  The only possible problem
that remains is that the right child of #v# may be red, in which case
the left-leaning property is violated.  We check this and perform a
#flipLeft(v)# to correct it if necessary.
\codeimport{ods/RedBlackTree.removeFixupCase2(u)}

\noindent
Case 3: #u#'s sibling is black and #u# is the right child of its parent,
#w#.  This case is symmetric to Case~2 and is handled mostly the same way.
The only differences come from the fact that the left-leaning property
is asymmetric, so it requires different handling.

As before, we begin with a call to #pullBlack(w)#, which makes #v# red
and #u# black.  A call to #flipRight(w)# promotes #v# to the root of
the subtree.  At this point #w# is red, and the code branches two ways
depending on the color of #w#'s left child, #q#.

If #q# is red, then the code finishes up exactly the same way that
Case~2 finishes up, but is even simpler since there is no danger of #v#
not satisfying the left-leaning property.

The more complicated case occurs when #q# is black.  In this case,
we examine the color if #v#'s left child.  If it is red, then #v# has
two red children and its extra black can be pushed down with a call to
#pushBlack(v)#.  At this point, #v# now has #w#'s original color and we
are done.

If #v#'s left child is black then #v# violates the left-leaning property
and we restore this with a call to #flipLeft(v)#.  The next iteration
of #removeFixup(u)# then continues with $#u#=#v#$.
\codeimport{ods/RedBlackTree.removeFixupCase3(u)}.

Each iteration of #removeFixup(u)# takes constant time.  Cases~2 and 3
either finish or move #u# closer to the root of the tree.  Case~0 (where
#u# is the root) always terminates and Case~1 leads immediately to Case~3,
which also terminates.  Since the height of the tree is at most $2\log
#n#$, we conclude that there are at most $O(\log #n#)$ iterations of
#removeFixup(u)# so #removeFixup(u)# runs in $O(\log #n#)$ time.


\section{Summary}

The following theorem summarizes the performance of the #RedBlackTree# data structure:

\begin{thm}
  A #RedBlackTree# implements the #SSet# interface. A #RedBlackTree#
  supports the operations #add(x)#, #remove(x)#, and #find(x)# in $O(\log
  #n#)$ worst-case time per operation.
\end{thm}

Not included in the above theorem is the extra bonus

\begin{thm}\thmlabel{redblack-amortized}
  Beginning with an empty #RedBlackTree#, any sequence of $m$
  #add(x)# and #remove(x)# operations results in a total of $O(m)$
  time spent during all calls #addFixup(u)# and #removeFixup(u)#. 
\end{thm}

We will only sketch a proof of \thmref{redblack-amortized}. By comparing
#addFixup(u)# and #removeFixup(u)# with the algorithms for adding
or removing a leaf in a 2-4 tree, we can convince ourselves that this
property is something that is inherited from a 2-4 tree.  In particular,
if we can show that the total time spent splitting, merging, and borrowing
in a 2-4 tree is $O(m)$, then this implies \thmref{redblack-amortized}.

The proof of this for 2-4 trees uses the potential method of amortized
analysis.\footnote{See the proofs of \lemref{dualarraydeque-amortized}
and \lemref{selist-amortized} for other applications of the potential
method.} Define the potential of an internal node #u# in a 2-4 tree as
\[
  \Phi(#u#) = 
    \begin{cases} 
      c & \text{if #u# has 2 children} \\ 
      0 & \text{if #u# has 3 children} \\ 
      3c & \text{if #u# has 3 children}  
    \end{cases}
\]
and the potential of a 2-4 tree as the sum of the potentials of its nodes.
When a split occurs, it is because a node of degree 4 becomes two nodes,
one of degree 2 and one of degree 3.  This means that the overall
potential drops by $3c-c-0 = c$.  Thus, the time spent performing a
split is accounted for by the drop in potential. When a merge occurs,
two nodes that used to have degree 2 are replaced by one node of degree
3. The result is a drop in potential of $2c-0=2c$.  This drop in potential
accounts for the cost of the merge.  A borrow operation happens only
once per deletion and, except for merges and splits, the potential can
only increase by a constant amount for each addition or removal of a leaf.

The above analysis implies that, beginning with an empty 2-4 tree,
any sequence of $m$ additions and removals of leaves results in at most
$O(m)$ time spent splitting nodes, merging nodes, and borrowing children
from nodes.  \thmref{redblack-amortized} is a consequence of this
analysis and the correspondence between 2-4 trees and red-black trees.

\section{Discussion and Exercises}

Red-black trees were first introduced by Guibas and Sedgewick \cite{gs78}.
Despite their high implementation complexity they are found in some of
the most commonly-used libraries and applications.  Most algorithms and
data structures discuss some variant of red-black trees.

Andersson \cite{a93} describes a left-leaning version of balanced trees
that are similar to red-black trees but have the additional constraint
that any node has at most one red child.  This implies that these trees
simulate 2-3 trees rather than 2-4 trees.  They are significantly simpler,
though, than the #RedBlackTree# structure presented in this chapter.

Sedgewick \cite{s08} describes at least two versions of left-leaning
red-black trees.  These use recursion along with a simulation of
top-down splitting and merging in 2-4 trees. The combination of these
two techniques makes for particularly short and elegant code.

A related, and older, data structure is the AVL tree \cite{avl62}.
AVL trees are \emph{height-balanced}: At each node $u$, the height
of the subtree rooted at #u.left# and the subtree rooted at #u.right#
differ by at most one.  It follows immediately that, if $F(h)$ is the
minimum number of leaves in a tree of height $h$, then $F(h)$ obeys the
Fibonacci recurrence
\[
   F(h) = F(h-1) + F(h-2)
\]
with base cases $F(0)=1$ and $F(1)=1$.  This means $F(h)$ is approximately
$\varphi^h/\sqrt{5}$, where $\varphi=(1+\sqrt{5})/2\approx1.61803399$ is the
\emph{golden ratio}.  (More precisely, $|\varphi^h/\sqrt{5} - F(h)|\le 1/2$.)
Arguing as in the proof of \lemref{twofour-height}, this implies
\[
   h \le \log_\varphi #n# \approx 1.440420088\log #n# \enspace ,
\]
so AVL trees have smaller height than red-black trees.
The height-balanced property can be maintained during #add(x)# and
#remove(x)# operations by walking back up the path to the root and
performing a rebalancing operation at each node #u# where the height of
#u#'s left and right subtrees differ by 2.  See \figref{avl-rebalance}.

\begin{figure}
  \begin{center}
    \includegraphics{figs/avl-rebalance}
  \end{center}
  \caption{Rebalancing in an AVL tree.  At most 2 rotations are required
  to convert a node whose subtrees have height $h$ and $h+2$ into a node
  whose subtrees each have height at most $h+1$.}
  \figlabel{avl-rebalance}
\end{figure}

Andersson's variant of red-black trees, Sedgewick's variant of red-black
trees, and AVL trees are all simpler to implement than the #RedBlackTree#
structure defined here.  Unfortunately, none of them can guarantee that
the amortized time spent rebalancing is $O(1)$ per update.  In particular,
there is no analogue of \thmref{redblack-amortized} for those structures.

\begin{exc}
  Why does the method #remove(x)# in the #RedBlackTree# implementation
  perform the assignment #u.parent=w.parent#?  Shouldn't this already
  be done by the call to #splice(w)#?
\end{exc}

\begin{exc}
  Suppose a 2-4 tree, $T$, has $#n#_\ell$ leaves and $#n#_i$ internal nodes.
  \begin{enumerate}
    \item What is the minimum value of $#n#_i$, as a function of $#n#_\ell$?
    \item What is the maximum value of $#n#_i$, as a function of $#n#_\ell$?
    \item If $T'$ is a red-black tree that represents $T$, then how many red
     nodes does $T'$ have?
  \end{enumerate}
\end{exc}

\begin{exc}
  Prove that, during an #add(x)# operation, an AVL tree must perform
  at most one rebalancing operation (that involves at most 2 rotations;
  see \figref{avl-rebalance}).  Give an example of an AVL tree and a
  #remove(x)# operation on that tree that requires on the order of $\log
  #n#$ rebalancing operations.
\end{exc}

\begin{exc}
  Implement an #AVLTree# class that implements AVL trees as described
  above.  Compare its performance to that of the #RedBlackTree#
  implementation.   Which implementation has a faster #find(x)# operation?
\end{exc}

\begin{exc}
  Design and implement a series of experiments that compare the relative
  performance of #find(x)#, #add(x)#, and #remove(x)# for #SkiplistSSet#,
  #ScapegoatTree#, #Treap#, and #RedBlackTree#.  Be sure to include
  multiple test scenarios, including cases where the data is random,
  already sorted, is removed in random order, is removed in sorted order,
  and so on.
\end{exc}
