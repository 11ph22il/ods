\chapter{Scapegoat Trees}
\chaplabel{scapegoat}

In this chapter we study a binary search tree data structure, the
#ScapegoatTree#, that keeps itself balanced by \emph{partial rebuilding
operations}.  During one of these operation, an entire subtree is
deconstructed and rebuilt into a perfectly balanced subtree.


There are many ways of rebuilding a subtree the subtree rooted at
node #u# into a perfectly balanced tree.  One of the simplest is
to traverse #u#'s subtree, gatering all its nodes into an array #a#
and then to recursively build a balanced subtree using #a#.  If we let
$#m#=#a.length#/2$, then the element #a[m]# becomes the root of the new
subtree, $#a#[0],\ldots,#a#[#m#-1]$ get stored recursively in the left
subtree and $#a#[#m#+1],\ldots,#a#[#a.length#-1]$ get stored recursively
in the right subtree. 
\javaimport{ods/ScapegoatTree.rebuild(u).packIntoArray(u,a,i).buildBalanced(a,i,ns)}
A call to #rebuild(u)# takes $O(#size(u)#)$ time.  The subtree built by
#rebuild(u)# has minimum height; there is no tree of smaller height that
has #size(u)# nodes.


\section{#ScapegoatTree#: A Binary Search Tree with Partial Rebuilding}

A #ScapegoatTree# is a #BinarySearchTree# that, in addition to keeping
track of the number, #n#, of  nodes in the tree also keep a counter #q#
that maintains an upper-bound on the number of nodes. 
\javaimport{ods/ScapegoatTree.q}
At all times, #n# and #q# obey the following inequalities:
\[
      #q#/2 \le  #n# \le #q#  \enspace .
\]
In addition, a #ScapegoatTree# has logarithmic height; at all times, the height of the scapegoat tree does not exceed:
\begin{equation}
     \log_{3/2} #q# \le \log_{3/2} 2#n# < \log_{3/2} #n# + 2\enspace .
     \eqlabel{scapegoat-height}
\end{equation}
Even with this constraint, a #ScapegoatTree# can look surprisingly unbalanced.  The tree in \figref{scapegoat-example} has $#q#=#n#=10$ and height $5<\log_{3/2}10 \approx 5.679$.

\begin{figure}
  \begin{center}
    \includegraphics{figs/scapegoat-insert-1}
  \end{center}
  \caption{A #ScapegoatTree# with 10 nodes and height 5.}
  \figlabel{scapegoat-example}
\end{figure}

Implementing the #find(x)# operation in a #ScapegoatTree# is done
using the standard algorithm for searching in a #BinarySearchTree#
(see \secref{binarysearchtree}).  This takes time proportional to the
height of the tree which, by \eqref{scapegoat-height} is $O(\log #n#)$.

To implement the #add(x)# operation, we first increment #n# and #q#
and then use the usual algorithm for adding #x# to a binary search
tree; we search for #x# and then add a new leaf #u# with $#u.x#=#x#$.
At this point, we may get lucky and the depth of #u# might not exceed
$\log_{3/2}#q#$ If so, then we leave well enough alone and don't do
anything else.

Unfortunately, it will sometimes happen that $#depth(u)# > \log_{3/2}
#q#$.  In this case we need to do something to reduce the height.
This isn't a big job; there is only one node, namely #u#, whose depth
exceeds $\log_{3/2} #q#$.  To fix #u#, we walk from #u# back up to the
root looking for a \emph{scapegoat}, #w#.  The scapegoat, #w#, is a very
unbalanced node.  It has the property that
\begin{equation}
   \frac{#size(w.child)#}{#size(w)#} > \frac{2}{3} \enspace ,
   \eqlabel{scapegoat}
\end{equation}
where #w.child# is the child of #w# on the path from the root to #u#.
We'll very shortly prove that a scapegoat exists.  For now, we can
take it for granted.  Once we've found the scapegoat #w#, we completely
destroy the subtree rooted at #w# and rebuild it into a perfectly balanced
binary search tree.  We know, from \eqref{scapegoat}, that, even before
the addition of #u#, #w#'s subtree was not a complete binary tree.
Therefore, when we rebuild #w#, the height decreases by at least 1 so that height of the #ScapegoatTree# is once again at most $\log_{3/2}#q#$.

\javaimport{ods/ScapegoatTree.add(x)}

\begin{figure}
  \begin{center}
    \begin{tabular}{cc}
      \includegraphics{figs/scapegoat-insert-3} &
      \includegraphics{figs/scapegoat-insert-4} 
    \end{tabular}
  \end{center}
  \caption{Inserting 3.5 into a #ScapegoatTree# increases its depth to 6, which violates \eqref{scapegoat-height} since $6 > \log_{3/2} 11 \approx 5.914$.  A scapegoat is found at the node containing 5.}
\end{figure}
If we ignore the cost of finding the scapegoat #w# and rebuilding the
subtree rooted at #w#, then the running-time of #add(x)# is dominated
by the initial search, which takes $O(\log #q#) = O(\log #n#)$ time.
We will account for the cost of finding the scapegoat and rebuilding
using amortized analysis in the next section.

The implementation of #remove(x)# in a #ScapegoatTree# is very simple.
We search for #x# and remove it using the usual algorithm for removing a
node from a #BinarySearchTree#.  (Note that this can never increase the
height of the tree.)  Next, we decrement #n# but leave #q# unchanged.
Finally, we check if $#q# > 2#n#$ and, if so, we \emph{rebuild the entire
tree} into a perfectly balanced binary search tree and set $#q#=#n#$
\javaimport{ods/ScapegoatTree.remove(x)}
Again, if we ignore the cost of rebuilding, the running-time of the
#remove(x)# operation is proportional to the height of the tree is
therefore $O(\log #n#)$.

\subsection{Analysis of Correctness and Running-Time}

In this section we analyze the correctness and amortized running-time
of operations on a #ScapegoatTree#.  We first prove the correctness by
showing that, when the #add(x)# operation results in a node that violates
Condition \eqref{scapegoat-height}, then we can always find a scapegoat:

\begin{lem}
  Let #u# be a node of depth $h>\log_{3/2} #q#$ in a #ScapegoatTree#.
  Then there exists a node $#w#$ on the path from #u# to the root
  such that
  \[
     \frac{#size(w)#}{#size(parent(w))#} > 2/3 \enspace .
  \]
\end{lem}

\begin{proof}
  Suppose, for the sake of contradiction, that this is not the case, and
  \[
     \frac{#size(w)#}{#size(parent(w))#} \le 2/3 \enspace .
  \]
  for all nodes #w# on the path from #u# to the root.  Denote the path
  from the root to #u# as $#r#=#u#_0,\ldots,#u#_h=#u#$.  Then, we have
  $#size(u#_0#)#=#n#$,
  $#size(u#_1#)#\le\frac{2}{3}#n#$, 
  $#size(u#_2#)#\le\frac{4}{9}#n#$ and, more generally,
  \[
  #size(u#_i#)#\le\left(\frac{2}{3}\right)^i#n# \enspace .
  \]
  But this gives a contradiction, since $#size(u)#\ge 1$, hence
  \[
    1 \le #size(u)# \le \left(\frac{2}{3}\right)^h#n#
   < \left(\frac{2}{3}\right)^{\log_{3/2} #q#}#n#
   \le \left(\frac{2}{3}\right)^{\log_{3/2} #n#}#n#
   = \left(\frac{1}{#n#}\right) #n#
   = 1 \enspace . \qedhere
  \]
\end{proof}

Next, we analyze the parts of the running time that we have not yet
accounted for.  There are two parts:  The cost of calls to #size(u)#
when search for scapegoat nodes, and the cost of calls to #rebuild(w)#
when we find a scapegoat #w#.
The cost of calls to #size(u)# can be related to the cost of calls to #rebuild(w)#, as follows:
\begin{lem}
During a call to #add(x)# in a #ScapegoatTree#, the cost of finding the scapegoat #w# and rebuilding the subtree rooted at #w# is $O(#size(w)#)$
\end{lem}

\begin{proof}
The cost of rebuilding the scapegoat node #w#, once we find it, is
$O(#size(w)#)$.  When searching for the scapegoat node, we call #size(u)#
on a sequence of nodes $#u#_0,\ldots,#u#_k$ until we find the scapegoat
$#u#_k=#w#$.  However, since $#u#_k$ is the first node in this sequence
that is a scapegoat, we know that
\[
  #size(u#_{i}#)# < \frac{2}{3}#size(u#_{i+1}#)#
\]
for all $i\in\{0,\ldots,k-2\}$.  Therefore, the cost of all calls to #size(u)# is
\begin{eqnarray*}
 O\left( \sum_{i=0}^k #size(u#_{k-i}#)# \right)
 &=& O\left(
  #size(u#_k#)# 
  + \sum_{i=0}^{k-1} #size(u#_{k-i-1}#)#
  \right) \\
 &=& O\left(
  #size(u#_k#)# 
  + \sum_{i=0}^{k-1} \left(\frac{2}{3}\right)^i#size(u#_{k}#)#
  \right) \\
&=& O\left(
  #size(u#_k#)#\left(1+ 
   \sum_{i=0}^{k-1} \left(\frac{2}{3}\right)^i
  \right)\right) \\
&=& O(#size(u#_k#)#) = O(#size(w)#) \enspace ,
\end{eqnarray*}
where the last line follows from the fact that the sum is a geometrically decreasing series.
\end{proof}

\begin{lem}
  Starting with an empty #ScapegoatTree# any sequence of $m$ #add(x)#
  and #remove(x)# operations causes at most $O(m\log m)$ time to be used
  by #rebuild(u)# operations.
\end{lem}

\begin{proof}
  To prove this, we will use a credit scheme.  Each node stores a number
  of credits.  Each credit can pay for some constant, $c$, units of time
  spent rebuilding.  The scheme gives out a total of $O(m\log m)$ credits
  and every call to #rebuild(u)# is paid for with credits stored at #u#.

  During an insertion or deletion, we give one credit to each node on the
  path to the inserted node, or deleted node, #u#.  In this way we hand
  out at most $\log_{3/2}#q#\le \log_{3/2}m\le$ credits per operation.
  During a deletion we also store an additional 1 credit ``on the side.''
  Thus, in total we give out at most $O(m\log m)$ credits.  All that
  remains is to show that these credits are sufficient to pay for all
  calls to #rebuild(u)#.

  If we call #rebuild(u)# during an insertion, it is because #u# is
  a scapegoat.  Suppose, without loss of generality, that
  \[
    \frac{#size(u.left)#}{#size(u)#} > \frac{2}{3} \enspace .
  \]
  Using the fact that
  \[
    #size(u)# = 1 + #size(u.left)# + #size(u.right)# 
  \]
  we deduce that
  \[
    \frac{1}{2}#size(u.left)# > #size(u.right)#  \enspace 
  \]
  and therefore
  \[
    #size(u.left)# - #size(u.right)# > \frac{1}{2}#size(u.left)# >
    \frac{1}{3}#size(u)#  \enspace .
  \]
  Now, the last time a subtree containing #u# was rebuilt (or when #u#
  was inserted, if a subtree containing #u# was never rebuilt), we had
  \[
    #size(u.left)# - #size(u.right)# \le 1 \enspace .
  \]
  Therefore, the number of operations #add(x)# or #remove(x)# operations
  that have affected #u.left# or #u.right# since then is at least
  \[
    \frac{1}{3}#size(u)# - 1 \enspace . 
  \]
  and there are therefore at least this many credits stored at #u#
  that are available to pay for the $O(#size(u)#)$ time it takes to
  call #rebuild(u)#.

  If we call #rebuild(u)# during a deletion, it is because $#q# > 2#n#$.
  In this case, we have $#q#-#n#> #n#$ credits stored ``on the side'' and
  we use these to pay for the $O(#n#)$ time it takes to rebuild the root.
  This completes the proof.
\end{proof}

\subsection{Summary and Conclusions}
The following theorem summarizes the performance of the #ScapegoatTree# data structure:

\begin{thm}\thmlabel{skiplist}
A #ScapegoatTree# implements the #SSet# interface. Ignoring the cost of #rebuild(u)# operations, a #ScapegoatTree# supports
the operations #add(x)#, #remove(x)#, and #find(x)# in $O(\log #n#)$
time per operation.

Furthermore, beginning with an empty #ScapegoatTree#, any sequence of $m$ #add(x)# and #remove(x)# operations results in a total of $O(m\log m)$ time spent during all calls to #rebuild(u)#.

\end{thm}

The term \emph{scapegoat tree} is due to Rivest and Galperin \cite{rg93},
who define and analyze these trees.  However, the same structure
was discovered earlier by Andersson \cite{a89,a90}, who called them
\emph{general balanced trees} since they can have any shape as long as
their height is small.

Experimenting with the #ScapegoatTree# implementation will reveal that
it is often considerably slower than the other #SSet# implementations
in this book. This may be somewhat surprising, since height bound of
\[
   \log_{3/2}#q# \approx 1.709\log #n# + O(1)
\] 
is better than the expected length of a search path in a #Skiplist# and
not too far from that of a #Treap#.  The implementation could be optimized
by storing the sizes of subtrees explicitly at each node (or at least
reusing already computed subtree sizes).  Even with these optimizations,
there will always be sequences of #add(x)# and #delete(x)# operation for
which a #ScapegoatTree# takes longer than other #SSet# implementations.

This gap in performance is due to the fact that, unlike the other #SSet#
implementations discussed in this book, a #ScapegoatTree# can spend a lot
of time restructuring itself.  \excref{scapegoat-nlogn} asks you to prove
that there are sequences of #n# operations in which a #ScapegoatTree#
will spend on the order of $#n#\log #n#$ time in calls to #rebuild(u)#.
This is in contrast to other #SSet# implementations discussed in this
book that only make $O(#n#)$ structural changes during a sequence of
#n# operations.  This is, unfortunately, a necessary consequence of
the fact that a #ScapegoatTree# does all its restructuring by calls to
#rebuild(u)# \cite{X}.

There are applications in which a #ScapegoatTree# could be the right
choice.  This would occur any time there is additional data associated
with nodes that can not be updated in constant time when a rotation is
performed, but that can be updated during a #rebuild(u)# operation.
In such cases, the #ScapegoatTree# and related structures based on
partial rebuilding may work.

\begin{exc}\exclabel{scapegoat-nlogn}
  Show that, if we start with an empty #ScapegoatTree# and call #add(x)#
  for $#x#=1,2,3,\ldots,#n#$, then the total time spent during calls to
  #rebuild(u)# is at least $c#n#\log #n#$ for some constant $c>0$.
\end{exc}

\begin{exc}
  Analyze and implement a #WeightBalancedTree#.  This is a tree in
  which each node #u#, except the root, maintains the \emph{balance
  invariant} that $#size(u)# \le (2/3)#size(u.parent)#$.  The #add(x)# and
  #remove(x)# operations are identical to the standard #BinarySearchTree#
  operations, except that any time the balance invariant is violated at
  a node #u#, the subtree rooted at #u.parent# is rebuilt.

  Your analysis should show that operations on a #WeightBalancedTree#
  run in $O(\log#n#)$ amortized time.  
\end{exc}

\begin{exc}
  Analyze and implement a #CountdownTree#.  In a #CountdownTree# each
  node #u# keeps a \emph{timer} #u.t#.  The #add(x)# and #remove(x)#
  operations are exactly the same as in a standard #BinarySearchTree#
  except that, whenever one of these operations affects #u#'s subtree,
  #u.t# is decremented.  When $#u.t#=0$ the entire subtree rooted at #u#
  is rebuilt into a perfectly balanced binary search tree.  When a node
  #u# is involved in a rebuilding operation (either because #u# is rebuilt
  or one of #u#'s ancestors is rebuilt) #u.t# is reset to $#size(u)#/3$.

  Your analysis should show that operations on a countdown tree run
  in $O(\log #n#)$ amortized time.  (Hint: First show that each node #u#
  satisifies some version of a balance invariant.)
\end{exc}

