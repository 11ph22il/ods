\chapter{Random Binary Search Trees}
\chaplabel{rbs}

In this chapter, we present a binary search tree structure that uses
randomization to achieve $O(\log #n#)$ expected time for all operations.

\section{Random Binary Search Trees}
\seclabel{rbst}

Consider the two binary search trees shown in \figref{rbs-lvc}.  The one
on the left is a list and the other is a perfectly balanced binary search
tree. The one on the left has height $#n#-1=14$ and the one on the right
has height three.

\begin{figure}
  \begin{center}
    \begin{tabular}{cc}
      \includegraphics{figs/bst-path} &
      \includegraphics{figs/bst-balanced}
    \end{tabular}
  \end{center}
  \caption{Two binary search trees containing the integers $0,\ldots,14$.}
  \figlabel{rbs-lvc}
\end{figure}

Imagine how these two trees could have been constructed.  The one on
the left occurs if we start with an empty #BinarySearchTree# and add
the sequence
\[
    \langle 0,1,2,3,4,5,6,7,8,9,10,11,12,13,14 \rangle \enspace .
\]
No other sequence of additions will create this tree (as you can prove
by induction on #n#). On the other hand, the tree on the right can be
created by the sequence
\[
    \langle 7,3,11,1,5,9,13,0,2,4,6,8,10,12,14 \rangle  \enspace .
\]
Other sequences work as well, including
\[
    \langle 7,3,1,5,0,2,4,6,11,9,13,8,10,12,14 \rangle  \enspace ,
\]
and
\[
    \langle 7,3,1,11,5,0,2,4,6,9,13,8,10,12,14 \rangle \enspace .
\]
In fact, there are $21,964,800$ addition sequences that generate the
tree on the right and only one that generates the tree on the left.

The above example gives some anecdotal evidence that, if we choose a
random permutation of $0,\ldots,14$, and add it into a binary search
tree then we are more likely to get a very balanced tree (the right
side of \figref{rbs-lvc}) than we are to get a very unbalanced tree
(the left side of \figref{rbs-lvc}).

We can formalize this notion by studying random binary search trees.
A \emph{random binary search tree} of size #n# is obtained in the
following way:  Take a random permutation $#x#_0,\ldots,#x#_{#n#-1}$
of $0,\ldots,#n#-1$ and add its elements, one by one, into a
#BinarySearchTree#.

Note that the values $0,\ldots,#n#-1$ could be replaced by any ordered
set of #n# elements without changing any of the properties of the
random binary search tree.  The element $#x#\in\{0,\ldots,#n#-1\}$ is
simply standing in for the element of rank #x# in an ordered set of
size #n#.

Before we can present our main result about random binary search trees,
we must take some time for a short digression to discuss a type of number
that comes up frequently when studying randomized structures. For a
non-negative integer, $k$, the $k$-th \emph{harmonic number}, denoted
$H_k$, is defined as
\[
  H_k = 1 + 1/2 + 1/3 + \cdots + 1/k \enspace .
\] 
The harmonic number $H_k$ has no simple closed form, but it is very
closely related to the natural logarithm of $k$.  In particular,
\[
  \ln k \le H_k \le \ln k + 1  \enspace .
\]
\newcommand{\hint}{\int_1^k\! (1/x)\, \mathrm{d}x}%
Readers who have studied calculus might notice that this is because
the integral $\hint = \ln k$.  Keeping in mind that an integral can be
interpreted as the area between a curve and the $x$-axis, the value of
$H_k$ can be lower-bounded by the integral $\hint$ and upper-bounded by
$1+ \hint$.  (See \figref{harmonic-integral} for a graphical explanation.)

\begin{figure}
  \begin{center}
    \begin{tabular}{cc}
      \includegraphics{figs/harmonic-2} & \includegraphics{figs/harmonic-3}
    \end{tabular}
  \end{center}
  \caption{The $k$th harmonic number $H_k=\sum_{i=1}^k 1/i$ is upper-bounded by $1+\hint$ and lower-bounded by $\hint$.}
  \figlabel{harmonic-integral}
\end{figure}


\begin{lem}\lemlabel{rbs}
  In a random binary search tree of size #n#, the following statements hold:
  \begin{enumerate}
    \item For any $#x#\in\{0,\ldots,#n#-1\}$, the expected length of
    the search path for #x# is $H_{#x#+1} + H_{#n#-#x#} - O(1)$.\footnote{The expressions $#x#+1$ and $#n#-#x#$ can be interpreted respectively as the number of elements in the tree less than or equal to #x# and the number of elements in the tree greater than or equal to #x#.}
    \item For any $#x#\in(-1,n)\setminus\{0,\ldots,#n#-1\}$, the expected length of the
    search path for #x# is $H_{\lceil#x#\rceil} + H_{#n#-\lceil#x#\rceil}$.
  \end{enumerate}
\end{lem}

We will prove \lemref{rbs} in the next section.  For now, consider what
the two parts of \lemref{rbs} tell us.  The first part tells us that if
we search for an element in a tree of size #n#, then the expected length
of the search path is at most $2\ln n + O(1)$.  The second part tells
us the same thing about searching for a value not stored in the tree.
When we compare the two parts of the lemma, we see that it is only
slightly faster to search for something that is in a tree compared to
something that is not in a tree.


\subsection{Proof of \lemref{rbs}}

The key observation needed to prove \lemref{rbs} is the following: The
search path for a value #x# in the open interval $(-1,#n#)$ in a random binary search tree, $T$, contains
the node with key $i < #x#$ if and only if, in the random permutation
used to create $T$, $i$ appears before any of $\{i+1,i+2,\ldots,\lfloor#x#\rfloor\}$.

To see this, refer to \figref{rbst-records} and notice that, until
some value in $\{i,i+1,\ldots,\lfloor#x#\rfloor\}$ is added, the search
paths for each value in the open interval $(i-1,\lfloor#x#\rfloor+1)$
are identical.  (Remember that for two search values to have
different search paths, there must be some element in the tree that
compares differently with them.)  Let $j$ be the first element in
$\{i,i+1,\ldots,\lfloor#x#\rfloor\}$ to appear in the random permutation.
Notice that $j$ is now and will always be on the search path for #x#.
If $j\neq i$ then the node $#u#_j$ containing $j$ is created before the
node $#u#_i$ that contains $i$.  Later, when $i$ is added, it will be
added to the subtree rooted at $#u#_j#.left#$, since $i<j$.  On the other
hand, the search path for #x# will never visit this subtree because it
will proceed to $#u#_j#.right#$ after visiting $#u#_j$.

\begin{figure}
  \begin{center}
    \includegraphics{figs/rbst-records}
  \end{center}
  \caption{The value $i<#x#$ is on the search path for #x# if and only
   if $i$ is the first element among $\{i,i+1,\ldots,\lfloor#x#\rfloor\}$ added to the tree.}
  \figlabel{rbst-records}
\end{figure}

Similarly, for $i>#x#$, $i$ appears in the search path for #x#
if and only if $i$ appears before any of $\{\lceil#x#\rceil,
\lceil#x#\rceil+1,\ldots,i-1\}$ in the random permutation used to
create $T$.

Notice that, if we start with a random permutation of $\{0,\ldots,#n#\}$,
then the subsequences containing only $\{i,i+1,\ldots,\lfloor#x#\rfloor\}$
and $\{\lceil#x#\rceil, \lceil#x#\rceil+1,\ldots,i-1\}$ are also random
permutations of their respective elements.  Each element, then, in the
subsets $\{i,i+1,\ldots,\lfloor#x#\rfloor\}$ and $\{\lceil#x#\rceil,
\lceil#x#\rceil+1,\ldots,i-1\}$ is equally likely to appear before
any other in its subset in the random permutation used to create $T$.
So we have
\[
  \Pr\{\mbox{$i$ is on the search path for #x#}\}
  = \left\{ \begin{array}{ll}
     1/(\lfloor#x#\rfloor-i+1) & \mbox{if $i < #x#$} \\
     1/(i-\lceil#x#\rceil+1) & \mbox{if $i > #x#$} 
     \end{array}\right . \enspace .
\]

With this observation, the proof of \lemref{rbs}
involves some simple calculations with harmonic numbers:

\begin{proof}[Proof of \lemref{rbs}]
Let $I_i$ be the indicator random variable that is equal to one when $i$
appears on the search path for #x# and zero otherwise.  Then the length
of the search path is given by
\[
  \sum_{i\in\{0,\ldots,#n#-1\}\setminus\{#x#\}} I_i
\]
so, if $#x#\in\{0,\ldots,#n#-1\}$, the expected length of the search
path is given by (see \figref{rbst-probs}.a)
\begin{align*}
  \E\left[\sum_{i=0}^{#x#-1} I_i + \sum_{i=#x#+1}^{#n#-1} I_i\right]
   & =  \sum_{i=0}^{#x#-1} \E\left[I_i\right]
         + \sum_{i=#x#+1}^{#n#-1} \E\left[I_i\right] \\
   & = \sum_{i=0}^{#x#-1} 1/(\lfloor#x#\rfloor-i+1)
         + \sum_{i=#x#+1}^{#n#-1} 1/(i-\lceil#x#\rceil+1) \\
   & = \sum_{i=0}^{#x#-1} 1/(#x#-i+1)
         + \sum_{i=#x#+1}^{#n#-1} 1/(i-#x#+1) \\
   & = \frac{1}{2}+\frac{1}{3}+\cdots+\frac{1}{#x#+1} \\
   & {} + \frac{1}{2}+\frac{1}{3}+\cdots+\frac{1}{#n#-#x#} \\
   & = H_{#x#+1} + H_{#n#-#x#} - 2  \enspace .
\end{align*}
The corresponding calculations for a search value
$#x#\in(-1,n)\setminus\{0,\ldots,#n#-1\}$ are almost identical (see
\figref{rbst-probs}.b).
\end{proof}

\begin{figure}
  \begin{center}
    \begin{tabular}{c}
      \includegraphics{figs/rbst-probs-a} \\ (a) \\[2ex]
      \includegraphics{figs/rbst-probs-b} \\ (b) \\[2ex]
    \end{tabular}
  \end{center}
  \caption{The probabilities of an element being on the search path for #x#
   when (a)~#x# is an integer and (b)~when #x# is not an integer.}
  \figlabel{rbst-probs}
\end{figure}

\subsection{Summary}

The following theorem summarizes the performance of a random binary
search tree:

\begin{thm}\thmlabel{rbs}
A random binary search tree can be constructed in $O(#n#\log #n#)$ time.
In a random binary search tree, the #find(x)# operation takes $O(\log
#n#)$ expected time.
\end{thm}

\section{#Treap#: A Randomized Binary Search Tree}
\seclabel{treap}

The problem with random binary search trees is, of course, that they are
not dynamic.  They don't support the #add(x)# or #remove(x)# operations
needed to implement the #SSet# interface.  In this section we describe
a data structure called a #Treap# that uses \lemref{rbs} to implement
the #SSet# interface.

A node in a #Treap# is like a node in a #BinarySearchTree# in that it has
a data value, #x#, but it also contains a unique numerical \emph{priority},
#p#, that is assigned at random:
\javaimport{ods/Treap.Node<T>}
\cppimport{ods/Treap.TreapNode}
In addition to being a binary search tree, the nodes in a #Treap#
also obey the \emph{heap property}:  At every node #u#, except the root,
$#u.parent.p# < #u.p#$.  That is, each node has a priority smaller than
that of its two children.  An example is shown in \figref{treap}.

\begin{figure}
  \begin{center}
    \includegraphics{figs/treap}
  \end{center}
  \caption{An example of a #Treap# containing the integers
  $0,\ldots,9$. Each node, #u#, is illustrated with $#u.x#,#u.p#$.}
  \figlabel{treap}
\end{figure}

The heap and binary search tree conditions together ensure that, once
the key (#x#) and priority (#p#) for each node are defined, the
shape of the #Treap# is completely determined. The heap property tells us that
the node with minimum priority has to be the root, #r#, of the #Treap#.
The binary search tree property tells us that all nodes with keys smaller
than #r.x# are stored in the subtree rooted at #r.left# and all nodes
with keys larger than #r.x# are stored in the subtree rooted at #r.right#.

The important point about the priority values in a #Treap# is that they
are unique and assigned at random.  Because of this, there are
two equivalent ways we can think about a #Treap#.  As defined above, a
#Treap# obeys the heap and binary search tree properties.  Alternatively,
we can think of a #Treap# as a #BinarySearchTree# whose nodes
were added in increasing order of priority.  For example, the #Treap#
in \figref{treap} can be obtained by adding the sequence of $(#x#,#p#)$
values 
\[
  \langle
   (3,1), (1,6), (0,9), (5,11), (4,14), (9,17), (7,22), (6,42), (8,49), (2,99)
  \rangle
\]
into a #BinarySearchTree#.

Since the priorities are chosen randomly, this is equivalent to taking a
random permutation of the keys --- in this case the permutation is
\[
  \langle 3, 1, 0, 5, 9, 4, 7, 6, 8, 2 \rangle
\]
--- and adding these to a #BinarySearchTree#.  But this means that the
shape of a treap is identical to that of a random binary search tree.
In particular, if we replace each key #x# by its rank,\footnote{The
rank of an element #x# in a set $S$ of elements is the number of
elements in $S$ that are less than #x#.} then \lemref{rbs} applies.
Restating \lemref{rbs} in terms of #Treap#s, we have:
\begin{lem}\lemlabel{rbs-treap}
  In a #Treap# that stores a set $S$ of #n# keys, the following statements hold:
  \begin{enumerate}
    \item For any $#x#\in S$, the expected length of
    the search path for #x# is $H_{r(#x#)+1} + H_{#n#-r(#x#)} - O(1)$.
    \item For any $#x#\not\in S$, the expected length of the
    search path for #x# is $H_{r(#x#)} + H_{#n#-r(#x#)}$.
  \end{enumerate}
  Here, $r(#x#)$ denotes the rank of #x# in the set $S\cup\{#x#\}$.
\end{lem}

\lemref{rbs-treap} tells us that #Treap#s can implement the #find(x)#
operation efficiently. However, the real benefit of a #Treap# is that
it can support the #add(x)# and #delete(x)# operations.  To
do this, it needs to perform rotations in order to maintain the heap property.  Refer to \figref{rotations}.
A \emph{rotation} in a binary
search tree is a local modification that takes a parent #u# of a node #w#
and makes #w# the parent of #u#, while preserving the binary search tree
property. Rotations come in two flavours: \emph{left} or \emph{right}
depending on whether #w# is a right or left child of #u#, respectively.

\begin{figure}
  \begin{center}
     \includegraphics{figs/rotation}
  \end{center}
  \caption{Left and right rotations in a binary search tree.}
  \figlabel{rotations}
\end{figure}

The code that implements this has to handle these two possibilities
and be careful of a boundary
case (when #u# is the root) so the actual code is a little longer than
\figref{rotations} would lead a reader to believe:
\codeimport{ods/BinarySearchTree.rotateLeft(u).rotateRight(u)}
In terms of the #Treap# data structure, the most important property of a
rotation is that the depth of #w# decreases by one while the depth of #u#
increases by one.

Using rotations, we can implement the #add(x)# operation as follows:
We create a new node, #u#, and assign #u.x=x# and pick a random value
for #u.p#.  Next we add #u# using the usual #add(x)# algorithm
for a #BinarySearchTree#, so that #u# is now a leaf of the #Treap#.
At this point, our #Treap# satisfies the binary search tree property,
but not necessarily the heap property.  In particular, it may be the
case that #u.parent.p > u.p#.  If this is the case, then we perform a
rotation at node #w#=#u.parent# so that #u# becomes the parent of #w#.
If #u# continues to violate the heap property, we will have to repeat this, decreasing #u#'s depth by one every time, until
#u# either becomes the root or $#u.parent.p# < #u.p#$.
\codeimport{ods/Treap.add(x).bubbleUp(u)}
An example of an #add(x)# operation is shown in \figref{treap-add}.

\begin{figure}
  \begin{center}
  \includegraphics{figs/treap-insert-a} \\
  \includegraphics{figs/treap-insert-b} \\
  \includegraphics{figs/treap-insert-c} \\
  \end{center}
  \caption{Adding the value 1.5 into the #Treap# from \figref{treap}.}
  \figlabel{treap-add}
\end{figure}

The running time of the #add(x)# operation is given by the time it
takes to follow the search path for #x# plus the number of rotations
performed to move the newly-added node, #u#, up to its correct location
in the #Treap#.  By \lemref{rbs-treap}, the expected length of the
search path is at most $2\ln #n#+O(1)$.  Furthermore, each rotation
decreases the depth of #u#.   This stops if #u# becomes the root, so
the expected number of rotations cannot exceed the expected length of
the search path.  Therefore, the expected running time of the #add(x)#
operation in a #Treap# is $O(\log #n#)$.  (\excref{treap-rotates}
asks you to show that the expected number of rotations performed during
an insertion is actually only $O(1)$.)

The #remove(x)# operation in a #Treap# is the opposite of the #add(x)#
operation.  We search for the node, #u#, containing #x# and then perform
rotations to move #u# downwards until it becomes a leaf and then we
splice #u# from the #Treap#.  Notice that, to move #u# downwards, we can
perform either a left or right rotation at #u#, which will replace #u#
with #u.right# or #u.left#, respectively.
The choice is made by the first of the following that apply:
\begin{enumerate}
\item If #u.left# and #u.right# are both #null#, then #u# is a leaf and no rotation is performed.
\item If #u.left# (or #u.right#) is #null#, then perform a right (or left, respectively) rotation at #u#.
\item If $#u.left.p# < #u.right.p#$ (or $#u.left.p# > #u.right.p#)$, then perform a right rotation (or left rotation, respectively) at #u#.
\end{enumerate}
These three rules ensure that the #Treap# doesn't become disconnected and that the heap property is maintained once #u# is removed.
\codeimport{ods/Treap.remove(x).trickleDown(u)}
An example of the #remove(x)# operation is shown in \figref{treap-remove}.
\begin{figure}
  \begin{center}
  \includegraphics{figs/treap-delete-a} \\
  \includegraphics{figs/treap-delete-b} \\
  \includegraphics{figs/treap-delete-c} \\
  \includegraphics{figs/treap-delete-d} 
  \end{center}
  \caption{Removing the value 9 from the #Treap# in \figref{treap}.}
  \figlabel{treap-remove}
\end{figure}

The trick to analyze the running time of the #remove(x)# operation is
to notice that this operation is the reverse of the #add(x)# operation.
In particular, if we were to reinsert #x#, using the same priority #u.p#,
then the #add(x)# operation would do exactly the same number of rotations
and would restore the #Treap# to exactly the same state it was in before
the #remove(x)# operation took place.  (Reading from bottom-to-top,
\figref{treap-remove} illustrates the insertion of the value 9 into a
#Treap#.) This means that the expected running time of the #remove(x)#
on a #Treap# of size #n# is proportional to the expected running time
of the #add(x)# operation on a #Treap# of size $#n#-1$.  We conclude
that the expected running time of #remove(x)# is $O(\log #n#)$.

\subsection{Summary}

The following theorem summarizes the performance of the #Treap# data
structure:

\begin{thm}
A #Treap# implements the #SSet# interface. A #Treap# supports
the operations #add(x)#, #remove(x)#, and #find(x)# in $O(\log #n#)$
expected time per operation.
\end{thm}

It is worth comparing the #Treap# data structure to the #SkiplistSSet#
data structure.  Both implement the #SSet# operations in $O(\log #n#)$
expected time per operation.  In both data structures, #add(x)# and #remove(x)#
involve a search and then a constant number of pointer changes
(see \excref{treap-rotates} below).  Thus, for both these
structures, the expected length of the search path is the critical value
in assessing their performance.  In a #SkiplistSSet#, the expected length
of a search path is
\[
     2\log #n# + O(1) \enspace ,
\]
In a #Treap#, the expected length of a search path is 
\[
    2\ln #n# +O(1) \approx 1.386\log #n#  + O(1) \enspace .
\]
Thus, the search paths in a #Treap# are considerably shorter and this
translates into noticeably faster operations on #Treap#s than #Skiplist#s.
\excref{skiplist-opt} in \chapref{skiplists} shows how the
expected length of the search path in a #Skiplist# can be reduced to
\[
     e\ln #n# + O(1) \approx 1.884\log #n# + O(1) 
\]
by using biased coin tosses.  Even with this optimization, the expected
length of search paths in a #SkiplistSSet# is noticeably longer than in
a #Treap#.

\section{Discussion and Exercises}

Random binary search trees have been studied extensively.  Devroye
\cite{d88} gives a proof of \lemref{rbs} and related results. There are
much stronger results in the literature as well.  The most impressive
of which is due to Reed \cite{r03}, who shows that the expected height
of a random binary search tree is
\[
  \alpha\ln n - \beta\ln\ln n + O(1)
\]
where $\alpha\approx4.31107$ is the unique solution on $[2,\infty)$ of the
equation $\alpha\ln((2e/\alpha))=1$ and $\beta=\frac{3}{2\ln(\alpha/2)}$ .
Furthermore, the variance of the height is constant.

The name #Treap# was coined by Aragon and Seidel \cite{as96} who discussed
#Treap#s and some of their variants.  However, their basic structure was
studied much earlier by Vuillemin \cite{v80} who called them Cartesian
trees.

One space-optimization of the #Treap# data structure that is sometimes
performed is the elimination of the explicit storage of the priority #p#
in each node. Instead, the priority of a node, #u#, is computed by
hashing #u#'s address in memory\javaonly{ (in 32-bit Java, this is equivalent
to hashing #u.hashCode()#)}.  Although a number of hash functions will
probably work well for this in practice, for the important parts of the
proof of \lemref{rbs} to remain valid, the hash function should be randomized
and have the \emph{min-wise independent property}:  For any distinct
values $x_1,\ldots,x_k$, each of the hash values $h(x_1),\ldots,h(x_k)$
should be distinct with high probability and, for each $i\in\{1,\ldots,k\}$,
\[
   \Pr\{h(x_i) = \min\{h(x_1),\ldots,h(x_k)\}\} \le c/k
\]
for some constant $c$.
One such class of hash functions that is easy to implement and fairly
fast is \emph{tabulation hashing} \cite{pt10}.

\begin{exc}
  Prove the assertion that there are $21,964,800$ sequences that generate
  the tree on the right hand side of \figref{rbs-lvc}.  (Hint: Give a
  recursive formula for the number of sequences that generate a complete
  binary tree of height $h$ and evaluate this formula for $h=3$.)
\end{exc}

\begin{exc}
  Design and implement the #permute(a)# method that takes as input an
  array, #a#, containing #n# distinct values and randomly permutes #a#.
  The method should run in $O(#n#)$ time and you should prove that each
  of the $#n#!$ possible permutations of #a# is equally probable. 
\end{exc}

\begin{exc}\exclabel{treap-rotates}
  Use both parts of \lemref{rbs-treap} to prove that the expected number
  of rotations performed by an #add(x)# operation (and hence also a
  #remove(x)# operation) is $O(1)$.
\end{exc}

\begin{exc}
  Design and implement a version of a #Treap# that includes a #get(i)#
  operation that returns the key with rank #i# in the #Treap#.  (Hint:
  Have each node, #u#, keep track of the size of the subtree rooted
  at #u#.)
\end{exc}

\begin{exc}\exclabel{treap-split}
  Design and implement a version of a #Treap# that supports the #split(x)#
  operation.  This operation removes all values from the #Treap# that
  are greater than #x# and returns a second #Treap# that contains all
  the removed values.

  For example, the code #t2 = t.split(x)# removes from #t# all values
  greater than #x# and returns a new #Treap# #t2# containing all
  these values.  The #split(x)# operation should run in $O(\log #n#)$
  expected time.
\end{exc}

