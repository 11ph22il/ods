\chapter{Random Binary Search Trees}
\chaplabel{rbs}

In this chapter, we present a binary search tree structure that uses randomization to achieve $O(\log #n#)$ expected time for all operations.

\section{Random Binary Search Trees}

Consider the two binary search trees shown in
\figref{rbs-lvc}.  The one on the left is a list and the
other is a perfectly balanced binary search tree. The one on the left
has height $#n#-1=14$ and the one on the right has height $3$.

\begin{figure}
  \begin{center}
    \begin{tabular}{cc}
      \includegraphics{figs/bst-path} &
      \includegraphics{figs/bst-balanced}
    \end{tabular}
  \end{center}
  \caption{Two binary search trees containing the integers $0,\ldots,14$.}
  \figlabel{rbs-lvc}
\end{figure}

Imagine how these two trees could have been constructed.  The one on
the left occurs if we start with an empty #BinarySearchTree# and add
the sequence
\[
    \langle 0,1,2,3,4,5,6,7,8,9,10,11,12,13,14 \rangle \enspace .
\]
No other sequence of additions will create this tree (as you can prove
by induction on #n#). On the other hand, the tree on the right can be
created by the sequence
\[
    \langle 7,3,11,1,5,9,13,0,2,4,6,8,10,12,14 \rangle  \enspace .
\]
Other sequences work as well, including
\[
    \langle 7,3,1,5,0,2,4,6,11,9,13,8,10,12,14 \rangle 
\]
and
\[
    \langle 7,3,1,11,5,0,2,4,6,9,13,8,10,12,14 \rangle \enspace .
\]
In fact, there are $21,964,800$ addition sequences that generate the
tree on the right and only 1 that generates the tree on the left.

The above example gives some anecdotal evidence that, if we choose a
random permutation of $0,\ldots,14$, and add it into a binary search
tree then we are more likely to get a very balanced tree (the right
hand of \figref{rbs-lvc}) than we are to get a very unbalanced tree
(the left hand of \figref{rbs-lvc}).

We can formalize this notion by studying random binary search trees.
A \emph{random binary search tree} of size #n# is obtained the
following way:  Take a random permutation $#x#_0,\ldots,#x#_{#n#-1}$
of $0,\ldots,#n#-1$ and add the elements, one by one, into a
#BinarySearchTree#.  

Before we can present our main result about random binary search trees,
we take time for short digression to discuss a type of number that comes
up frequently when studying randomized structures. For a non-negative
integer, $k$, the $k$-th \emph{harmonic number}, denoted $H_k$, is
defined as
\[
  H_k = 1 + 1/2 + 1/3 + \cdots + 1/k \enspace .
\] 
The harmonic number $H_k$ has no simple closed form, but it is very
closely related to the natural logarithm of $k$.  In particular,
\[
  \ln k \le H_k \le \ln k + 1  \enspace .
\]
\newcommand{\hint}{\int_1^k\! (1/x)\, \mathrm{d}x}
Readers who have studied calculus might notice that this is because the
integral $\hint = \ln k$.  Keeping in mind that an integral can
be interpreted as the area between a curve and the $x$-axis, the value
of $H_k$ can be lower-bounded by the integral $\hint$ and
upper-bounded by $1+ \hint$.  (See \figref{harmonic-integral}
for an explanation.)

\begin{figure}
  \begin{center}
    \begin{tabular}{cc}
      \includegraphics{figs/harmonic-2} & \includegraphics{figs/harmonic-3}
    \end{tabular}
  \end{center}
  \caption{The $k$th harmonic number $H_k=\sum_{i=1}^k 1/i$ is upper-bounded by $1+\hint$ and lower-bounded by $\hint$.}
  \figlabel{harmonic-integral}
\end{figure}


\begin{lem}\lemlabel{rbs}
  In a random binary search tree of size #n#, the following statements hold:
  \begin{enumerate}
    \item For any $#x#\in\{0,\ldots,#n#-1\}$, the expected length of
    the search path for #x# is $H_{#x#+1} + H_{#n#-#x#} - O(1)$.
    \item For any $#x#\in\{-1,\ldots,#n#-1\}$, the expected length of the
    search path for $#x#+1/2$ is $H_{#x#+1} + H_{#n#-#x#}$.
  \end{enumerate}
\end{lem}

We will prove \thmref{rbs} in the next section.  For now, we can consider
what the two parts of \thmref{rbs} tell us.  The first part tells us
that if we search for an element #x# that is in the tree, then expected
length of the search path is at most $2\ln n + O(1)$.  The second part
tells us the same thing for elements #x# that are not stored in the tree.
When we compare the two parts, we see that it is only slightly faster
to search for something that is in the tree compared to something that
is not in the tree.


\subsection{Proof of \lemref{rbs}}

The key observation needed to prove \lemref{rbs} is the following: The
search path for $#x#+1/2$ in a random binary search tree $T$ contains
the node with key $i \le #x#$ if and only if, in the random permutation
used to create $T$, $i$ appears before any of $\{i+1,i+2,\ldots,#x#\}$.

Refer to \figref{rbst-records}.
To see this, notice that, until some value in $\{i,i+1,\ldots,#x#\}$
is added, the search paths for each of $\{i,i+1,\ldots,#x#,#x#+1/2\}$
are identical.
Let $j$ be the first element in $\{i,i+1,\ldots,#x#\}$ to appear in the
random permutation.  Notice that $j$ is now and will always be on the
search path for $#x#+1/2$.  If $j\neq i$ then the node $#u#_j$ containing
$j$ is created before the node $#u#_i$ that contains $i$.  Later, when
$i$ is added, it will be added to the subtree rooted at $#u#_j#.left#$,
since $i<j$.  On the other hand, the search path for $#x#+1/2$ will
never visit this subtree because it will proceed to $#u#_j#.right#$
after visiting $#u#_j$.

\begin{figure}
  \begin{center}
    \includegraphics{figs/rbst-records}
  \end{center}
  \caption{The value $i\le x$ is on the search path to $#x#+1/2$ if and only
   if $i$ is the first element among $\{i,i+1,\ldots,#x#\}$ added to the tree.}
  \figlabel{rbst-records}
\end{figure}

Similarly, for $i>#x#$, $i$ appears in the search path for $#x#+1/2$ if
and only if $i$ appears before any of $\{#x#+1,#x#+2,\ldots,i-1\}$ in the
random permutation used to create $T$.  Since, in a random permutation,
each element is equally likely to occur at any location, we have
\[
  \Pr\{\mbox{$i$ is on the search path for $#x#+1/2$}\}
  = \left\{ \begin{array}{ll}
     1/(#x#-i+1) & \mbox{if $i \le #x#$} \\
     1/(i-#x#) & \mbox{if $i > #x#$} 
     \end{array}\right.
\]

With this observation, the proof of the second part of \lemref{rbs}
involves some simple calculations with harmonic numbers:

\begin{proof}[Proof of \lemref{rbs} Part 2]
Let $I_i$ be the indicator random variable that is equal to 1 when $i$ appears on the search path for $#x#+1/2$ and 0 otherwise.  Then the length of the search path is given by
\[
  \sum_{i=0}^{#n#-1} I_i
\]
so the expected length of the search path is given by
\begin{align*}
  \E\left[\sum_{i=0}^{#n#-1} I_i\right] 
   & =  \sum_{i=0}^{#n#-1} \E\left[I_i\right] \\
   & =  \sum_{i=0}^{#x#} \E\left[I_i\right]
         + \sum_{i=#x#+1}^{#n#-1} \E\left[I_i\right] \\
   & =  \sum_{i=0}^{#x#} 1/(#x#-i+1)
         + \sum_{i=#x#+1}^{#n#-1} 1/(i-#x#) \\
   & =  1+\frac{1}{2}+\frac{1}{3}+\cdots+\frac{1}{#x#+1}  \\
   &     {} + 1 + \frac{1}{2}+\frac{1}{3}+\cdot+\frac{1}{#n#-#x#}  \\
   & =  H_{#x#+1} + H_{#n#-#x#}  \enspace . \qedhere
\end{align*}
\end{proof}

Proving the first part of \lemref{rbs} is a slightly more complicated
exercise in manipulating harmonic series:  
\begin{proof}[Proof of \lemref{rbs} Part 1]
We will study the number of nodes at which the search path for #x# goes
from a node #u# to its left child #u.left# and show that the expected
number of such nodes is
\[
   H_{#n#-#x#} - O(1) \enspace .
\]
All these nodes contain values in the set $X=\{#x#+1,\ldots,#n#-1\}$.
The permutation that created $T$ induces a permutation
$P=p_0,\ldots,p_{#n#-#x#}$ on the set $X\cup\{#x#\}$.  Suppose #x# occurs
at position $i$ in $P$, so that $p_i=#x#$.  Then, the same reasoning used
in the proof of the first part of the lemma shows that the expected number
of elements of $P$ on the search path for #x# is  $H_{i}$.  Now, #x#
is equally likely to occur in the $#n#-#x#+1$ possible position in $P$,
so the expected number of elements in $P$ on the search path for #x#
is given by
\newcommand{\nx}{\mathtt{n}-\mathtt{x}}
\begin{eqnarray*}
 \sum_{i=0}^{#n#-#x#}\Pr\{p_i=#x#\}\cdot H_{i} 
 &= &\sum_{i=0}^{#n#-#x#}\frac{1}{#n#-#x#+1}\cdot H_{i}  \\
  &=& \frac{1}{#n#-#x#+1}\left(\begin{array}{l@{\,}l}
         1 + {} \\
   1 + \frac{1}{2} + {} \\
   1 + \frac{1}{2} + \frac{1}{3} + \\
   \vdots & \ddots \\
   1 + \frac{1}{2} + \frac{1}{3} + & \cdots + \frac{1}{#n#-#x#} \\
    \end{array}\right) \\
  &=& \frac{1}{#n#-#x#+1}\left(\begin{array}{l@{\,}l}
   \color{red}{1 + \frac{1}{2} + \frac{1}{3} +} & \color{red}{\cdots + \frac{1}{\nx}} \\
   1 + \color{red}{\frac{1}{2} + \frac{1}{3} +} & \color{red}{\cdots +\frac{1}{\nx}} \\
   1 + \frac{1}{2} + \color{red}{\frac{1}{3} +} & \color{red}{\cdots +\frac{1}{\nx}} \\
   1 + \frac{1}{2} + \frac{1}{3} \color{red}{{} +} &\color{red}{\cdots +\frac{1}{\nx}} \\
   \vdots & \ddots \quad\,\,\,\,  \color{red}{\vdots} \\
   1 + \frac{1}{2} + \frac{1}{3} + &\cdots + \frac{1}{\nx} \\
    \end{array}\right) - \color{red}{(\nx)}\\
   &=& \frac{1}{#n#-#x#+1}\left((#n#-#x#+1)H_{#n#-#x#} - (#n#-#x#)\right) \\
   &=& H_{#n#-#x#} - \frac{#n#-#x#}{#n#-#x#+1} \\
   &=& H_{#n#-#x#} - O(1)
\end{eqnarray*}

A symmetric argument to the one above proves that the expected number of
nodes at which the search path proceeds from a node to its right child is
$H_{#x#+1} - O(1)$.
Therefore, the expected length of the search
path for #x# is $H_{#x#+1} + H_{#n#-#x#}-O(1)$, as required.
\end{proof}

\subsection{Summary}

The following theorem summarizes the performance of a random binary
search tree:

\begin{thm}\thmlabel{rbs}
A random binary search tree can be constructed in $O(#n#\log #n#)$ time.
In a random binary search tree, the #find(x)# operation takes $O(\log
#n#)$ expected time.
\end{thm}

\section{#Treap#: A Randomized Binary Search Tree}

The problem with random binary search trees is, of course, that they are
not dynamic.  They don't support the #add(x)# or #remove(x)# operations
needed to implement the #SSet# interface.  In this section we described
a data structure, called a #Treap#, that uses \lemref{rbs} to implement
the #SSet# interface.

A node in a #Treap# is like a node in a #BinarySearchTree# in that it has
a data value #x#, but it also contains a unique numerical \emph{priority}
#p# that is assigned at random:
\javaimport{ods/Treap.Node<T>}
In addition to being a binary search tree, the nodes in a #Treap#
also obey the \emph{heap property}:  At every node #u# except the root,
$#u.parent.p# < #u.p#$.  That is, each node has a priority smaller than
that of its two children.  An example is shown in \figref{treap}

\begin{figure}
  \begin{center}
    \includegraphics{figs/treap}
  \end{center}
  \caption{An example of a #Treap# containing the integer
  $0,\ldots,9$. Each nodes #u# is illustrated with $#u.x#,#u.p#$}
  \figlabel{treap}
\end{figure}

The heap and binary search tree conditions together ensure that, once
the key (#x#) and priority (#p#) for each node are defined, the
shape of the #Treap# is completely determined. The heap property tells us that
the node with minimum priority has to be the root, #r#, of the #Treap#.
The binary search tree property tells us that all nodes with keys smaller
than #r.x# are stored in the subtree rooted at #r.left# and all nodes
with keys larger than #r.x# are stored in the subtree rooted at #r.right#.

The important point about the priority values in a #Treap# is that they
are unique and they are assigned at random.  Because of this, there are
two equivalent ways we can think about a #Treap#.  As defined above, a
treap obeys the heap and binary search tree properties.  Alternatively,
we can think of first sorting the nodes, by increasing order of priority
and then adding these into a #BinarySearchTree#.  For example, the #Treap#
in \figref{treap} can be obtained by adding the sequence of $(#x#,#p#)$
values 
\[
  \langle
   (3,1), (1,6), (0,9), (5,11), (9,12), (4,14), (7,22), (6,42), (8,49), (2,99)
  \rangle
\]
into a #BinarySearchTree#.

Since the priorities are chosen randomly, this is equivalent to take a
random permutation of the keys --- in this case the permutation is
\[
  \langle 3, 1, 0, 5, 9, 4, 7, 6, 8, 2 \rangle
\]
--- and adding these to a #BinarySearchTree#.  But this means that the
shape of a treap is identical to that of a random binary search tree.
In particular, if we replace each key #x# by it's rank,\footnote{The
rank of an element #x# in a set $S$ of elements is the number of
elements in $S$ that are less than #x#.} then \lemref{rbs} applies.
Restating \lemref{rbs} in terms of #Treap#s, we have:
\begin{lem}\lemlabel{rbs-treap}
  In a #Treap# that stores a set $S$ of #n# keys, the following statements hold:
  \begin{enumerate}
    \item For any $#x#\in S$, the expected length of
    the search path for #x# is $H_{r(#x#)+1} + H_{#n#-r(#x#)} - O(1)$.
    \item For any $#x#\not\in S$, the expected length of the
    search path for #x# is $H_{r(#x#)} + H_{#n#-r(#x#)}$.
  \end{enumerate}
  Here, $r(#x#)$ denotes the rank of #x# in the set $S\cup\{#x#\}$.
\end{lem}

\lemref{rbs-treap} tells us that #Treap#s can implement the #find(x)#
operation efficiently.  However, the real benefit of a #Treap# is that
it can support the #add(x)# and #delete(x)# operations.  However, to
do this, it needs to perform rotations.  Refer to \figref{rotations}.
A \emph{rotation} in a binary
search tree is a local modification that takes a parent #u# of a node #w#
and makes #w# the parent of #u#, while preserving the binary search tree
property. Rotations come in two flavours: \emph{left} or \emph{right}
depending on whether #w# is a right or left child of #u#, respectively.

\begin{figure}
  \begin{center}
     \includegraphics{figs/rotation}
  \end{center}
  \caption{Left and right rotations in a binary search tree.}
  \figlabel{rotations}
\end{figure}

The code that implements this has to handle two cases (whether #u#
is a left or right child of its parent) and be careful of a boundary
case (when #u# is the root) so the actual code is a little longer than
\figref{rotations} would lead a reader to believe:
\javaimport{ods/BinarySearchTree.rotateLeft(u).rotateRight(u)}
In terms of the #Treap# data structure, the most important property of a
rotation is that the depth of #w# decreases by 1 while the depth of #u#
increases by 1.

Using rotations, we can implement the #add(x)# operation as follows:
We create a new node #u# and assign #u.x=x# and pick a random value
for #u.p#.  Next we add #u# using the usual #add(x)# algorithm
for a #BinarySearchTree#, so that #u# is now a leaf of the #Treap#.
At this point, our #Treap# satisfies the binary search tree property,
but not necessarily the heap property.  In particular, it may be the
case that #u.parent.p > u.p#.  If this is the case, then we perform a
rotation at node #w#=#u.parent# so that #u# becomes the parent of #w#.
We may have to repeat this several times until #u#'s depth decreases
enough so that #u# becomes the root or $#u.parent.p# < #u.p#$.
\javaimport{ods/Treap.add(x).bubbleUp(u)}
An example of the operation of an #add(x)# operation is shown in \figref{treap-add}.

\begin{figure}
  \begin{center}
  \includegraphics{figs/treap-insert-a} \\
  \includegraphics{figs/treap-insert-b} \\
  \includegraphics{figs/treap-insert-c} \\
  \end{center}
  \caption{Adding the value 1.5 into the treap from \figref{treap}.}
  \figlabel{treap-add}
\end{figure}

The running-time of the #add(x)# operation is given by the time it
takes to follow the search path for #x# plus the number of rotations
performed to move the newly-added node #u# up to its correct location
in the #Treap#.  By \lemref{rbs-treap}, the expected length of the
search path is at most $2\ln #n#+O(1)$.  Furthermore, each rotation
decreases the depth of #u#.   This stops if #u# becomes the root, so
the expected number of rotations can not exceed the expected length of
the search path.  Therefore, the expected running-time of the #add(x)#
operation in a #Treap# is $O(\log #n#)$.  (\excref{treap-rotates}
asks you to show that the expected number of rotations performed during
an insertion is actually only $O(1)$.)

The #remove(x)# operation in a #Treap# is the opposite of the #add(x)#
operation.  We search for the node #u# containing #x# and then perform
rotations to move #u# downwards until it becomes a leaf and then we
splice #u# from the #Treap#.  Notice that, to move #u# downwards, we can
perform either a left or right rotation at #u#, which will replace #u#
with #u.right# or #u.left#, respectively.  Deciding whether to make a left or right rotation is done as follows:
\begin{enumerate}
\item If #u.left# and #u.right# are both #null#, then #u# is a leaf and no rotation is performed.
\item If #u.left# (or #u.right#) is #null# then perform a right (or left, respectively) rotation at #u#
\item If $#u.left.p# < #u.right.p#$ (or $#u.left.p# > #u.right.p#)$ then perform a right rotation at #u# (or left rotation, respectively, at #u#).
\end{enumerate}
These three rules ensure that the #Treap# doesn't become disconnected and that the heap property is maintained once #u# is removed.
\javaimport{ods/Treap.remove(x).trickleDown(u)}
An example of the #remove(x)# operation is shown in \figref{treap-remove}.
\begin{figure}
  \begin{center}
  \includegraphics{figs/treap-delete-a} \\
  \includegraphics{figs/treap-delete-b} \\
  \includegraphics{figs/treap-delete-c} \\
  \includegraphics{figs/treap-delete-d} 
  \end{center}
  \caption{Removing the value 9 from the treap in \figref{treap}.}
  \figlabel{treap-remove}
\end{figure}


The trick to analyzing the running time of the #remove(x)# operation is
to notice that this operation is the reverse of the #add(x)# operation.
In particular, if we were to reinsert #x#, using the same priority #u.p#,
then the #add(x)# operation would do exactly the same number of rotations
and would restore the #Treap# to exactly same state it was in before
the #remove(x)# operation took place.  (Reading from bottom-to-top,
\figref{treap-remove} illustrates the insertion of the value 9 into a
#Treap#.) This means that the expected running time of the #remove(x)#
on a #Treap# of size #n# is proportional to the expected running time
of the #add(x)# operation on a #Treap# of size $#n#-1$.  We conclude
that the running time of #remove(x)# is $O(\log #n#)$.

\subsection{Summary}

The following theorem summarizes the performance of the #Treap# data
structure:

\begin{thm}
A #Treap# implements the #SSet# interface. A #Treap# supports
the operations #add(x)#, #remove(x)#, and #find(x)# in $O(\log #n#)$
expected time per operation.
\end{thm}

It is worth comparing the #Treap# data structure to the #SkiplistSet#
data structure.  Both implement the #SSet# operations in $O(\log #n#)$
time per operation.  In both data structures, #add(x)# and #remove(x)#
involve a search and then a constant number of pointer changes
(see \excref{treap-rotates} below).  Thus, for both these
structures, the expected length of the search path is the critical value
in assessing their performance.  In a #SkiplistSet#, the expected length
of a search path is
\[
     2\log_2 #n# + O(1) \enspace ,
\]
In a treap, the expected length of a search path is 
\[
    2\ln #n# +O(1) \approx 1.386\log_2 #n#  + O(1) \enspace .
\]
Thus, the search paths in a #Treap# are considerably shorter and this
translates into noticeably faster operations on #Treap#s than #Skiplist#s.
\excref{skiplist-opt} in \chapref{skiplists} shows how the
expected length of the search path in a #Skiplist# can be reduced to
\[
     e\ln #n# + O(1) \approx 1.884\log_2 #n# + O(1) 
\]
by using biased coin tosses.  Even with this optimization, the expected
length of search paths in a #SkiplistSet# is noticeably longer than in
a #Treap#.

\section{Summary and Exercises}

Random binary search trees have been studied extensively.  Devroye
\cite{d88} gives a proof of \lemref{rbs} and related results.  However,
much stronger results than \lemref{rbs} are known.  The most impressive
such result is due to Reed \cite{r03} who shows that the expected height
of a random binary search tree is
\[
  \alpha\ln n - \beta\ln\ln n + O(1)
\]
where $\alpha\approx4.31107$ is the unique solution on $[2,\infty)$ of the
equation $\alpha\ln((2e/\alpha))=1$ and $\beta=\frac{3}{2\ln(\alpha/2)}$ .
Furthermore, the variance of the height is constant.

The name #Treap# was coined by Aragon and Seidel \cite{as96} who discussed
treaps and some of their variants.  However, the basic structure was
studied much earlier by Vuillemin \cite{v80} who called these Cartesian
trees.

One space-optimization of the #Treap# data structure that is sometimes
performed is to eliminate the explicit storage of the priority #p#
in each node. Instead, one computes the priority of a node #u# by
hashing #u#'s address in memory (in 32-bit Java, this is equivalent
to hashing #u.hashCode()#).  Although a number of hash functions will
probably work well for this in practice, for the important parts of the
proof of \lemref{rbs} to work, the hash function should be randomized
and have the \emph{min-wise independent property}:  For any distinct
values $x_1,\ldots,x_k$, each of the hash values $h(x_1),\ldots,h(x_k)$
should be distinct and, for each $i\in\{1,\ldots,k\}$,
\[
   \Pr\{h(x_i) = \min\{h(x_1),\ldots,h(x_k)\}\} \le c/k
\]
One such class of hash functions that is easy to implement and fairly
fast is \emph{tabulation hashing} \cite{pt10}.

\begin{exc}
 Prove the assertion that there are $21,964,800$ sequences that
generate the tree on the right hand side of \figref{rbs-lvc}.  (Hint:
Give a recursive formula for the number of sequences that generate a
complete binary tree of height $h$ and evaluate this formula for $h=3$.)
\end{exc}

\begin{exc}\exclabel{treap-rotates}
 Use both parts of \lemref{rbs-treap} to prove that the expected number of rotations performed by an #add(x)# operation (and hence also a #remove(x)# operation) is $O(1)$.
\end{exc}

\begin{exc}
 Design and implement a version of a #Treap# that includes a #get(i)# operation that returns the key with rank #i# in the #Treap#.  (Hint: Have each node #u# keep track of the size of the subtree rooted at #u#.)
\end{exc}

\begin{exc}
 Design and implement a version of a #Treap# that supports the
#split(x)# operation.  This operation removes all values from the #Treap#
that are greater than #x# and returns a second #Treap# that contains all
the removed values.  

For example, the code #t2 = t.split(x)# removes from #t# all values
greater than #x# and returns a new #Treap# #t2# containing all these
values.  The #split(x)# operation should run in $O(\log #n#)$ expected
time.
\end{exc}

