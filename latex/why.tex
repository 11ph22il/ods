\chapter*{Why This Book?}

There are plenty of books that teach introductory data structures.
Some of them are very good.  Most of them cost money, and the vast
majority of computer science undergraduate students will shell-out at
least some cash on a data structures book.

Several free data structures books are available online.  Some are very
good, but most of them are getting old.  The majority of these books
became free when their authors and/or publishers decided to stop updating
them.  Updating these books is usually not possible, for two reasons:
(1)~The copyright belongs to the author and/or publisher, either of whom
may not allow it.  (2)~The \emph{source code} for these books is often
not available.  That is, the Word, WordPerfect, FrameMaker, or \LaTeX\
source for the book is not available, and the version of the software
that handles this source may not even be available.

The goal of this project is to free undergraduate computer science
students from having to pay for an introductory data structures book.
I have decided to implement this goal by treating this book like an Open
Source software project.  The \LaTeX\ source, \lang\ source, and build
scripts for the book are available to download from the book's website
(\url{opendatastructures.org}) and also, more importantly, on a reliable
source code management site (\url{github.com/patmorin/ods}).

This source code is released under a Creative Commons Attribution license,
meaning that anyone is free to \emph{share}: to copy, distribute and
transmit the work; and to \emph{remix}: to adapt the work, including the
right to make commercial use of the work.  The only condition on these
rights is \emph{attribution}: you must acknowledge that the derived work
contains code and/or text from \url{opendatastructures.org}.

Anyone can contribute corrections/fixes using the \texttt{git} source-code
management system.  Anyone can also fork the current version of the
book to develop a separate version (for example, in another programming
language).  That individual can then request that his or her changes be
merged back into my version.  My hope is that, by doing things this way,
this book will continue to be a useful textbook long after my interest
in the project or my pulse, (whichever comes first) has waned.


