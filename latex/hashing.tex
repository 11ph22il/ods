\chapter{Hash Tables}
\chaplabel{hashtables}

Hash tables are an efficient method of storing a small number,
#n#, of integers from a large range $U=\{0,\ldots,2^{#w#}-1\}$.
The term \emph{hash table} includes a broad range of data structures.
This chapter focuses on one of the most common implementations of hash
tables, namely hashing with chaining.

Very often hash tables store data that is not integers.  In this case,
an integer \emph{hash code} is associated with each data item and this
hash code is used in the hash table.  The second part of this chapter
discusses how such hash codes are generated.


\chaplabel{hashing}
\section{HashTable: Hashing with Chaining}

A #HashTable# data structure uses \emph{hashing with chaining} to store
data as an array #t# of lists.  An integer, #n#, keeps track of the
total number of items in all lists:
\javaimport{ods/HashTable.t.n}
The \emph{hash value} of a data item #x#, denoted #hash(x)# is a value in the range $\{0,\ldots,#t.length#-1\}$.  All items with hash value #i# are stored in the list at #t[i]#.
To ensure that lists don't get too long, we maintain the invariant that
\[
    #n# \le #t.length#
\]
so that the average number of elements stored in one of these lists is 
$#n#/#t.length# \le 1$.

To add an element #x# to the hash table, we first check if the length
of #t# needs to be increased and, if so, we grow #t#.  With
this out of the way we hash #x# to get an integer #i# in the range
$\{0,\ldots,#t.length#-1\}$ and we append #x# to the list #t[i]#:
\javaimport{ods/HashTable.add(x)}
Growing the table, if necessary, involves doubling the lenght of #t#
and reinserting all elements into the new table.  This is exactly the
same strategy used in the implementation of #ArrayStack# and the same
result applies: The cost of growing is only constant when amortized
over a sequence of insertions (See~\lemref{arraystack-amortized} on
Page~\pageref{lem:arraystack-amortized}).

Besides growing, the only other work done when adding #x# to a
#HashTable# involves appending #x# to the #t[hash(x)#.  For any of
the list implementations described in Chapters~\ref{chap:arrays} or
\ref{chap:linkedlists}, this takes only constant time.

To remove an element #x# from the hash table, we again hash it to get
an integer #i# and we iterate over the list #t[i]# until we find #x# so that we can remove it:
\javaimport{ods/HashTable.remove(x)}
\javaimport{ods/HashTable.removeOne(x)}
This takes $O(#n#_#i#)$time, where $#n#_#i#$ is the the length of the
list stored at #t[i]#.

Searching for the element #x# in a hash table is similar.  We hash #x#
to get an integer #i# and and then perform a linear search on the list
#t[i]#:
\javaimport{ods/HashTable.find(x)}
Again, this takes time proportional to the length of the list #t[i]#.

The performance of a hash table depends critically on the choice of the
hash function.  A good hash function will spread the elements evenly
among the #t.length# lists, so that the expected size of the list
#t[hash(x)]# is $O(#n#/#t.length)# = O(1)$.  On the other
hand, a bad hash function will hash all values (including #x#) to the
same table location, in which case the size of the list #t[hash(x)]#
will be #n#.  In the next section we describe a good hash function.

\subsection{Multiplicative Hashing}
\seclabel{multihash}

Multiplicative hashing is an efficient method of getting hash values
that is based on modular arithmetic (discussed in \secref{arrayqueue})
and integer division.  It uses the $\ddiv$ operator, which does division
without remainder.  For any integers $a\ge 0$ and $b\ge 1$, $a\ddiv b =
\lfloor a/b\rfloor$.

In multiplicative hashing, we use a hash table of size $2^{#d#}$ for some
integer #d# (called the \emph{dimension}).  The formula for hashing an
integer $#x#\in\{0,\ldots,2^{#w#}-1\}$ is
\[
    #hash#(#x#) = ((#z#\cdot#x#) \bmod 2^{#w#}) \ddiv 2^{#w#-#d#} \enspace .
\]
Here #z# is a randomly chosen \emph{odd} integer in
$\{1,\ldots,2^{#w#}-1\}$.
This hash function can be realized very
efficiently by observing that, by default, operations on integers
are already done modulo $2^{#w#}$ where $#w#$ is the number of bits in
an integer.  (See \figref{multihashing}.) Furthermore, integer division
by $2^{#w#-#d#}$ is equivalent to dropping the rightmost $#w#-#d#$ bits in
a binary representation (which is implemented by shifting the bits
right by $#w#-#d#$).  In this way, the code that
implements the above formula is simpler than the formula itself:
\javaimport{ods/HashTable.hash(x)}

\begin{figure}
  \begin{center}
    \begin{tabular}{|lr@{}r|}\hline
    $2^#w#$ (4294967296)&            #1#&#00000000000000000000000000000000# \\
    #z# (4102541685)&                   &#11110100100001111101000101110101# \\
    #x# (42) &                          &#00000000000000000000000000101010# \\
    $#z#\cdot#x#$ &             #101000#&#00011110010010000101110100110010# \\
    $(#z#\cdot#x#)\bmod 2^{#w#}$ &      &#00011110010010000101110100110010# \\
    $((#z#\cdot#x#)\bmod 2^{#w#})\ddiv 2^{#w#-#d#}$ &&
                      \multicolumn{1}{@{}l|}{#00011110#} \\\hline
    \end{tabular}
  \end{center}
  \caption{The operation of the multiplicative hash function with $#w#=32$
    and $#d#=8$.}
  \figlabel{multihashing}
\end{figure}

The following lemma, whose proof is deferred until later in this section,
shows that multiplicative hashing does a good job of avoiding collisions:

\begin{lem}\lemlabel{universal-hashing}
  Let #x# and #y# be any two values in $\{0,\ldots,2^{#w#}-1\}$ with
  $#x#\neq #y#$. Then $\Pr\{#hash(x)#=#hash(y)#\} \le 2/2^{#d#}$.
\end{lem}

With \lemref{universal-hashing}, the performance of #remove(x)#, and
#find(x)# are easily to analyze:

\begin{lem}
  For any data value #x#, the expected length of the list #t[hash(x)]#
  is at most $#n#_{#x#} + 2$, where $#n#_{#x#}$ is the number of
  occurrences of #x# in the hash table.
\end{lem}

\begin{proof}
  Let $S$ be the (multi-)set of elements stored in the hash table that
  are not equal to #x#.  For an element $#y#\in S$, define the indicator
  variable
    \[ I_{#y#} = \left\{\begin{array}{ll}
       1 & \mbox{if $#hash(x)#=#hash(y)#$} \\
       0 & \mbox{otherwise}
       \end{array}\right.
    \]
  and notice that, by \lemref{universal-hashing}, $\E[I_{#y#}] \le
  2/#t.length#$.  The expected length of the list #t[hash(x)]#
  is given by
  \begin{eqnarray*}
   \E\left[|#t[hash(x)]#|\right] &=& \E\left[#n#_{#x#} + \sum_{#y#\in S} I_{#y#}\right] \\
    &=& #n#_{#x#} + \sum_{#y#\in S} \E [I_{#y#} ] \\
    &\le& #n#_{#x#} + \sum_{#y#\in S} 2/#t.length# \\
    &\le& #n#_{#x#} + \sum_{#y#\in S} 2/#n# \\
    &\le& #n#_{#x#} + (#n#-#n#_{#x#})2/#n# \\
    &\le& #n#_{#x#} + 2 \enspace ,
  \end{eqnarray*}
  as required.
\end{proof}

Now, we want to prove \lemref{universal-hashing}, but we first need
a number-theory result.  In the following proof, we use the notation
$(b_r,\ldots,b_0)_2$ to denote $\sum_{i=0}^r b_i2^i$, i.e,. the integer
whose binary representation is given by $b_r,\ldots,b_0$.  Each bit $b_i$
is either a 0 or a 1.  We use $\star$ to denote a bit whose value we
don't know.

\begin{lem}\lemlabel{hashing-mapping}
  Let $S$ be the set of odd integers in $\{1,\ldots,2^{#w#}-1\}$, Let $q$
  and $i$ be any two elements in $S$.  Then there is exactly one value
  $#z#\in S$ such that $#z#q\bmod 2^{#w#} = i$.
\end{lem}

\begin{proof}
  Since the number of choices for $#z#$ and $i$ are the same, it is
  sufficient to prove that there is at most one value $#z#\in S$ that
  satisifies $#z#q\bmod 2^{#w#} = i$.

  Suppose, for the sake of contradiction, that there are two such values
  #z# and #z'#, with $#z#>#z#'$.  Then
  \[
     #z#q\bmod 2^{#w#} = #z#'q \bmod 2^{#w#} = i
  \]
  So
  \[ 
     (#z#-#z#')q\bmod 2^{#w#} = 0 
  \]
  But this means that 
  \begin{equation}
    (#z#-#z#')q = k 2^{#w#} \eqlabel{factors} 
  \end{equation}
  for some integer $k$.  Thinking in terms of binary numbers, we have 
  \[
    (#z#-#z#')q = k\cdot(1,\underbrace{0,\ldots,0}_{#w#})_2 \enspace ,
  \]
  so that the $w$ trailing bits in the binary representation of
  $(#z#-#z#')q$ are all 0's.

  Furthermore $k\neq 0$ since $q\neq 0$ and $#z#-#z#'\neq 0$.  Since $q$
  is odd, it has no trailing 0's in its binary representation:
  \[
    q = (\star,\ldots,\star,1)_2 \enspace .
  \]
  Since $|#z#-#z#'| < 2^{#w#}$, $#z#-#z#'$ has fewer than #w# trailing
  0's in its binary representation:
  \[
    #z#-#z#' = (\star,\ldots,\star,1,\underbrace{0,\ldots,0}_{<#w#})_2
      \enspace .
  \]
  Therefore, the product $(#z#-#z#')q$ has fewer than #w# trailing 0's in
  its binary representation:
  \[
   (#z#-#z#')q = (\star,\cdots,\star,1,\underbrace{0,\ldots,0}_{<#w#})_2 
    \enspace .
  \]
  Therefore $(#z#-#z#')q$ cannot satisfy \eqref{factors}, yielding a
  contradiction and completing the proof.
\end{proof}

The utility of \lemref{hashing-mapping} comes from the following
observation:  If #z# is chosen uniformly at random from $S$, then #zt#
is uniformly distributed over $S$.  In the following proof, it helps
to think of the binary representation of #z#, which consists of $#w#-1$
random bits followed by a 1.

\begin{proof}[Proof of \lemref{universal-hashing}]
First we note that the condition #hash(x)=hash(y)# is equivalent to
the statement ``the highest-order #d# bits of $#z##x#\bmod2^{#w#}$
and the highest-order #d# bits of $#z##y#\bmod 2^{#w#}$ are the same.''
Equivalently, the highest-order #d# bits in the binary representation
of $#z#(#x#-#y#)\bmod 2^{#w#}$ are either all 0's or all 1's.  That is,
\begin{equation}
    #z#(#x#-#y#)\bmod 2^{#w#} = 
    (\underbrace{0,\ldots 0}_{#d#},\underbrace{\star,\ldots,\star}_{#w#-#d#})_2 
    \eqlabel{all-zeros}
\end{equation}
when $#zx#\bmod 2^{#w#} \ge #zy#\bmod 2^{#w#}$ or
\begin{equation}
    #z#(#x#-#y#)\bmod 2^{#w#} = 
    (\underbrace{1,\ldots 1}_{#d#},\underbrace{*,\ldots,*}_{#w#-#d#})_2 
     \enspace .
    \eqlabel{all-ones}
\end{equation}
when $#zx#\bmod 2^{#w#} < #zy#\bmod 2^{#w#}$.
Therefore, we only have to bound the probability that 
$#z#(#x#-#y#)\bmod 2^{#w#}$ looks like \eqref{all-zeros} or \eqref{all-ones}.

Let $q$ be the unique odd integer such that $#x#-#y#=q2^r$ for some integer
$r\ge 0$. By \lemref{hashing-mapping}, the binary representation of
$#z#q\bmod 2^{#w#}$ has $#w#-1$ random bits, followed by a 1:
\[
 #z#q\bmod 2^{#w#}  = (\underbrace{b_{#w#-1},\ldots b_{1}}_{#w#-1},1)_2
\]
Therefore, the binary representation of $#z#(#x#-#y#)\bmod 2^{#w#}=#z#q2^r\bmod 2^{#w#}$ has
$#w#-r-1$ random bits, followed by a 1, followed by $r$ 0's:
\[
#z#(#x#-#y#)\bmod 2^{#w#}  =
#z#q2^{r} =
    (\underbrace{b_{#w#-r-1},\ldots b_{1}}_{#w#-r-1},1,\underbrace{0,0,\ldots,0}_{r})_2
\]
We can now finish the proof:  If $r > #w#-#d#$, then the #d#
higher order bits of $#z#(#x#-#y#)\bmod 2^{#w#}$  contain both 0's
and 1's, so the probability that $#z#(#x#-#y#)\bmod 2^{#w#}$ looks
like \eqref{all-zeros} or \eqref{all-ones} is 0.  If $#r#=#w#-#d#$,
then the probability of looking like \eqref{all-zeros} is 0, but the
probability of looking like \eqref{all-ones} is $1/2^{#d#-1}=2/2^{#d#}$
(since we must have $b_1,\ldots,b_{d-1}=0,\ldots,0$).  If $r < #w#-#d#$
then we must have $b_{#w#-r-1},\ldots,b_{#w#-r-#d#}=0,\ldots,0$ or
$b_{#w#-r-1},\ldots,b_{#w#-r-#d#}=1,\ldots,1$.  The probability of each
of these cases is $1/2^{#d#}$ and they are mutually exclusive, so the
probability of either of these cases is $2/2^{#d#}$.  This completes
the proof.
\end{proof}

\subsection{Summary}



\section{Hash Codes}

The hash tables discussed in the previous section are used to associate
data with integer keys consisting of $2^{#w#}$ bits.  In many cases, we
have keys that are not integers.  They may be strings, objects, arrays,
or other compound structures.  To use hash tables for these types of data,
we must map these data types to hash codes.  Hash codes should have the
following properties:

\begin{enumerate}
\item If #x# and #y# are equal, then #x.hashCode()# and #y.hashCode()#
are equal.

\item If #x# and #y# are not equal, then the probability that
#x.hashCode()# equals #y.hashCode()# should be small.
\end{enumerate}

The first property ensure that if we store #x# in a hash table and later
lookup a value #y# equal to #x# then we will find #x#, as we should.
The second requirement ensures that we don't lose anything by converting
our objects to integers. Unequal objects have different hash codes and
are likely to be at different locations in our hash table.

\subsection{Hash Codes for Primitive Data Types}

Small primitive data types like #char#, #byte#, #int#, and #float# are
usually easy to find hash codes for.  These data types always have a
binary representation and this binary representation usually consists
of #w# or fewer bits. (For example, in Java, #byte# is an 8-bit type
and #float# is a 32 bit type.) In these cases, we just treat these bits
as the representation of an integer in the range $\{0,\ldots,#w#\}$.
If two values are different, they get different hash codes.  If they
are the same, they get the same hash code.

A few primitive data types are made up of more than #w# bits, usually
$c#w#$ bits for some constant integer $c$. (Java's #long# and #double#
types are examples of this.)  These data types can be treated as compound
objects made of $c$ parts, as described in the next section.

\subsection{Hash Codes for Compound Objects}

For a compound object, we want to create a hash code by combining the
individual hash codes of the object's parts.  This is not as easy as
it sounds.  Although one can find many hacks for this (for example,
combining them with bitwise exclusive-or); many of these hacks turn out
to be easy to foil.

However, if one is willing to do arithmetic in a larger space, then things
become easier.  Suppose we have an object #o# made up of several parts
$P_0,\ldots,P_{r-1}$ whose hash codes are $#x#_0,\ldots,#x#_{r-1}$.
Then we can choose random #w#-bit integers $#z#_0,\ldots,#z#_{r-1}$
and a random $2#w#$-bit odd integer #z# and compute a hash code for our
object as
\[
   h(#x#_0,\ldots,#x#_{r-1}) =  
   \left(#z#\left(\sum_{i=0}^{r-1} #z#_i #x#_i\right)\bmod 2^{2#w#}\right)
   \ddiv 2^{#w#}
\]
Note that this hash code has a final step (multiplying by #z# and
dividing by $2^{#w#}$) that uses the multiplicative hash function
from \secref{multihash} to take the $2#w#$-bit intermediate result and
reduce it to a #w#-bit final result.  Here is an example of this method applied to a simple compound object with 3 parts #x0#, #x1#, and #x2#:
\javaimport{junk/Point3D.x0.hashCode()}
The following theorem shows that, in addition to being straightforward to implement, this method is provably good:

\begin{thm}\thmlabel{multihash}
Let $#x#_0,\ldots,#x#_{r-1}$ and $#y#_0,\ldots,#y#_{r-1}$ each be sequences of #w# bit integers in $\{0,\ldots,2^{#w#}-1\}$ and assume $#x#_i \neq #y#_i$ for at least one index $i\in\{0,\ldots,r-1\}$. Then 
\[
   \Pr\{ h(#x#_0,\ldots,#x#_{r-1}) =  h(#y#_0,\ldots,#y#_{r-1}) \} 
        \le 3/2^{#w#} \} \enspace .  
\] 
\end{thm}

\begin{proof}
  We will first ignore the final multiplicative hashing step and see how
  that step contributes later.  Define:
  \[
    h'(#x#_0,\ldots,#x#_{r-1}) =  
       \left(\sum_{i=0}^{r-1} #z#_i #x#_i\right)\bmod 2^{2#w#} \enspace .
  \]
  Suppose that $h'(#x#_0,\ldots,#x#_{r-1}) =  h'(#y#_0,\ldots,#y#_{r-1})$.
  We can rewrite this as:
  \begin{equation}  \eqlabel{bighash-x}
      #z#_i(#x#_i-#y#_i) \bmod 2^{2#w#} = t
  \end{equation}
  where 
  \[
     t = \left(\sum_{j=0}^{i-1} #z#_j(#y_j#-#x#_j) + \sum_{j=i+1}^{r-1} #z#_j(#y_j#-#x#_j)\right) \bmod 2^{2#w#}
  \]
  If we assume, without loss of generality that $#x#_i> #y#_i$, then
  \eqref{bighash-x} becomes
  \begin{equation}
      #z#_i(#x#_i-#y#_i) = t \eqlabel{bighash-xx}
  \end{equation}
  since each of $#z#_i$ and $(#x#_i-#y#_i)$ are at most $2^{#w#}-1$,
  so their product is at most $2^{2#w#}-2^{#w#+1}+1 < 2^{2#w#}-1$.
  By assumption, $#x#_i-#y#_i\neq 0$, so \eqref{bighash-xx} has only one
  solution in #z_i#.  Therefore, the probability that we select $z_i$
  so that $h'(#x#_0,\ldots,#x#_{r-1})=h'(#y#_0,\ldots,#y#_{r-1})$ is at most
  $1/2^{#w#}$.

  The final step of the hash function is to apply multiplicative hashing
  to reduce our $2#w#$ bit intermediate result $h'(#x#_0,\ldots,#x#_{r-1})$ to
  a #w# bit final result $h(#x#_0,\ldots,#x#_r)$.  By \thmref{multihash},
  if $h'(#x#_0,\ldots,#x#_r)\neq h'(#y#_0,\ldots,#y#_r)$, then
  $\Pr\{h(#x#_0,\ldots,#x#_r) = h(#y#_0,\ldots,#y#_r)\} \le 2/2^{#w#}$.

  To summarize, 
  \begin{eqnarray*}
    \Pr\{h(#x#_0,\ldots,#x#_r) = h(#y#_0,\ldots,#y#_r)\}
      &=& \Pr\left\{\begin{array}{ll}
            \mbox{$h'(#x#_0,\ldots,#x#_r) = h'(#y#_0,\ldots,#y#_r)$ or} \\
            \mbox{$h'(#x#_0,\ldots,#x#_r) \neq h'(#y#_0,\ldots,#y#_r)$} \\
                  \mbox{\quad and
$#z#h(#x#_0,\ldots,#x#_r)\ddiv2^{#w#} = #z# h'(#y#_0,\ldots,#y#_r)\ddiv 2^{#w#}$}
          \end{array}\right\} \\
      &\le& 1/2^{#w#} + 2/2^{#w#} = 3/2^{#w#}
  \end{eqnarray*}
  and this completes the proof.
\end{proof}


\subsection{Hash Codes for Arrays and Strings}

The method from the previous section works well for objects that have a
fixed, constant, number of components.  However, it breaks down when we
want to use it with objects that have a variable number of components
since it requires a random #w#-bit integer $#z#_i$ for each component.
We could use a pseudorandom sequence to generate as many $#z#_i$'s
as we need, but then the $#z#_i$'s are not mutually independent,
and it becomes difficult to prove that the pseudorandom numbers don't
interact badly with the hash function we are using.  In particular,
it becomes hard, or impossible, to give a non-trivial upper-bound on
$\Pr\{h'(#x#_0,\ldots,#x#_r) = h'(#y#_0,\ldots,#y#_r)\}$.

A more rigorous approach is to base our hash codes on polynomials over
prime fields.  This is based on the following Theorem, which says that
polynomials over prime fields behave pretty-much like usual polynomials:

\begin{thm}\thmlabel{prime-polynomial}
 Let $#p#$ be a prime number, and let $f(#z#) = #x#_{r-1}#z#^{r-1} +
 #x#_{r-1}#z#^{r-1} +\cdots + #x#_0#z#^0)$ be a non-trivial polynomial
 with coefficients in $\{0,\ldots,#p#-1\}$. Then the equation
 $f(#z#)\bmod #p# = 0$ has at most $r+1$ solutions.
\end{thm}

To use \thmref{prime-polynomial}, we hash a sequence of integers
$#x#_0,\ldots,#x#_{r-1}$ with each $#x#_i\in \{0,\ldots,#p#-2\}$ using
a random integer $#z#\in\{0,\ldots,#p#-1\}$ via the formula
\[
   h(#x#_0,\ldots,#x#_{r-1}) 
    = \left(#x#_0#z#^0+\cdots+#x#_{r-1}#z#^{r-1}+(#p#-1)#z#^r \right)\bmod #p# \enspace .
\]

The following theorem shows that this hash function gives a good return for the small amount of randomization needed to choose #z#:

\begin{thm}\thmlabel{stringhash}
Let $#p#>2^{#w#}$ be a prime,
let $#x#_0,\ldots,#x#_{r-1}$ and $#y#_0,\ldots,#y#_{r-1}$ each be sequences of #w# bit integers in $\{0,\ldots,2^{#w#}-1\}$, and assume $#x#_i \neq #y#_i$ for at least one index $i\in\{0,\ldots,r-1\}$. Then 
\[
   \Pr\{ h(#x#_0,\ldots,#x#_{r-1}) =  h(#y#_0,\ldots,#y#_{r-1}) \} 
        \le r/#p# \} \enspace .  
\] 
\end{thm}

\begin{proof}
The equation $h(#x#_0,\ldots,#x#_{r-1}) =  h(#y#_0,\ldots,#y#_{r-1})$
can be rewritten as
\[
  \left((#x#_0-#y#_0)#z#^0+\cdots+(#x#_{#r#-1}-#x#_{#r#-1})#z#^{#r#-1}\right)\bmod #p# = 0.
\]
Since $#x_i#\neq #y_i#$, this polynomial is non-trivial.  Therefore,
by \thmref{prime-polynomial} it has at most $r$ solutions in #z#.
The probability that we pick #z# to be one of these solutions is therefore
at most $r/#p#$.
\end{proof}

Note that this hash function deals with the case in which two sequences
have different lengths, even when one of the sequences is a prefix of
the other. This is because this function effectively hashes the sequence
\[
#x#_0,\ldots,#x#_{r-1}, #p#-1 \enspace .
\]
This guarantees that if we have two sequences of length #r# and #r'#
with $r < r'$, then these two sequences differ at index #i=r#.  We can
think of $#p#-1$ as an end-of-sequence marker.

The following example code shows how this is done for an object that
contains an array #x# of values:
\javaimport{junk/GeomVector.hashCode()}

The above code sacrifices some collision probability for implementation
convenience.  In particular, the multiplier #z# was chosen randomly
from $\{0,\ldots,2^{30}-1\}$ to avoid overflow with 63 bit arithmetic.
This means tha the probability of two different sequences of length $r$
having the same hash code is $r/2^{30}$ rather than the $r/(2^32+15)$
specified in \thmref{stringhash}.

\section{Summary}

Hash tables and hash codes are a enormous and active area of research
that is just touched upon in this chapter.

\section{Exercises}

\begin{enumerate}
\item Prove that the bound $2/2^{#d#}$ in \lemref{universal-hashing} is
the best possible by showing that, if $x=2^{#w#-#d#-2}$ and $#y#=3#x#$,
then $\Pr\{#hash(x)#=#hash(y)#\}=2/2^{#d#}$.  (Hint look at the binary
representations of #zx# and #z#3#x# and use the fact that #z#3#x# =
#z##x#+2#z##x#.)

\item Suppose you have an object made up of two #w#-bit integers #x# and #y#.  Show why $#x#\oplus#y#$ does not make a good hash code for your object.  Give an example of a set of objects that all have hash code 0.

\item Suppose you have an object made up of two #w#-bit integers #x# and #y#.  Show why $#x#+#y#$ does not make a good hash code for your object.
Give an example of a large set of objects that all have the same hash code.

\item Suppose you have an object made up of two #w#-bit integers #x#
and #y#.  Suppose that the hash code for your object is defined by some
deterministic function $h(x,y)$.  Prove that there exists a large set
of objects that have the same hash code.


\end{enumerate}
